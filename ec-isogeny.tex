\documentclass[]{article}

\usepackage[english]{babel}
\usepackage[utf8]{inputenc}
\usepackage{bbm}
\usepackage{amsmath}
\usepackage{amssymb}
\usepackage{mathrsfs}
\usepackage{mdwlist}
\usepackage{graphicx}
\usepackage{mathrsfs}
\usepackage{url}
\usepackage{color}

%%\usepackage[pdftex,plainpages=false]{hyperref}
%%\usepackage[pdftex]{color,graphicx}
%%\usepackage{ifthen}
%%\usepackage[all]{xy}

%% Stuff
\renewcommand{\le}{\leqslant}
\renewcommand{\ge}{\geqslant}  % comme François le demande...
%% Algèbre
\newcommand{\trans}[1]{#1^\top}  % transposé
\newcommand{\dual}[1]{#1^\ast}  % dual
\newcommand{\clot}[1]{\bar{#1}}  % clôture algèbrique
\newcommand{\card}[1]{\# #1}  % cardinalité d'un ensemble
\DeclareMathOperator{\car}{car}  % caractéristique d'un corps
\DeclareMathOperator{\Frac}{Frac}  % corps des fractions
\newcommand{\N}{\mathbb{N}}  % les naturels
\newcommand{\Z}{\mathbb{Z}}  % les entiers
\newcommand{\K}{\mathbb{K}}  % un corps
\newcommand{\LK}{\mathbb{L}}  % encore un corps
\newcommand{\U}{\mathbb{U}}  % encore un corps
\newcommand{\F}{\mathbb{F}}  % un corps fini
\newcommand{\Q}{\mathbb{Q}}  % les rationnels
\newcommand{\R}{\mathbb{R}}  % les réels
\newcommand{\C}{\mathbb{C}}  % les complexes
\newcommand{\isom}{\cong}  % isomorphisme de corps
\newcommand{\frob}{\varphi}  % fröbenius
\DeclareMathOperator{\Gal}{Gal}  % groupe de Galois
\DeclareMathOperator{\Tr}{Tr}  % trace
\DeclareMathOperator{\PTr}{T}  % pseudotrace
\DeclareMathOperator{\Norm}{N} % norme
\DeclareMathOperator{\coeff}{coeff}  % coefficient
\newcommand{\res}{\rho}  % the residue form
\newcommand{\euler}{\phi}  % indicatrice d'Euler
\DeclareMathOperator{\ord}{ord}  % l'ordre d'un élément
\DeclareMathOperator{\rev}{rev}  % le reverse d'un polynôme
\newcommand{\Cyclo}{\Phi}  % le polynome cyclotomique
\DeclareMathOperator{\Fix}{Fix}  % les points fixes
%% Courbes
\newcommand{\Proj}{\mathbb{P}}  % espace projectif
\newcommand{\0}{\mathcal{O}}  % point de base d'une courbe
\newcommand{\ecpoint}[3]{[#1:#2:#3]}  % un point d'une courbe
\newcommand{\isog}[1]{\mathcal{#1}}  % la police des isogénies
\newcommand{\I}{\isog{I}}  % une isogénie I
\newcommand{\Hasse}{H}  % l'invariant de Hasse
\newcommand{\divpol}{f}  % polynôme de division
%% Autre
\newcommand{\tildO}{\tilde{O}}  % la notation O~ qui oublie les log
\newcommand{\Mult}{\mathrm{\sf M}}  % fonction de multiplication
\newcommand{\ModComp}{\mathrm{\sf C}}  % fonction de composition modulaire
\newcommand{\alg}[1]{{\sf #1}}  % la police des algorithmes
\newcommand{\algref}[1]{\alg{\ref{#1}}}  % la police des algorithmes
\newcommand{\wrt}{\dashv}  % appartenance forte, a\wrt A signifie que a est représenté comme un élément de A
\newcommand{\ndiv}{\nmid}  % ne divise pas


%%Théorèmes
\newtheorem{definition}{Definition}
\newtheorem{theorem}[definition]{Theorem}
\newtheorem{lemma}[definition]{Lemma}
\newtheorem{corollary}[definition]{Corollary}
\newtheorem{proposition}[definition]{Proposition}
\newtheorem{remark}[definition]{Remark}

\def\myproof{\vspace{-\parsep}\penalty-51\noindent{\sc Proof.}~}
\def\foorp{\hfill$\square$}


\newenvironment{example}[1][]{\begin{flushleft}\noindent\begin{minipage}{\textwidth}
    \textbf{Example\ifthenelse{\equal{#1}{}}
      {}
      { (#1)}~}
}{\end{minipage}\end{flushleft}}


\newenvironment{algorithm}[3]{\small\begin{center}\begin{minipage}{0.48\textwidth}
      \sf
      \rule{\textwidth}{0.2pt}\\
      \makebox[\textwidth][c]{\textbf{#1}}\\
      \rule[0.5\baselineskip]{\textwidth}{0.2pt}\\

      \vspace{-15pt}

      \parbox{\textwidth}{\textbf{Input} #2}
      \parbox{\textwidth}{\textbf{Output} #3}

      \vspace{-7pt}

      \begin{enumerate*}}{\end{enumerate*}
      \vspace{-11pt}
      \rule{\textwidth}{0.2pt}
  \end{minipage}\end{center}
}


\begin{document}

\title{Fast algorithms for computing isogenies between elliptic curves
  in small characteristic}
\author{Luca De Feo\footnote{LIX, {\'E}cole
Polytechnique, 91128 Palaiseau, France}}

\maketitle
\begin{abstract}
  We improve Couveignes' algorithm \cite{Cou96}.
\end{abstract}

\section{Introduction}

The problem of computing an explicit degree $\ell$ isogeny between two
given elliptic curves over $\F_q$ was originally motivated by point
counting methods based on Schoof's algorithm \cite{Atk91},
\cite{Elk91}, \cite{Sch95}. A review of the most efficient algorithms
to solve this problem is given in \cite{BoMoSaSc08} together with a
new quasi-optimal algorithm; however, all the algorithms presented in
\cite{BoMoSaSc08} compute the power series expansion of the
Weierstrass functions of the curves up to a certain precision, thus
they are limited to the case $\ell\ll p$ where $p$ is the
characteristic of $\F_q$. This is satisfactory for cryptographic
applications where one takes $p=q$ or $p=2$; indeed in the former case
Schoof's algorithm needs $\ell\in O(\log p)$, while in the latter case
there's no need to compute explicit isogenies since $p$-adic methods
based on \cite{Sat00} are preferred to Schoof's algorithm.

Nevertheless, the problem of computing explicit isogenies in the case
where $p$ is small compared to $\ell$ stays of theoretical interest
and can find practical applications in newer cryptosystems such as
\cite{Tes06}, \cite{RoSt06}. The first algorithm to solve this problem
was given by Couveignes and made use of formal groups \cite{Cou94}; it
takes $\tildO(\ell^3\log q)$ operations in $\F_p$ assuming $p$ is
constant, however it has an exponential complexity in $\log
p$. Another algorithm by Lercier specific to $p=2$ uses some linear
properties of the problem to build a linear system from whose solution
the isogeny can be deduced \cite{Ler96}; its complexity is conjectured
to be $\tildO(\ell^3\log q)$ operations in $\F_p$, but it has a much
better constant factor than \cite{Cou94}. At the moment we write the
latter algorithm is by many orders of magnitude the fastest algorithm
to solve practical instances of the problem when $p=2$, thus being the
\emph{de facto} standard for cryptographic use.

$p$-adic methods were used by Joux and Lercier \cite{JL06} and Lercier
and Sirvent \cite{LeSi09} to solve the isogeny problem. The former
method has complexity $\tildO(\ell^2(1 + \ell/p)\log q)$ operations in
$\F_p$, which makes it well adapted to the case where $p\sim\log
q$. The latter has complexity $\tildO(\ell^3 + \ell\log q^2)$
operations in $\F_p$, making it the best algorithm to our knowledge
for the case where $p$ is not constant.

\paragraph{The algorithm C2 and its variants}
Finally, the algorithm having the best asymptotic complexity in $\ell$
was proposed again by Couveignes in \cite{Cou96}, we will refer to
this original version as ``C2''. Its complexity --supposing $p$ is
fixed-- was estimated in \cite{Cou96} as being $\tildO(\ell^2\log q)$
operations in $F_p$, but with a precomputation step requiring
$\tildO(\ell^3\log q)$ operations and large memory
requirements. However this estimate was wrong as we will argue in
Section \ref{sec:C2} and C2 has an overall asymptotic complexity of
$\tildO(\ell^3\log q)$ operations.

Subsequent work by Couveignes \cite{Cou00} and more recently
\cite{DFS09}, use Artin-Schreier theory to avoid the precomputation
step of C2 and drop the memory requirements to $\tildO(\ell\log q +
\log^2 q)$ elements of $\F_p$. However, this is still not enough to
reduce the overall complexity of the algorithm as we will argue in
Section \ref{sec:C2-AS}. We refer to this variant as ``C2-AS''.

In the present paper we give a complete review of Couveignes'
algorithm, we present new variants that reach the foreseen quadratic
bound in $\ell^2$ and prove the accurate complexity estimate of
$\tildO(\todo)$ operations which doesn't suppose $p$ to be
fixed. In Section \ref{sec:} we also extend the algorithm beyond its
original scope to solve the more general problem of \emph{bounded
  degree isogeny finding}; interestingly, this new algorithm turns out
to have a better asymptotic complexity than any previously known
approach.

\paragraph{Notation and plan}
In the rest of the paper $p$ is a prime, $d$ a positive integer,
$q=p^d$ and $\F_q$ is the field with $q$ elements. For an elliptic
curve $E$ and a field $\K$ embedded in an algebraic closure
$\clot{\K}$, we note by $E(\K)$ the set of $\K$-rational points and by
$E[m]$ the $m$-torsion subgroup of $E(\clot{\K})$. The group law on
the elliptic curve is noted additively, its zero is the point at
infinity, noted $\0$. For an affine point $P$ we note by $x(P)$ its
abscissa and by $y(P)$ its ordinate. We will restrict ourselves to the
case of ordinary elliptic curves, thus $E[p^k]\isom\Z/p^k\Z$.

Unless otherwise stated, all time complexities will be measured in
number of operations in $\F_p$ and all space complexities in number of
elements of $\F_p$; we do not assume $p$ to be constant. We use the
$O$, $\Theta$ and $\Omega$ notations to state respectively upper
bounds, tight bounds and lower bounds for asymptotic complexities. We
also use the notation $\tildO_x$ that forgets polylogarithmic factors
in the variable $x$, thus $O(xy\log x \log y)\subset\tildO_x(xy\log
y)\subset\tildO_{x,y}(xy)$. We simply note $\tildO$ when the variables
are clear from the context. We let $\Mult:\N\rightarrow\N$ be a
\emph{multiplication function}, such that polynomials of degree at
most $n$ with coefficients in $\F_p$ can be multiplied in $\Mult(n)$
operations, under the conditions of \cite[Ch. 8.3]{vzGG}. Typical
orders of magnitude are $O(n^{\log_23})$ for Karatsuba multiplication
or $O(n\log n\log\log n)$ for FFT multiplication. We let
$2<\omega\le3$ be the exponent of linear algebra, that is an integer
such that $n\times n$ matrices can be multiplied in $n^\omega$
operations.

The plan is the following.\\
Preliminaries on isogenies\\
C2\\
extension to $p=2$\\
complexity of C2\\
C2-AS\\
FAAST\\
interpolation\\
complexity of C2-AS-FI\\
modular composition\\
complexity of C2-AS-FI-MC\\
bounded degree problem\\
complexity of CBD-AS-FI-MC, comparison to other methods\\
implementation\\
benchmarks\\



% Local Variables:
% mode:flyspell
% ispell-local-dictionary:"british"
% mode:TeX-PDF
% TeX-master: "ec-isogeny"
% End:
%
% LocalWords:  Schreier Artin pseudotrace frobenius bivariate Joux Sirvent FFT
% LocalWords:  Couveignes isogenies Schoof isogeny cryptosystems Lercier
% LocalWords:  precomputation arithmetics polylogarithmic Karatsuba

\section{Preliminaries on Isogenies}

Let $E$ be an ordinary elliptic curve over the field $\F_q$, we
suppose it is given to us as the locus of zeroes of an affine
Weierstrass equation
\[y^2 + a_1xy + a_3y = x^3 + a_2x^2 + a_4x + a_6
\qquad a_1,\ldots,a_6\in\F_q\text{.}\]

\paragraph{Simplified forms} If $p>3$ it is well known that the curve
$E$ is isomorphic to a curve in the form
\begin{equation}
  \label{eq:weierstrass>3}
  y^2 = x^3 + ax + b
\end{equation}
and its $j$-invariant is $j(E) = \frac{1728(4a)^3}{16(4a^3 + 27b^2)}$.

When $p=3$, since $E$ is ordinary, it is isomorphic to a curve
\begin{equation}
  \label{eq:weierstrass=3}
  y^2 = x^3 + ax^2 + b
\end{equation}
and its $j$-invariant is $j(E) = -\frac{a^3}{b}$.

Finally, when $p=2$, since $E$ is ordinary, it is isomorphic to a curve
\begin{equation}
  \label{eq:weierstrass=2}
  y^2 + xy = x^3 + ax^2 + b
\end{equation}
and its $j$-invariant is $j(E) = \frac{1}{b}$.

These isomorphism are easy to compute and we will always assume that
the elliptic curves given to our algorithms are in such simplified
forms.

\paragraph{Isogenies}
Elliptic curves are endowed with the classic group structure through
the chord-tangent law. A group morphism having finite kernel and
sending $\0$ over $\0$ is called an \emph{isogeny}. An isogeny whose
kernel has cardinality $\ell$ is called a degree $\ell$ isogeny or
simply an $\ell$-isogeny. Isogenies are algebraic maps, as such they
can be represented by rational functions. An isogeny is said to be
$\K$-rational if it is $\K$-rational as algebraic map.

One important property about isogenies is that they factor the
multiplication-by-$m$ map.

\begin{definition}[Dual isogeny]
  Let $\I : E \rightarrow E'$ be a degree $m$ isogeny. There exists an
  unique isogeny $\hat{\I} : E' \rightarrow E$, called the \emph{dual
    isogeny} such that
  \[\I\circ\hat{\I} = [m]_E \qquad\text{and}\qquad \hat{\I}\circ\I =
  [m]_{E'}\]
\end{definition}

As algebraic maps, isogenies can be separable, inseparable or purely
inseparable. In the case of finite fields, purely inseparable
isogenies are easily understood as powers of the frobenius map. Let
\[E^{(p)} : y^2 + a_1^pxy + a_3^py = x^3 + a_2^px^2 + a_4^px + a_6^p\]
then the map
\begin{align*}
  \frobisog : E &\rightarrow E^{(p)}\\
          (x,y) &\mapsto (x^p,y^p)
\end{align*}
is a degree $p$ purely inseparable isogeny. Any purely inseparable
isogeny is a composition of such frobenius isogenies.

Let $E$ and $E'$ be two elliptic curves defined over $\F_q$, by
finding an \emph{explicit isogeny} we mean to find an
($\F_q$-rational) rational function from $E(\clot{\F_q})$ to
$E'(\clot{\F_q})$ such that the map it defines is an isogeny.

\begin{figure}
  \centering
  \[\xymatrix{
    E \ar[r]^{[m]}\ar@/_1pc/[rrr]_{\I'} & E \ar[r]^\I & E' \ar[r]^{\frobisog^n} & E'^{(p^n)}\\
  }\]
  \caption{Factorization of an isogeny. $\I'$ has kernel $E[m]\oplus\ker\I$.}
\end{figure}

Since an isogeny can be uniquely factored in the product of a
separable and a purely inseparable isogeny, we focus ourselves on the
problem of computing explicit separable isogenies. Furthermore one can
factor out multiplication-by-$m$ maps, thus reducing the problem to
compute explicit separable isogenies with cyclic kernel.

In the rest of this paper, unless otherwise stated, by $\ell$-isogeny
we mean a separable isogeny with kernel isomorphic to $\Z/\ell\Z$.

\paragraph{Vélu formulae}
For any finite subgroup $G \subset E(\clot{\K})$, Vélu formulae
\cite{Vel71} give an elliptic curve $\tilde{E}$ and an explicit
isogeny in a canonical way. The isogeny is $\K$-rational if and only
if the polynomial vanishing on the abscissae of $G$ belongs to
$\K[X]$.

In practice, if $E$ is defined over $\F_q$ and if
\[h(X) = \prod_{\substack{P\in G\\P\ne\0}}(X - x(P)) \in \F_q[X]\]
is known, Vélu formulae compute a rational function
\begin{equation}
  \label{eq:isog}
  \I(x,y) = \left(\frac{g(x)}{h(x)}, \frac{k(x,y)}{l(x)}\right)  
\end{equation}
and a curve $\tilde{E}$ such that $\I : E\rightarrow\tilde{E}$ is an $\F_q$-rational
isogeny of kernel $G$. A consequence of Vélu formulae is
\begin{equation}
  \label{eq:velu-deg}
  \deg g = \deg h + 1 = \card{G}
  \text{.}
\end{equation}

Given two curves $E$ and $E'$, Vélu formulae reduce the problem of
finding an explicit isogeny between $E$ and $E'$ to that of finding
the kernel of an isogeny between them. Once the polynomial $h(X)$
vanishing on $\ker\I$ is found, the explicit isogeny is computed
composing Vélu formulae with the isomorphism between $\tilde{E}$ and
$E'$ as in figure \ref{fig:velu}.

\begin{figure}
  \centering
  \[\xymatrix{
    E \ar[r]^{\tilde{\I}} \ar[rd]^\I & \tilde{E} \ar[d]^{\simeq}\\
    & E'
  }\]
  \caption{Using Vélu formulae to compute an explicit isogeny.}
  \label{fig:velu}
\end{figure}




% Local Variables:
% mode:flyspell
% ispell-local-dictionary:"british"
% mode:TeX-PDF
% TeX-master: "ec-isogeny"
% End:
%
% LocalWords:  Schreier Artin pseudotrace frobenius bivariate Joux Sirvent FFT
% LocalWords:  Couveignes isogenies Schoof isogeny cryptosystems Lercier
% LocalWords:  precomputation arithmetics polylogarithmic Karatsuba
% LocalWords:  endomorphisms


\scriptsize

\begin{thebibliography}{10}

\bibitem{Atk91}
  A.O.L.~Atkin.
  \newblock \emph{The number of points on an elliptic curve modulo a prime}.
  \newblock Email on the Number Theory Mailing List, 1991.

\bibitem{BK78}R.P.~Brent and H.T.~Kung, Fast algorithms for
  manipulating formal power series.  \emph{J. ACM} 25(4):581--595, 1978.

\bibitem{BlSeSm99}
  I.~Blake, G.~Seroussi, and N.~Smart.
  \newblock {\em Elliptic curves in cryptography}.
  \newblock Cambridge University Press, 1999.

\bibitem{BoMoSaSc08}
  A.~Bostan, F.~Morain, B.~Salvy and É.~Schost
  \newblock Fast algorithms for computing isogenies between elliptic curves.
  \newblock \emph{Math. Comp.} 77, 263:1755-1778, 2008.

\bibitem{Magma}
  W.~Bosma, J.~Cannon, C.~Playoust.
  \newblock The Magma algebra system. I. The user language.
  \newblock \emph{J. Symb. Comp.}, 24(3-4):235-265, 1997.
  
\bibitem{gf2x}
  R.~Brent, P.~Gaudry, E.~Thom{\'e}, P.~Zimmermann.
  \newblock Faster multi\-plication in GF$(2)[x]$.
  \newblock In {\em ANTS'08}, 153-166. Springer, 2008.

\bibitem{Cou94}
  J.-M.~Couveignes.
  \newblock \emph{Quelques calculs en théorie des nombres}.
  \newblock PhD thesis. 1994.

\bibitem{Cou96}
  J.-M.~Couveignes. 
  \newblock Computing {$\ell$}-isogenies using the {$p$}-torsion.
  \newblock in {\em ANTS'II}, 59--65. Springer, 1996.

\bibitem{Cou00}
  J.-M.~Couveignes. 
  \newblock Isomorphisms between {A}rtin-{S}chreier towers. 
  \newblock \emph{Math. Comp.} 69(232): 1625--1631, 2000.
  
\bibitem{DFS09}
  L.~De~Feo and É.~Schost
  \newblock Fast Arithmetics in Artin Schreier Towers.
  \newblock \emph{ISSAC '09}, 2009.

\bibitem{Elk91}
  N.D.~Elikes
  \newblock \emph{Explicit Isogenies}.
  Draft, 1991.

\bibitem{EnMo03}
  A.~Enge and F.~Morain,
  \newblock Fast decomposition of polynomials with known Galois group.
  \newblock in {\em AAECC-15}, 254--264. Springer, 2003.

\bibitem{JL06}A.~Joux, R. Lercier.
  \newblock Counting points on elliptic curves in medium characteristic.
  \newblock \emph{Cryptology ePrint Archive} 2006/176, 2006.

\bibitem{vzGG} 
  J.~von~zur~Gathen and J.~Gerhard. 
  \newblock \emph{Modern Computer Algebra}. 
  \newblock Cambridge University Press, 1999.

\bibitem{Gun76}H.~Gunji,
  \newblock The Hasse Invariant and $p$-division Points of an Elliptic Curve.
  \newblock \emph{Archiv der Mathematik} 27(2), Springer, 1976.

\bibitem{Ler96}
  R.~Lercier.
  \newblock Computing isogenies in GF($2^n$).
  \newblock In {\em ANTS-II}, LNCS vol 1122, pages 197--212. Springer, 1996.

\bibitem{Ler97}
  R.~Lercier.
  \newblock Algorithmique des courbes elliptiques dans les corps finis. 
  \newblock Ph.D. Thesis, {\'E}cole polytechnique, 1997.

\bibitem{LeSi09}
  R.~Lercier, T.~Sirvent.
  \newblock On Elkies subgroups of $\ell$-torsion points in curves defined over
  a finite field.
  \newblock To appear in {\em J. Th\'eor. Nombres Bordeaux}.

\bibitem{RoSt06}
  A.~Rostovtsev and A.~Stolbunov.
  \newblock Public-key cryptosystem based on isogenies.
  \newblock Cryptology ePrint Archive, Report 2006/145.
  
\bibitem{Sat00}
  T.~Satoh.
  \newblock The canonical lift of an ordinary elliptic curve over a nite eld and its point counting.
  \newblock \emph{Journal of the Ramanujan Mathematical Society}, 2000.

\bibitem{Sch95}
  R.~Schoof.
  \newblock Counting points on elliptic curves over finite fields.
  \newblock \emph{J. Théorie des Nombres de Bordeaux} 7:219--254, 1995.

\bibitem{NTL}
  V.~Shoup.
  \newblock {NTL}: A library for doing number theory.
  \newblock \url{http://www.shoup.net/ntl/}.

\bibitem{Tes06}
  E.~Teske.
  \newblock An elliptic trapdoor system.
  \newblock {\em Journal of Cryptology}, 19(1):115--133, 2006.

\bibitem{Vel71}J.~Vélu,
  \newblock Isogénies entre courbes elliptiques.
  \newblock \emph{Comptes Rendus de l'Académie des Sciences de Paris} 273,
  Série A, 238--241, 1971.

  
  
\end{thebibliography}


\end{document}


% Local Variables:
% mode:flyspell
% ispell-local-dictionary:"british"
% End:
%
% LocalWords:  Schreier Artin pseudotrace frobenius bivariate memoization
% LocalWords:  Couveignes isogenies
