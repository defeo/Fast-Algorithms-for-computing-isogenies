\documentclass[preprint,1p]{elsarticle}

\usepackage[english]{babel}
\usepackage[utf8]{inputenc}
\usepackage{bbm}
\usepackage{amsmath}
\usepackage{amsthm}
\usepackage{amssymb}
\usepackage{mathrsfs}
\usepackage{mdwlist}
\usepackage{graphicx}
\usepackage{mathrsfs}
\usepackage{url}
\usepackage{color}
\usepackage[all]{xypic}

%%\usepackage[pdftex,plainpages=false]{hyperref}
%%\usepackage[pdftex]{color,graphicx}
%%\usepackage{ifthen}
%%\usepackage[all]{xy}

%% Stuff
\renewcommand{\le}{\leqslant}
\renewcommand{\ge}{\geqslant}  % comme François le demande...
\newcommand{\todo}{\dots}  % un marqueur pour un trou
%% Algèbre
\newcommand{\trans}[1]{#1^\top}  % transposé
\newcommand{\dual}[1]{#1^\ast}  % dual
\newcommand{\clot}[1]{\bar{#1}}  % clôture algèbrique
\newcommand{\card}[1]{\# #1}  % cardinalité d'un ensemble
\DeclareMathOperator{\car}{car}  % caractéristique d'un corps
\DeclareMathOperator{\Frac}{Frac}  % corps des fractions
\newcommand{\N}{\mathbb{N}}  % les naturels
\newcommand{\Z}{\mathbb{Z}}  % les entiers
\newcommand{\K}{\mathbb{K}}  % un corps
\newcommand{\LK}{\mathbb{L}}  % encore un corps
\newcommand{\U}{\mathbb{U}}  % encore un corps
\newcommand{\F}{\mathbb{F}}  % un corps fini
\newcommand{\Q}{\mathbb{Q}}  % les rationnels
\newcommand{\R}{\mathbb{R}}  % les réels
\newcommand{\C}{\mathbb{C}}  % les complexes
\newcommand{\isom}{\cong}  % isomorphisme de corps
\newcommand{\frob}{\varphi}  % l'isomorphisme de Frobenius
\DeclareMathOperator{\Gal}{Gal}  % groupe de Galois
\DeclareMathOperator{\Tr}{Tr}  % trace
\DeclareMathOperator{\PTr}{T}  % pseudotrace
\DeclareMathOperator{\Norm}{N} % norme
\DeclareMathOperator{\coeff}{coeff}  % coefficient
\newcommand{\res}{\rho}  % the residue form
\newcommand{\euler}{\phi}  % indicatrice d'Euler
\DeclareMathOperator{\ord}{ord}  % l'ordre d'un élément
\DeclareMathOperator{\rev}{rev}  % le reverse d'un polynôme
\newcommand{\Cyclo}{\Phi}  % le polynome cyclotomique
\DeclareMathOperator{\Fix}{Fix}  % les points fixes
%% Courbes
\newcommand{\Proj}{\mathbb{P}}  % espace projectif
\newcommand{\0}{\mathcal{O}}  % point de base d'une courbe
\newcommand{\ecpoint}[3]{[#1:#2:#3]}  % un point d'une courbe
\newcommand{\isog}[1]{\mathcal{#1}}  % la police des isogénies
\newcommand{\I}{\isog{I}}  % une isogénie I
\newcommand{\frobisog}{\phi}  % l'isogénie de Frobenius
\newcommand{\Hasse}{H}  % l'invariant de Hasse
\newcommand{\divpol}{f}  % polynôme de division
%% Autre
\newcommand{\tildO}{\tilde{O}}  % la notation O~ qui oublie les log
\newcommand{\Mult}{\mathrm{\sf M}}  % fonction de multiplication
\newcommand{\ModComp}{\mathrm{\sf C}}  % fonction de composition modulaire
\newcommand{\alg}[1]{{\sf #1}}  % la police des algorithmes
\newcommand{\algref}[1]{\alg{\ref{#1}}}  % la police des algorithmes
\newcommand{\wrt}{\dashv}  % appartenance forte, a\wrt A signifie que a est représenté comme un élément de A
\newcommand{\ndiv}{\nmid}  % ne divise pas

%%Théorèmes
\newtheorem{definition}{Definition}
\newtheorem{theorem}{Theorem}
\newtheorem{lemma}[definition]{Lemma}
\newtheorem{corollary}[definition]{Corollary}
\newtheorem{proposition}[definition]{Proposition}
\newtheorem{problem}[definition]{Problem}

\journal{Journal of Number Theory}

\begin{document}

\begin{frontmatter}

\title{Fast algorithms for computing isogenies between ordinary
  elliptic curves in small characteristic}
\author{Luca De Feo}
\ead{luca.defeo@polytechnique.edu}
\address{LIX, {\'E}cole Polytechnique, 91128 Palaiseau, France}


\begin{abstract}
  The problem of computing an explicit isogeny between two given
  elliptic curves over $\F_q$, originally motivated by point counting,
  has recently awaken new interest in the cryptology community thanks
  to the works of Teske and Rostovstev \& Stolbunov.

  While the large characteristic case is well understood, only
  suboptimal algorithms are known in small characteristic; they are
  due to Couveignes, Lercier, Lercier \& Joux and Lercier \& Sirvent.
  In this paper we discuss the differences between them and run some
  comparative experiments. We also present the first complete
  implementation of Couveignes' second algorithm and present
  improvements that make it the algorithm having the best asymptotic
  complexity in the degree of the isogeny.
\end{abstract}

\begin{keyword}
  Elliptic curves \sep Isogenies \sep Cryptography \sep Algorithms
\end{keyword}

\end{frontmatter}

\section{Introduction}

The problem of computing an explicit degree $\ell$ isogeny between two
given elliptic curves over $\F_q$ was originally motivated by point
counting methods based on Schoof's algorithm \cite{Atk91},
\cite{Elk91}, \cite{Sch95}. A review of the most efficient algorithms
to solve this problem is given in \cite{BoMoSaSc08} together with a
new quasi-optimal algorithm; however, all the algorithms presented in
\cite{BoMoSaSc08} compute the power series expansion of the
Weierstrass functions of the curves up to a certain precision, thus
they are limited to the case $\ell\ll p$ where $p$ is the
characteristic of $\F_q$. This is satisfactory for cryptographic
applications where one takes $p=q$ or $p=2$; indeed in the former case
Schoof's algorithm needs $\ell\in O(\log p)$, while in the latter case
there's no need to compute explicit isogenies since $p$-adic methods
based on \cite{Sat00} are preferred to Schoof's algorithm.

Nevertheless, the problem of computing explicit isogenies in the case
where $p$ is small compared to $\ell$ stays of theoretical interest
and can find practical applications in newer cryptosystems such as
\cite{Tes06}, \cite{RoSt06}. The first algorithm to solve this problem
was given by Couveignes and made use of formal groups \cite{Cou94}; it
takes $\tildO(\ell^3\log q)$ operations in $\F_p$ assuming $p$ is
constant, however it has an exponential complexity in $\log
p$. Another algorithm by Lercier specific to $p=2$ uses some linear
properties of the problem to build a linear system from whose solution
the isogeny can be deduced \cite{Ler96}; its complexity is conjectured
to be $\tildO(\ell^3\log q)$ operations in $\F_p$, but it has a much
better constant factor than \cite{Cou94}. At the moment we write the
latter algorithm is by many orders of magnitude the fastest algorithm
to solve practical instances of the problem when $p=2$, thus being the
\emph{de facto} standard for cryptographic use.

$p$-adic methods were used by Joux and Lercier \cite{JL06} and Lercier
and Sirvent \cite{LeSi09} to solve the isogeny problem. The former
method has complexity $\tildO(\ell^2(1 + \ell/p)\log q)$ operations in
$\F_p$, which makes it well adapted to the case where $p\sim\log
q$. The latter has complexity $\tildO(\ell^3 + \ell\log q^2)$
operations in $\F_p$, making it the best algorithm to our knowledge
for the case where $p$ is not constant.

\paragraph{The algorithm C2 and its variants}
Finally, the algorithm having the best asymptotic complexity in $\ell$
was proposed again by Couveignes in \cite{Cou96}, we will refer to
this original version as ``C2''. Its complexity --supposing $p$ is
fixed-- was estimated in \cite{Cou96} as being $\tildO(\ell^2\log q)$
operations in $F_p$, but with a precomputation step requiring
$\tildO(\ell^3\log q)$ operations and large memory
requirements. However this estimate was wrong as we will argue in
Section \ref{sec:} and C2 has an overall asymptotic complexity of
$\tildO(\ell^3\log q)$ operations.

Subsequent work by Couveignes used Artin-Schreier theory to avoid the
precomputation step of C2, drop the memory requirements to
$\tildO(\ell\log q + \log^2 q)$ elements of $\F_p$ and solve the issue
that made the estimate for the complexity of C2 false \cite{Cou00}. In
conclusion this variant has complexity $\tildO(\ell^2\log q)$, we
refer to it as ``C2-AS''.

However it has been recently shown in \cite{DFS09} that \cite{Cou00}
laid over some false assumptions on how fast arithmetics can be
performed in a tower of Artin-Schreier extensions. \cite{DFS09} fixes
the issue by giving asymptotically fast algorithms for Artin-Schreier
towers, but this is not enough to patch C2-AS as we will argue in
Section \ref{sec:}, thus bringing C2-AS back in the complexity class
$\tildO(\ell^3\log q)$.

The aim of this paper is to give a complete review of Couveignes'
algorithm and its variants, provide the missing piece to make it have
a quadratic dependency in $\ell$ and prove the accurate complexity
estimate of $\tildO(\incomplete)$ operations in $\F_p$. In Section
\ref{sec:} we also extend the algorithm beyond its original scope to
solve the more general problem of \emph{bounded degree isogeny
finding}; interestingly, this new algorithm turns out to have a better
asymptotic complexity than any previously known approach.

\paragraph{Notation and plan}
In the rest of the paper $p$ is a prime, $d$ a positive integer,
$q=p^d$ and $\F_q$ is the field with $q$ elements. For an elliptic
curve $E$ and a field $\K$ embedded in an algebraic closure
$\clot{\K}$, we note by $E(\K)$ the set of $\K$-rational points and by
$E[m]$ the $m$-torsion subgroup of $E(\clot{\K})$. The group law on
the elliptic curve is noted additively, its zero is the point at
infinity, noted $\0$. For an affine point $P$ we note by $x(P)$ its
abscissa and by $y(P)$ its ordinate. We will restrict ourselves to the
case of ordinary elliptic curves, thus $E[p^k]\isom\Z/p^k\Z$.

Unless otherwise stated, all time complexities will be measured in
number of operations in $\F_p$ and all space complexities in number of
elements of $\F_p$; we do not assume $p$ to be constant. We use the
notation $\tildO_x$ that forgets polylogarithmic factors in the
variable $x$, thus $O(xy\log x \log y)\subset\tildO_x(xy\log
y)\subset\tildO_{x,y}(xy)$. We simply note $\tildO$ when the variables
are clear from the context. We let $\Mult:\N\rightarrow\N$ be a
\emph{multiplication function}, such that polynomials of degree at
most $n$ with coefficients in $\F_p$ can be multiplied in $\Mult(n)$
operations, under the conditions of \cite[Ch. 8.3]{vzGG}. Typical
orders of magnitude are $O(n^{\log_23})$ for Karatsuba multiplication
or $O(n\log n\log\log n)$ for FFT multiplication.

The plan is the following.\\
Preliminaries on isogenies\\
C2\\
extension to $p=2$\\
complexity of C2\\
C2-AS\\
FAAST\\
interpolation\\
complexity of C2-AS-FI\\
modular composition\\
complexity of C2-AS-FI-MC\\
bounded degree problem\\
complexity of CBD-AS-FI-MC, comparison to other methods\\
implementation\\
benchmarks\\



% Local Variables:
% mode:flyspell
% ispell-local-dictionary:"british"
% mode:TeX-PDF
% TeX-master: "ec-isogeny"
% End:
%
% LocalWords:  Schreier Artin pseudotrace frobenius bivariate Joux Sirvent FFT
% LocalWords:  Couveignes isogenies Schoof isogeny cryptosystems Lercier
% LocalWords:  precomputation arithmetics polylogarithmic Karatsuba

\section{Preliminaries on Isogenies}

Let $E$ be an ordinary elliptic curve over the field $\F_q$, we
suppose it is given to us as the locus of zeroes of a Weierstrass
equation
\[y^2 + a_1xy + a_3y = x^3 + a_2x^2 + a_4x + a_6
\qquad a_1,\ldots,a_6\in\F_q\text{.}\]

\paragraph{Simplified forms} If $p>3$ it is well known that the curve
$E$ is isomorphic to a curve in the form
\begin{equation}
  \label{eq:weierstrass>3}
  y^2 = x^3 + ax + b
\end{equation}
and its $j$-invariant is $j(E) = \frac{1728(4a)^3}{16(4a^3 + 27b^2)}$.

When $p=3$, since $E$ is ordinary, it is isomorphic to a curve
\begin{equation}
  \label{eq:weierstrass=3}
  y^2 = x^3 + ax^2 + b
\end{equation}
and its $j$-invariant is $j(E) = -\frac{a^3}{b}$.

Finally, when $p=2$, since $E$ is ordinary, it is isomorphic to a curve
\begin{equation}
  \label{eq:weierstrass=2}
  y^2 + xy = x^3 + ax^2 + b
\end{equation}
and its $j$-invariant is $j(E) = \frac{1}{b}$.

These isomorphism are easy to compute and we will always assume that
the elliptic curves given to our algorithms are in such simplified
forms.

\paragraph{Quadratic twists}

\paragraph{Isogenies}

\paragraph{Vélu formulae}




% Local Variables:
% mode:flyspell
% ispell-local-dictionary:"british"
% End:
%
% LocalWords:  Schreier Artin pseudotrace frobenius bivariate Joux Sirvent FFT
% LocalWords:  Couveignes isogenies Schoof isogeny cryptosystems Lercier
% LocalWords:  precomputation arithmetics polylogarithmic Karatsuba

\section{The algorithm C2}

The algorithm we refer to as C2 was originally proposed in
\cite{Cou96}. It takes as input two elliptic curves $E, E'$ and an odd
prime $\ell$ different from $p$ and it returns, if it exists, an
$\F_q$-rational isogeny of degree $\ell$ between $E$ and $E'$. It only
works in odd characteristic.

\paragraph{The idea} Suppose it exists an $F_q$-rational isogeny
$\I:E\rightarrow E'$ of degree $\ell$. Since $\ell\ne p$ one has
$\I(E[p^k]) = E'[p^k]$.

Recall that $E[p^k]$ and $E'[p^k]$ are cyclic groups, C2 iteratively
computes generators $P_k,P_k'$ of $E[p^k]$ and $E'[p^k]$ respectively
for $k$ large enough. Now C2 makes the guess $\I(P_k) = P_k'$, then if
$\I$ is given by rational fractions as in \eqref{eq:isog},
\begin{equation}
  \label{eq:C2:I}
  \frac{g\bigl(x([i]P_k)\bigr)}{h\bigl(x([i]P_k)\bigr)} = x([i]P_k')
  \quad\text{for $i\in(\Z/p^k\Z)^\ast$.} 
\end{equation}

C2 computes $T_k(X) = \prod_{i\in(\Z/p^k\Z)^\ast}\bigl(X-x([i]P_k)\bigr)$ and
interpolates the polynomial $A(X)$ such that
\begin{equation}
  \label{eq:C2:A}
  A\bigl(x([i]P_k)\bigr) = x([i]P_k') \quad\text{for $i\in(\Z/p^k\Z)^\ast$,} 
\end{equation}
it is then clear that
\begin{equation}
  \label{eq:C2:RR}
  A(X) \equiv \frac{g(X)}{h(X)} \bmod T(X)
  \text{.}
\end{equation}

C2 computes the rational fraction $\frac{g(X)}{h(X)}$ through a
rational function reconstruction algorithm \cite[\incomplete]{vzGG}
and checks that it corresponds to the restriction of an isogeny to the
$x$-axis. If it does, the whole isogeny is computed through Vélu
formulae and the algorithm terminates. If it does not, the guess
$\I(P_k) = P_k'$ was false; then it computes a new generator for
$E'[p^k]$ and starts over again.

We now go through the details of the algorithm.

\paragraph{The $p$-torsion}
The computation of the $p$-torsion points follows from the work of
Gunji \cite{Gun76}. 

\begin{definition}
  \label{def:hasse}
  Let $E$ have equation $y^2 = f(x)$, one defines the \emph{Hasse
    invariant} of $E$, noted $H_E$ as the coefficient of $X^{p-1}$ in
  $f(X)^{\frac{p-1}{2}}$.
\end{definition}

\begin{proposition}
  Let $c=\sqrt[p-1]{H_E}$, then the $p$-torsion points of $E$ are
  defined in $\F_q[c]$ and their abscissae are defined in $\F_q[c^2]$.
\end{proposition}


\paragraph{The $p^k$-torsion}


% Local Variables:
% mode:flyspell
% ispell-local-dictionary:"british"
% mode:TeX-PDF
% TeX-master: "ec-isogeny"
% End:
%
% LocalWords:  Schreier Artin pseudotrace frobenius bivariate Joux Sirvent FFT
% LocalWords:  Couveignes isogenies Schoof isogeny cryptosystems Lercier
% LocalWords:  precomputation arithmetics polylogarithmic Karatsuba
% LocalWords:  endomorphisms

\input{C2-AS}
\input{C2-AS-FI}
\input{C2-AS-FI-MC}
%\documentclass{article}

\usepackage[english]{babel}
\usepackage[utf8]{inputenc}
\usepackage{bbm}
\usepackage{amsmath}
\usepackage{amsthm}
\usepackage{amssymb}
\usepackage{mathrsfs}
\usepackage{mdwlist}
\usepackage{graphicx}
\usepackage{mathrsfs}


%% Stuff
\renewcommand{\le}{\leqslant}
\renewcommand{\ge}{\geqslant}  % comme François le demande...
\newcommand{\todo}{\dots}  % un marqueur pour un trou
%% Algèbre
\newcommand{\trans}[1]{#1^\top}  % transposé
\newcommand{\dual}[1]{#1^\ast}  % dual
\newcommand{\clot}[1]{\bar{#1}}  % clôture algèbrique
\newcommand{\card}[1]{\# #1}  % cardinalité d'un ensemble
\DeclareMathOperator{\car}{car}  % caractéristique d'un corps
\DeclareMathOperator{\Frac}{Frac}  % corps des fractions
\newcommand{\N}{\mathbb{N}}  % les naturels
\newcommand{\Z}{\mathbb{Z}}  % les entiers
\newcommand{\K}{\mathbb{K}}  % un corps
\newcommand{\LK}{\mathbb{L}}  % encore un corps
\newcommand{\U}{\mathbb{U}}  % encore un corps
\newcommand{\F}{\mathbb{F}}  % un corps fini
\newcommand{\Q}{\mathbb{Q}}  % les rationnels
\newcommand{\R}{\mathbb{R}}  % les réels
\newcommand{\C}{\mathbb{C}}  % les complexes
\newcommand{\isom}{\cong}  % isomorphisme de corps
\newcommand{\frob}{\varphi}  % l'isomorphisme de Frobenius
\DeclareMathOperator{\Gal}{Gal}  % groupe de Galois
\DeclareMathOperator{\Tr}{Tr}  % trace
\DeclareMathOperator{\PTr}{T}  % pseudotrace
\DeclareMathOperator{\Norm}{N} % norme
\DeclareMathOperator{\coeff}{coeff}  % coefficient
\newcommand{\res}{\rho}  % the residue form
\newcommand{\euler}{\phi}  % indicatrice d'Euler
\DeclareMathOperator{\ord}{ord}  % l'ordre d'un élément
\DeclareMathOperator{\rev}{rev}  % le reverse d'un polynôme
\newcommand{\Cyclo}{\Phi}  % le polynome cyclotomique
\DeclareMathOperator{\Fix}{Fix}  % les points fixes
%% Courbes
\newcommand{\Proj}{\mathbb{P}}  % espace projectif
\newcommand{\0}{\mathcal{O}}  % point de base d'une courbe
\newcommand{\ecpoint}[3]{[#1:#2:#3]}  % un point d'une courbe
\newcommand{\isog}[1]{\mathcal{#1}}  % la police des isogénies
\newcommand{\I}{\isog{I}}  % une isogénie I
\newcommand{\frobisog}{\phi}  % l'isogénie de Frobenius
\newcommand{\Hasse}{H}  % l'invariant de Hasse
\newcommand{\divpol}{f}  % polynôme de division
%% Autre
\newcommand{\tildO}{\tilde{O}}  % la notation O~ qui oublie les log
\newcommand{\Mult}{\mathrm{\sf M}}  % fonction de multiplication
\newcommand{\ModComp}{\mathrm{\sf C}}  % fonction de composition modulaire
\newcommand{\alg}[1]{{\sf #1}}  % la police des algorithmes
\newcommand{\algref}[1]{\alg{\ref{#1}}}  % la police des algorithmes
\newcommand{\wrt}{\dashv}  % appartenance forte, a\wrt A signifie que a est représenté comme un élément de A
\newcommand{\ndiv}{\nmid}  % ne divise pas

%%Théorèmes
\newtheorem{definition}{Definition}
\newtheorem{theorem}{Theorem}
\newtheorem{lemma}[definition]{Lemma}
\newtheorem{corollary}[definition]{Corollary}
\newtheorem{proposition}[definition]{Proposition}
\newtheorem{problem}[definition]{Problem}


\title{On computing isogenies of unknown degree}
\author{Luca De Feo}

\begin{document}

\maketitle

\begin{abstract}
  We present an extension to Couveignes algorithm that, given two
  elliptic curves $E$ and $E'$, permits to compute all isogenies of
  degrees up to a certain bound $N$ in time $\tildO(N^2)$.
\end{abstract}

It is well known that two curves having the same number of points over
a finite field are isogenous, however this doesn't say anything on the
degree of the isogeny connecting them. Given two elliptic curves $E$
and $E'$ defined over $\F_q$ and having the same number of points, we
want to find the smallest degree isogeny between them.

The simplest solution is to take any algorithm computing a fixed
degree isogeny and try all the degrees until an isogeny is found. If
$\ell$ is the degree of the smallest isogeny, this of course adds a
factor $\ell$ to the complexity of any polynomial time algorithm.

Couveignes' algorithm \cite{Cou96}, however, can be easily adapted to
solve this problem at no additional cost. This short paper is not
self-contained: we make references to the variants and improvements to
Couveignes' algorithm given in \cite{D}.

Observe that in Couveignes' algorithm, apart for the choice of $k$,
the computation of $E[p^k]$ and the polynomial interpolation step do
not depend at all on $\ell$. The degree of the isogeny only comes into
play in the last part of the Cauchy interpolation, that is the
rational function reconstruction. We study more in detail this last
step.

\paragraph{Rational Function Reconstruction}
Rational function reconstruction takes as input a degree $n$
polynomial $T$, a polynomial $A$ of degree less than $n$ and a target
degree $m\le n$ and outputs the unique rational function such that
\begin{equation*}
  A \equiv \frac{R}{V} \bmod T
\end{equation*}
and $\deg R < m$, $\deg V \le n-m$. This is done by computing a Bezout
relation $AV + TU = R$ with the expected degrees via an XGCD
algorithm. If a classical XGCD algorithm is used, one simply computes
all the lines
\begin{equation}
  \label{eq:XGCD}
  \begin{aligned}
    R_0 &= T, & U_0 &= 1, & V_0 &= 0,\\
    R_1 &= A, & U_1 &= 0, & V_1 &= 1,\\
    R_{i-1} &= Q_iR_i + R_{i+1}, & U_{i+1} &= U_{i-1}-Q_iU_i, & V_{i+1} &= V_{i-1}-Q_iV_i
  \end{aligned}
\end{equation}
and stops as soon as a remainder $R_{i+1}$ with $\deg R_{i+1}<m$ is
found. If a fast XGCD algorithm as \cite[Algo. 11.4]{vzGG} is used,
one directly aims at the two lines
\begin{equation}
  \label{eq:FastGCD}
  \begin{aligned}
    R_{h-2} &= Q_{h-1}R_{j-1} + R_h\\
    R_{h-1} &= Q_hR_h + R_{h+1}
  \end{aligned}
\end{equation}
such that $\deg R_{h+1} < m \le \deg R_h$ without computing the
intermediate lines.

When looking for an $\ell$-isogeny, one simply sets
$m=\ell+1$. Observe that if the algorithm doesn't return a rational
fraction $\frac{R}{V}$ with $\deg R = \ell$ and $\deg V = \ell -1 $,
then no such fraction congruent to $A$ modulo $T$ exists.

If $\ell$ is not \emph{a priori} known, we can still use the fact that
a separable isogeny with cyclic kernel must have $\deg R = \deg V +
1$. In fact if we suppose $R = R_i$ and $V = V_i$, then
\begin{align*}
  \deg T &= \deg V_{i+1} + \deg R_i,\\
  \deg R_i - \deg V_i &= \deg R_{i-1} - \deg V_{i+1}
\end{align*}
implies
\begin{equation*}
  \deg T + 1 = \deg R_{i-1} + \deg R_i  
  \;\text{.}
\end{equation*}
Hence, if $A$ is congruent to an $\ell$-isogeny with $\ell =
\left\lfloor\frac{\deg T}{2}\right\rfloor - t$ for some $t\ge0$, then
\begin{equation}
  \label{eq:degseq}
  \deg R_{i-1} =
  \left\lceil\frac{\deg T}{2}\right\rceil + t + 1 >
  \left\lfloor\frac{\deg T}{2}\right\rfloor - t = \deg R_i
  \;\text{.}
\end{equation}
Thus we can recover any isogeny having degree less than
$\left\lfloor\frac{\deg T}{2}\right\rfloor$ using either a classical
or a fast XGCD algorithm setting $m = \left\lceil\frac{\deg
    T}{2}\right\rceil + 1$.


\paragraph{Recognising an isogeny}
Once we have a rational fraction with the required degree, we have to
test if it really is an isogeny. In order to understand how often we
have to make this test, we introduce some more terminology. Let $n_i =
\deg R_i$, we call $(n_0,\ldots,n_r)$ the \emph{degree sequence} of
$A$ and $T$; a degree sequence is said \emph{normal} if $n_i = n_{i+1}
+ 1$ for any $i$.

\begin{proposition}
  \label{th:normseq}
  Let $f,g\in\F_q[X]$ be uniformly chosen random polynomials of
  respective degrees $n_0>n_1>0$ and let $(n_0, n_1, \ldots, n_r)$ be
  their degree sequence. For $0\le i < n_1$ define the binary random
  variables $X_i = 1 \Leftrightarrow i\in(n_0,n_1,\ldots,n_r)$, then
  the $X_i$ are independent random variables and $\mathrm{Prob}(X_i=0) =
  \frac{1}{q}$.
\end{proposition}
\begin{proof}
  Pairs of polynomials $f,g$ are in bijection with the GCD-sequence
  $(R_r, Q_r, \ldots, Q_1)$ constituted by their GCD and the quotients
  of the GCD algorithm. To each such sequence is associated a degree
  sequence
  \begin{equation*}
    (n_0,n_1,\ldots,n_r) =
    \left(\deg R_r + \sum_{i=1}^r\deg Q_i, \ldots, \deg R_r + \sum_{i=1}^1\deg Q_i, \deg R_r\right)
    \;\text{,} 
  \end{equation*}
  thus for any given degree sequence there are
  \begin{equation*}
    (q-1)q^{n_0-n_1}\cdot(q-1)q^{n_1-n_2}\cdot\cdots\cdot(q-1)q^{n_r} =
    (q-1)^{r+1}q^{n_0}
  \end{equation*}
  GCD-sequences.
 
  Let $I$ and $O$ be two disjoints subsets of $\{X_i\}$, the number of
  GCD-sequences such that $X\in I \Rightarrow X=1$ and $X\in O
  \Rightarrow X=0$,
  \begin{equation*}
     \sum_{s=0}^{n_1-\card{I}-\card{O}}\binom{n_1-\card{I}-\card{O}}{s}(q-1)^{s+2+\card{I}}q^{n_0} =
    (q-1)^{2+\card{I}}q^{n_0}q^{n_1-\card{I}-\card{O}}
    \;\text{.}
  \end{equation*}
  There are $(q-1)^2q^{n_0}q^{n_1}$ pairs of polynomials of degrees
  $n_0,n_1$, thus
  \begin{equation}
   \label{th:normseq:prob}
    \mathrm{Prob}\bigl(\{X = 1 \mid X\in I\},
    \{X=0\mid X\in O\}\bigr) = \left(\frac{q-1}{q}\right)^{\card{I}}\left(\frac{1}{q}\right)^{\card{O}}
    \;\text{.}
  \end{equation}
  The claim follows.
\end{proof}

Degree sequences associated to isogenies are in general not normal, in
fact if $\ell\le\left\lfloor\frac{\deg T}{2}\right\rfloor-t$, equation
\eqref{eq:degseq} shows that there must be at least a gap of degree
$2c$ in the degree sequence. Heuristically, we can expect that if the
polynomial $A$ doesn't correspond to an isogeny, then $A$ and $T$ act
like random polynomials, thus, by the proposition above, the
probability that $A$ looks like an isogeny of degree
$\ell\le\left\lfloor\frac{\deg T}{2}\right\rfloor-t$ is less than $\frac{1}{q^{2t}}$.

Therefore, by choosing an appropriate $t\in O(\log_q p^k)$, our
variant can find any isogeny of degree less than $\frac{p^k-1}{4}-t$
at the same cost of one run of Couveignes' algorithm.

No other method for computing isogenies is known to have a similar
generalisation. This makes Couveignes' algorithm a rather surprising
exception and we wonder whether this simple idea can find interesting
applications.

\begin{thebibliography}{9}
\bibitem{Cou96}
  J.-M.~Couveignes. 
  \newblock Computing {$\ell$}-isogenies using the {$p$}-torsion.
  \newblock in {\em ANTS'II}, 59--65. Springer, 1996.

\bibitem{D}
  L.~De~Feo.
  \newblock Fast algorithms for computing isogenies between ordinary
  elliptic curves in small characteristic,
  \newblock To appear in {\em Journal of Number Theory}, 2010.

\bibitem{vzGG} 
  J.~von~zur~Gathen and J.~Gerhard. 
  \newblock \emph{Modern Computer Algebra}. 
  \newblock Cambridge University Press, 1999.

\end{thebibliography}


\end{document}

% Local Variables:
% mode:flyspell
% ispell-local-dictionary:"british"
% mode:TeX-PDF
% End:
%
% LocalWords:  Schreier Artin pseudotrace Frobenius bivariate Joux Sirvent FFT
% LocalWords:  Couveignes isogenies Schoof isogeny cryptosystems Lercier
% LocalWords:  precomputation arithmetics polylogarithmic Karatsuba precomputes
% LocalWords:  endomorphisms  isogenous

\section{Implementation}
\label{sec:implementation}

We implemented C2-AS-FI and C2SD-AS-FI as \texttt{C++} programs using
the libraries \texttt{NTL} for finite field arithmetics, \texttt{gf2x}
for fast arithmetics in characteristic $2$ and \texttt{FAAST} for fast
arithmetics in Artin-Schreier towers.

This section mainly deals with some tricks we implemented in order to
speed up the computation.

\subsection{Building $E[p^k]$ and $E'[p^k]$}
\label{sec:impl:torsion}

\paragraph{$p$-torsion}
For $p\ne2$, C2 and its variants require to build the extension
$\F_q[c]$ where $c$ is a $p-1$-th root of $H_E$. In order to deal with
the lowest possible extension degree, it is a good idea to modify the
curve so that $[\F_q[c]:\F_q]$ is the smallest possible.

$[\F_q[c]:\F_q]$ is invariant under isomorphism, but taking a twist
can save us a quadratic extension. Let $u=c^{-2}$, the curve
\begin{equation*}
  \bar{E} : y^2 = x^3 + a_2ux^2 + a_4u^2x + a_6u^3
\end{equation*}
is defined over $\F_q[c^2]$ and is isomorphic to $E$ over $\F_q[c]$
via $(x,y)\mapsto(\sqrt{u}^2x,\sqrt{u}^3y)$. Its Hasse invariant is
$H_{\bar{E}} = (u)^{\frac{p-1}{2}}H_E = 1$, thus its $p$-torsion
points are defined over $\F_q[c^2]$.

In order to compute the $p^k$-torsion points of $E$ we build
$\F_q[c^2]$, we compute $\bar{P}$ a $p^k$-torsion points of $\bar{E}$
using $p$-descent, then we invert the isomorphism to compute the
abscissa of $P\in E[p^k]$. Since the Cauchy interpolation only needs
the abscissae of $E[p^k]$, this is enough to complete the
algorithm. Scalar multiples of $P$ can be computed without knowledge
of $y(P)$ using Montgomery formulae \cite{Mon87}.

Remark that for $p=2$ we use the same construction in an implicit way
since we do a $p$-descent on the Kummer surface.


\paragraph{$p^k$-torsion points}
For $p\ne2$ we use Voloch's $p$-descent to compute the $p^k$-torsion
points iteratively as described in Section \ref{sec:C2}. To factor the
Artin-Schreier polynomial \eqref{th:voloch:cover}, we use the
algorithms from \cite{Cou00} and \cite{DFS09} that were analysed in
Section \ref{sec:C2-AS}. All these algorithms were provided by the
library \texttt{FAAST}.

To solve system \eqref{th:voloch:isom} we first compute
\begin{equation*}
  V(x,y) = \left(\frac{g(x)}{h^2(x)}, 
    sy\left(\frac{g(x)}{h^2(x)}\right)'\right)
\end{equation*}
through Vélu formulae.\footnote{Vélu formulae compute this isogeny up
  to an indeterminacy on the sign of the ordinate, the actual value of
  $s$ must be determined by composing $V$ with $\frobisog$ and
  verifying that it corresponds to $[p]$ by trying some random
  points.} Recall that we work on a curve having Hasse invariant $1$,
system \eqref{th:voloch:isom} can then be rewritten
\begin{equation*}
  \left\{
    \begin{aligned}
      X &= \frac{g(x)}{h^2(x)}\\
      Y &= sy\left(\frac{g(x)}{h^2(x)}\right)'\\
      Z &= -2y\frac{h'(x)}{h(x)}
    \end{aligned}
  \right.
\end{equation*}
where $(X,Y,Z)$ is the point on the cover $C$ that we want to pull
back. After some substitutions this is equivalent to
\begin{equation*}
  \left\{
    \begin{aligned}
      Xh^2(x) - g(x) &= 0\\
      \left(Xh^2(x) - g(x) - \frac{Y}{sZ}h^2(x)\right)' &= 0
    \end{aligned}
  \right.
\end{equation*}
Then a solution to this system given by the GCD of the two
equations. Remark that proposition \ref{th:voloch} ensures there is
one unique solution. This formulae are slightly more efficient than
the ones in \cite[$\S$6.2]{Ler97}.

For $p=2$ we use the library \texttt{FAAST} (for solving
Artin-Schreier equations) on top of \texttt{gf2x} (for better
performance). There is nothing special to remark about the
$2$-descent.


\subsection{Cauchy interpolation and loop}
\label{sec:impl:cauchy}
The polynomial interpolation step is done as described in Section
\ref{sec:C2-AS-FI}. As a result of this implementation, the polynomial
interpolation algorithm was added to the library \texttt{FAAST}.

The rational fraction reconstruction is implemented using a fast XGCD
algorithm on top of \texttt{NTL} and \texttt{gf2x}. This algorithm was
added to \texttt{FAAST} too.

To check that the rational fractions are isogenies we test their
degrees, that their denominator is a square and that they act as group
morphisms on a fixed number of random points. All these checks take a
negligible amount of time compared to the rest of the algorithm.

However asymptotically fast, the polynomial interpolation step is
quite expensive for reasonably sized data. Instead of repeating it
$\frac{\euler(p^k)}{2}$ times, we used composition with the frobenius
endomorphism $\frobisog_E$ in order to reduce the number of
interpolations in the final loop. This works as follows. Suppose we
have computed the polynomial $A_0\in\F_q[X]$ such that
\begin{equation*}
  A_0\bigl(x\bigl([n]P\bigr)\bigr) = x\bigl([n]P'\bigr)
  \quad\text{for any $n$,}
\end{equation*}
we view $A_0$ as the restriction to $K_E[p^k]$ of an $\F_q$-rational
algebraic map\footnote{Notice this is not necessarily a group
  morphism.}  $\mathcal{A}_0:E\rightarrow E'$.

Let $T$ be the polynomial vanishing on the abscissae of $E[p^k]$
computed by the algorithm in Section \ref{sec:C2-AS-FI}, compute
$F\in\F_q[X]$, the restriction of $\frobisog_E$ to $K_E[p^k]$, this is
\begin{equation}
  \label{eq:frob}
  F(X) = X^q \bmod T(X)
  \text{.}
\end{equation}

Define the algebraic maps
\begin{equation*}
  \mathcal{A}_i := \mathcal{A}_0\circ\frobisog_E^i = \frobisog_{E'}^i\circ\mathcal{A}_0
\end{equation*}
where the equality follows from the fact that $\mathcal{A}_0$ is
$\F_q$-rational. Then their restrictions to $K_E[p^k]$ are given by
\begin{equation}
  \label{eq:modcomp}
  A_i(X) = A_{i-1}(X)\circ F(X) \bmod T(X)
\end{equation}
and 
\begin{equation*}
  A_i\bigl(x\bigl([n]P\bigr)\bigr) = x\bigl([n]\frobisog_{E'}^i(P')\bigr)
  \quad\text{for any $n$.}
\end{equation*}
Since $\frob_{E'}^i(P')$ is a primitive element of $E'[p^k]$, for any
$0 < i < [\U_k:\F_q]$ we get a new interpolating polynomial as
required by C2 and we only have to do the rational fraction
reconstruction step of the Cauchy interpolation. If
$\frac{2[\U_k:\F_q]}{\euler(p^k)} = p^{i_0-1}r > 1$, we must compute $p^{i_0-1}r$
polynomial interpolations and apply this algorithm to each of them in
order to deduce all the polynomials needed by C2.

We compute \eqref{eq:frob} via square-and-multiply, this costs
$\Theta(d\Mult(p^kd)\log p)$ operations. Each application of
\eqref{eq:modcomp} is done via a \emph{modular composition}. The best
algorithm known for this problem is \cite{Uma09} and using it would
yield an asymptotic complexity similar\footnote{\cite{Uma09} has a
  slightly better dependency in $p$ and $d$ and a slightly worse
  dependency in $\ell$.} to the one of C2-AS-FI; however, \cite{Uma09}
is not useful on practical instances of the problem, thus we have to
resort to \cite{BrKu78} which has a much worse asymptotic complexity,
but performs much faster then our interpolation algorithm in practice.

A similar approach could be used inside the polynomial interpolation
step\footnote{See Section \ref{sec:C2-AS-FI}.} to deduce $A_k^{(0)}$
from $A_0^{(0)}$ using modular composition with the multiplication
maps of $E$ and $E'$ as described in \cite[$\S$2.3]{Cou96}. We didn't
implement this variant, though.


\subsection{Parallelisation of the loop}
\label{parallel}

The most expensive step of C2-AS-FI and C2SD-AS-FI, in theory as well
as in practice, is the final loop over the points of
$E'[p^k]$. Fortunately, this phase is very easy to parallelise with
very few overhead.

Let $n$ be the number of processors we wish to parallelise on,
suppose\footnote{Otherwise the parallelisation is straightforward.}
that $[\U_k:\F_q] = \frac{\euler(p^k)}{2}$ and set
$m=\left\lfloor\frac{\euler(p^k)}{2n}\right\rfloor$. We compute the
restriction of $\frobisog_E^{m}$ to $K_E[p^k]$ as
\begin{equation*}
  F^{(m)}(X) = F(X) \circ \cdots \circ F(X) \bmod T(X)
  \;\text{,}
\end{equation*}
this can be done with $\Theta(\log m)$ modular compositions via a
binary square-and-multiply-like approach.

Then compute the $n$ polynomials
\begin{equation*}
  A_{mi}(X) = A_{m(i-1)}(X) \circ F^{(m)}(X) \bmod T(X)
\end{equation*}
and distribute them to the $n$ processors so that they each work on a
separate slice of the $A_i$'s.



% Local Variables:
% mode:flyspell
% ispell-local-dictionary:"british"
% mode:TeX-PDF
% TeX-master: "ec-isogeny"
% End:
%
% LocalWords:  Schreier Artin pseudotrace frobenius bivariate Joux Sirvent FFT
% LocalWords:  Couveignes isogenies Schoof isogeny cryptosystems Lercier Hasse
% LocalWords:  precomputation arithmetics polylogarithmic Karatsuba precomputes
% LocalWords:  endomorphisms 

\section{Benchmarks}
\label{sec:benchmarks}
We ran various experiments to compare the different variants of the
algorithm C2 between themselves and to the other algorithms. All the
experiments were run on four dual-core Intel Xeon E5430 (2.6GHz),
eventually using the parallelised version of the algorithm.

\begin{figure}
  \centering
  \includegraphics[width=0.9\textwidth]{p2}
  \caption{Comparative timings for different implementations of C2-AS-FI-MC with curves defined over $\F_{2^{101}}$. Plot in logarithmic scale.}
  \label{fig:2-101}
\end{figure}

The first set of experiments was run to evaluate the benefits of using
the fast algorithms in~\cite{DFS09}. We selected pairs of isogenous
curves over $\F_{2^{101}}$ such that the height of the tower is
maximal (observe that this is always the case for cryptographic
curves). The library \texttt{FAAST} offers two types for finite field
arithmetics in characteristic $2$: \texttt{zz\_p} which is a generic
type for word-precision $p$ and \texttt{GF2} which uses the optimised
algorithms of the library \texttt{gf2x}. We compared implementations
of C2-AS-FI-MC using these two types with an implementation written in
Magma. The results are in figure~\ref{fig:2-101}: we plot a line for
the average running time of the algorithm and bars around it for
minimum and maximum execution times of the final loop. Besides the
dramatic speedup obtained by using the ad-hoc type \texttt{GF2}, the
algorithmic improvements of \texttt{FAAST} over Magma are evident as
even \texttt{zz\_p} is one order of magnitude faster.

\begin{table}
  \centering
  \begin{tabular}{r r r r r r r r}
    \hline
    $\ell$ & $E[p^k]$ & $E'[p^k]$ & FI & RFR & MC & Avg tries & Avg loop time\\
    \hline
    31  &  0.3529 &  0.3529 & 0.3569 & 0.00125 & 0.00055 &  32 &   0.058\\
    61  &  0.9848 &  0.9848 & 0.8268 & 0.00343 & 0.00228 &  64 &   0.365\\
    127 &  2.6636 &  2.6626 & 1.8927 & 0.01090 & 0.00872 & 128 &   2.511\\
    251 &  6.9809 &  6.9779 & 4.2833 & 0.03092 & 0.03494 & 256 &  16.860\\
    397 & 18.1052 & 18.0952 & 9.7385 & 0.07325 & 0.14117 & 512 & 109.783\\
    \hline
  \end{tabular}
  \caption{Comparative timings for the phases of C2-AS-FI-MC for curves over $\F_{2^{101}}$.}
  \label{tab:C2}
\end{table}

Table~\ref{tab:C2} shows detailed timings for each phase of
C2-AS-FI-MC. The column FI reports the time for one interpolation, the
column MC the time for one modular composition; comparing these two
columns the gain from passing from C2-AS-FI to C2-AS-FI-MC is
evident. Columns RFR (rational fraction reconstruction) and MC
constitute the Cauchy interpolation step that is repeated in the final
loop. The last column reports the average time spent in the loop: it
is by far the most expensive phase and this justifies the attention we
paid to FI and MC; only on some huge examples we approached the
crosspoint between these two algorithms.

\begin{figure}
  \centering
  \includegraphics[height=0.45\textwidth]{C2-LS}
   \includegraphics[height=0.45\textwidth]{C2-LS2}
   \caption{Comparative timings for C2-AS-FI-MC (C2) and LS over
     different curves. Plot in logarithmic scale.}
  \label{fig:comp}
\end{figure}

Next, we compare the running times of C2-AS-FI-MC and LS over curves
of half the cryptographic size in figure~\ref{fig:comp} (left). We
only plot average times for C2, in characteristic $2$ we only plot the
timings for \texttt{GF2}. From the plot it is clear that C2-AS-FI-MC
only performs better than LS for $p=2$, but in this case the algorithm
of~\cite{Ler96} is by far better.  Figure~\ref{fig:comp} (right) shows
that LS slowly gets worse than C2, however comparing a Magma prototype
to our highly optimised implementation of C2-AS-FI-MC is somewhat
unfair and probably the crosspoint between the two algorithms lies
much further. Furthermore, it is unlikely that C2-AS-FI-MC could be
practical for any $p>3$ because of its high dependence on $p$, while
LS scales pretty well with the characteristic as shown in
figure~\ref{fig:LSp}.

\begin{figure}
  \centering
  \includegraphics[width=0.9\textwidth]{LSp}
  \caption{Timings for LS for different fields. We increase $p$ while
    keeping constant $d$ and the isogeny degree.}
  \label{fig:LSp}
\end{figure}

We can hardly hide our disappointment concluding that, despite their
good asymptotic behaviour and our hard work implementing them, the
variants derived from C2 don't seem to be practical, at least for
present day data sizes. We hope that in the future the algorithms
presented here may turn useful to compute very large data that are
currently out of reach.




% Local Variables:
% mode:flyspell
% ispell-local-dictionary:"british"
% TeX-master: "ec-isogeny"
% End:
%
% LocalWords:  Schreier Artin pseudotrace Frobenius bivariate Joux Sirvent FFT
% LocalWords:  Couveignes isogenies Schoof isogeny cryptosystems Lercier
% LocalWords:  precomputation arithmetics polylogarithmic Karatsuba precomputes
% LocalWords:  endomorphisms  isogenous


\section*{Acknowledgements}
We would like to thank J.-M.~Couveignes, F.~Morain, E.~Schost and
B.~Smith for useful discussions and precious proof-reading.


\scriptsize

\begin{thebibliography}{10}

\bibitem{Atk91}
  A.O.L.~Atkin.
  \newblock \emph{The number of points on an elliptic curve modulo a prime}.
  \newblock Email on the Number Theory Mailing List, 1991.

\bibitem{BK78}R.P.~Brent and H.T.~Kung, Fast algorithms for
  manipulating formal power series.  \emph{J. ACM} 25(4):581--595, 1978.

\bibitem{BlSeSm99}
  I.~Blake, G.~Seroussi, and N.~Smart.
  \newblock {\em Elliptic curves in cryptography}.
  \newblock Cambridge University Press, 1999.

\bibitem{BoMoSaSc08}
  A.~Bostan, F.~Morain, B.~Salvy and É.~Schost
  \newblock Fast algorithms for computing isogenies between elliptic curves.
  \newblock \emph{Math. Comp.} 77, 263:1755-1778, 2008.

\bibitem{Magma}
  W.~Bosma, J.~Cannon, C.~Playoust.
  \newblock The Magma algebra system. I. The user language.
  \newblock \emph{J. Symb. Comp.}, 24(3-4):235-265, 1997.
  
\bibitem{gf2x}
  R.~Brent, P.~Gaudry, E.~Thom{\'e}, P.~Zimmermann.
  \newblock Faster multi\-plication in GF$(2)[x]$.
  \newblock In {\em ANTS'08}, 153-166. Springer, 2008.

\bibitem{Cou94}
  J.-M.~Couveignes.
  \newblock \emph{Quelques calculs en théorie des nombres}.
  \newblock PhD thesis. 1994.

\bibitem{Cou96}
  J.-M.~Couveignes. 
  \newblock Computing {$\ell$}-isogenies using the {$p$}-torsion.
  \newblock in {\em ANTS'II}, 59--65. Springer, 1996.

\bibitem{Cou00}
  J.-M.~Couveignes. 
  \newblock Isomorphisms between {A}rtin-{S}chreier towers. 
  \newblock \emph{Math. Comp.} 69(232): 1625--1631, 2000.
  
\bibitem{DFS09}
  L.~De~Feo and É.~Schost
  \newblock Fast Arithmetics in Artin Schreier Towers.
  \newblock \emph{ISSAC '09}, 2009.

\bibitem{Elk91}
  N.D.~Elikes
  \newblock \emph{Explicit Isogenies}.
  Draft, 1991.

\bibitem{EnMo03}
  A.~Enge and F.~Morain,
  \newblock Fast decomposition of polynomials with known Galois group.
  \newblock in {\em AAECC-15}, 254--264. Springer, 2003.

\bibitem{JL06}A.~Joux, R. Lercier.
  \newblock Counting points on elliptic curves in medium characteristic.
  \newblock \emph{Cryptology ePrint Archive} 2006/176, 2006.

\bibitem{vzGG} 
  J.~von~zur~Gathen and J.~Gerhard. 
  \newblock \emph{Modern Computer Algebra}. 
  \newblock Cambridge University Press, 1999.

\bibitem{Gun76}H.~Gunji,
  \newblock The Hasse Invariant and $p$-division Points of an Elliptic Curve.
  \newblock \emph{Archiv der Mathematik} 27(2), Springer, 1976.

\bibitem{Ler96}
  R.~Lercier.
  \newblock Computing isogenies in GF($2^n$).
  \newblock In {\em ANTS-II}, LNCS vol 1122, pages 197--212. Springer, 1996.

\bibitem{Ler97}
  R.~Lercier.
  \newblock Algorithmique des courbes elliptiques dans les corps finis. 
  \newblock Ph.D. Thesis, {\'E}cole polytechnique, 1997.

\bibitem{LeSi09}
  R.~Lercier, T.~Sirvent.
  \newblock On Elkies subgroups of $\ell$-torsion points in curves defined over
  a finite field.
  \newblock To appear in {\em J. Th\'eor. Nombres Bordeaux}.

\bibitem{RoSt06}
  A.~Rostovtsev and A.~Stolbunov.
  \newblock Public-key cryptosystem based on isogenies.
  \newblock Cryptology ePrint Archive, Report 2006/145.
  
\bibitem{Sat00}
  T.~Satoh.
  \newblock The canonical lift of an ordinary elliptic curve over a nite eld and its point counting.
  \newblock \emph{Journal of the Ramanujan Mathematical Society}, 2000.

\bibitem{Sch95}
  R.~Schoof.
  \newblock Counting points on elliptic curves over finite fields.
  \newblock \emph{J. Théorie des Nombres de Bordeaux} 7:219--254, 1995.

\bibitem{NTL}
  V.~Shoup.
  \newblock {NTL}: A library for doing number theory.
  \newblock \url{http://www.shoup.net/ntl/}.

\bibitem{Tes06}
  E.~Teske.
  \newblock An elliptic trapdoor system.
  \newblock {\em Journal of Cryptology}, 19(1):115--133, 2006.

\bibitem{Vel71}J.~Vélu,
  \newblock Isogénies entre courbes elliptiques.
  \newblock \emph{Comptes Rendus de l'Académie des Sciences de Paris} 273,
  Série A, 238--241, 1971.

  
  
\end{thebibliography}


\end{document}


% Local Variables:
% mode:flyspell
% ispell-local-dictionary:"british"
% mode:TeX-PDF
% End:
%
% LocalWords:  Schreier Artin pseudotrace Frobenius bivariate memoization
% LocalWords:  Couveignes isogenies
