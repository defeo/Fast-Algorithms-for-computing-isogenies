\documentclass[preprint,1p]{elsarticle}

\usepackage[english]{babel}
\usepackage[utf8]{inputenc}
\usepackage{bbm}
\usepackage{amsmath}
\usepackage{amsthm}
\usepackage{amssymb}
\usepackage{mathrsfs}
\usepackage{mdwlist}
\usepackage{graphicx}
\usepackage{mathrsfs}
\usepackage{url}
\usepackage{color}
\usepackage[all]{xypic}

%%\usepackage[pdftex,plainpages=false]{hyperref}
%%\usepackage[pdftex]{color,graphicx}
%%\usepackage{ifthen}
%%\usepackage[all]{xy}

%% Stuff
\renewcommand{\le}{\leqslant}
\renewcommand{\ge}{\geqslant}  % comme François le demande...
\newcommand{\todo}{\dots}  % un marqueur pour un trou
%% Algèbre
\newcommand{\trans}[1]{#1^\top}  % transposé
\newcommand{\dual}[1]{#1^\ast}  % dual
\newcommand{\clot}[1]{\bar{#1}}  % clôture algèbrique
\newcommand{\card}[1]{\# #1}  % cardinalité d'un ensemble
\DeclareMathOperator{\car}{car}  % caractéristique d'un corps
\DeclareMathOperator{\Frac}{Frac}  % corps des fractions
\newcommand{\N}{\mathbb{N}}  % les naturels
\newcommand{\Z}{\mathbb{Z}}  % les entiers
\newcommand{\K}{\mathbb{K}}  % un corps
\newcommand{\LK}{\mathbb{L}}  % encore un corps
\newcommand{\U}{\mathbb{U}}  % encore un corps
\newcommand{\F}{\mathbb{F}}  % un corps fini
\newcommand{\Q}{\mathbb{Q}}  % les rationnels
\newcommand{\R}{\mathbb{R}}  % les réels
\newcommand{\C}{\mathbb{C}}  % les complexes
\newcommand{\isom}{\cong}  % isomorphisme de corps
\newcommand{\frob}{\varphi}  % l'isomorphisme de frobenius
\DeclareMathOperator{\Gal}{Gal}  % groupe de Galois
\DeclareMathOperator{\Tr}{Tr}  % trace
\DeclareMathOperator{\PTr}{T}  % pseudotrace
\DeclareMathOperator{\Norm}{N} % norme
\DeclareMathOperator{\coeff}{coeff}  % coefficient
\newcommand{\res}{\rho}  % the residue form
\newcommand{\euler}{\phi}  % indicatrice d'Euler
\DeclareMathOperator{\ord}{ord}  % l'ordre d'un élément
\DeclareMathOperator{\rev}{rev}  % le reverse d'un polynôme
\newcommand{\Cyclo}{\Phi}  % le polynome cyclotomique
\DeclareMathOperator{\Fix}{Fix}  % les points fixes
%% Courbes
\newcommand{\Proj}{\mathbb{P}}  % espace projectif
\newcommand{\0}{\mathcal{O}}  % point de base d'une courbe
\newcommand{\ecpoint}[3]{[#1:#2:#3]}  % un point d'une courbe
\newcommand{\isog}[1]{\mathcal{#1}}  % la police des isogénies
\newcommand{\I}{\isog{I}}  % une isogénie I
\newcommand{\frobisog}{\phi}  % l'isogénie de frobenius
\newcommand{\Hasse}{H}  % l'invariant de Hasse
\newcommand{\divpol}{f}  % polynôme de division
%% Autre
\newcommand{\tildO}{\tilde{O}}  % la notation O~ qui oublie les log
\newcommand{\Mult}{\mathrm{\sf M}}  % fonction de multiplication
\newcommand{\ModComp}{\mathrm{\sf C}}  % fonction de composition modulaire
\newcommand{\alg}[1]{{\sf #1}}  % la police des algorithmes
\newcommand{\algref}[1]{\alg{\ref{#1}}}  % la police des algorithmes
\newcommand{\wrt}{\dashv}  % appartenance forte, a\wrt A signifie que a est représenté comme un élément de A
\newcommand{\ndiv}{\nmid}  % ne divise pas

%%Théorèmes
\newtheorem{definition}{Definition}
\newtheorem{theorem}{Theorem}
\newtheorem{lemma}[definition]{Lemma}
\newtheorem{corollary}[definition]{Corollary}
\newtheorem{proposition}[definition]{Proposition}
\newtheorem{problem}[definition]{Problem}

\journal{Journal of Number Theory}

\begin{document}

\begin{frontmatter}

\title{Fast algorithms for computing isogenies between ordinary
  elliptic curves in small characteristic}
\author{Luca De Feo}
\ead{luca.defeo@polytechnique.edu}
\address{LIX, {\'E}cole Polytechnique, 91128 Palaiseau, France}


\begin{abstract}
  The problem of computing an explicit isogeny between two given
  elliptic curves over $\F_q$, originally motivated by point counting,
  has recently awaken new interest in the cryptology community thanks
  to the works of Teske and Rostovstev \& Stolbunov.

  While the large characteristic case is well understood, only
  suboptimal algorithms are known in small characteristic; they are
  due to Couveignes, Lercier, Lercier \& Joux and Lercier \& Sirvent.
  In this paper we discuss the differences between them and run some
  comparative experiments. We also present the first complete
  implementation of Couveignes' second algorithm and present
  improvements that make it the algorithm having the best asymptotic
  complexity in the degree of the isogeny.
\end{abstract}

\begin{keyword}
  Elliptic curves \sep Isogenies \sep Cryptography \sep Algorithms
\end{keyword}

\end{frontmatter}

\section{Introduction}

The problem of computing an explicit degree $\ell$ isogeny between two
given elliptic curves over $\F_q$ was originally motivated by point
counting methods based on Schoof's algorithm \cite{Atk91},
\cite{Elk91}, \cite{Sch95}. A review of the most efficient algorithms
to solve this problem is given in \cite{BoMoSaSc08} together with a
new quasi-optimal algorithm; however, all the algorithms presented in
\cite{BoMoSaSc08} compute the power series expansion of the
Weierstrass functions of the curves up to a certain precision, thus
they are limited to the case $\ell\ll p$ where $p$ is the
characteristic of $\F_q$. This is satisfactory for cryptographic
applications where one takes $p=q$ or $p=2$; indeed in the former case
Schoof's algorithm needs $\ell\in O(\log p)$, while in the latter case
there's no need to compute explicit isogenies since $p$-adic methods
based on \cite{Sat00} are preferred to Schoof's algorithm.

Nevertheless, the problem of computing explicit isogenies in the case
where $p$ is small compared to $\ell$ stays of theoretical interest
and can find practical applications in newer cryptosystems such as
\cite{Tes06}, \cite{RoSt06}. The first algorithm to solve this problem
was given by Couveignes and made use of formal groups \cite{Cou94}; it
takes $\tildO(\ell^3\log q)$ operations in $\F_p$ assuming $p$ is
constant, however it has an exponential complexity in $\log
p$. Another algorithm by Lercier specific to $p=2$ uses some linear
properties of the problem to build a linear system from whose solution
the isogeny can be deduced \cite{Ler96}; its complexity is conjectured
to be $\tildO(\ell^3\log q)$ operations in $\F_p$, but it has a much
better constant factor than \cite{Cou94}. At the moment we write the
latter algorithm is by many orders of magnitude the fastest algorithm
to solve practical instances of the problem when $p=2$, thus being the
\emph{de facto} standard for cryptographic use.

$p$-adic methods were used by Joux and Lercier \cite{JL06} and Lercier
and Sirvent \cite{LeSi09} to solve the isogeny problem. The former
method has complexity $\tildO(\ell^2(1 + \ell/p)\log q)$ operations in
$\F_p$, which makes it well adapted to the case where $p\sim\log
q$. The latter has complexity $\tildO(\ell^3 + \ell\log q^2)$
operations in $\F_p$, making it the best algorithm to our knowledge
for the case where $p$ is not constant.

\paragraph{The algorithm C2 and its variants}
Finally, the algorithm having the best asymptotic complexity in $\ell$
was proposed again by Couveignes in \cite{Cou96}, we will refer to
this original version as ``C2''. Its complexity --supposing $p$ is
fixed-- was estimated in \cite{Cou96} as being $\tildO(\ell^2\log q)$
operations in $F_p$, but with a precomputation step requiring
$\tildO(\ell^3\log q)$ operations and large memory
requirements. However this estimate was wrong as we will argue in
Section \ref{sec:C2} and C2 has an overall asymptotic complexity of
$\tildO(\ell^3\log q)$ operations.

Subsequent work by Couveignes \cite{Cou00} and more recently
\cite{DFS09}, use Artin-Schreier theory to avoid the precomputation
step of C2 and drop the memory requirements to $\tildO(\ell\log q +
\log^2 q)$ elements of $\F_p$. However, this is still not enough to
reduce the overall complexity of the algorithm as we will argue in
Section \ref{sec:C2-AS}. We refer to this variant as ``C2-AS''.

In the present paper we give a complete review of Couveignes'
algorithm, we present new variants that reach the foreseen quadratic
bound in $\ell^2$ and prove the accurate complexity estimate of
$\tildO(\todo)$ operations which doesn't suppose $p$ to be
fixed. In Section \ref{sec:} we also extend the algorithm beyond its
original scope to solve the more general problem of \emph{bounded
  degree isogeny finding}; interestingly, this new algorithm turns out
to have a better asymptotic complexity than any previously known
approach.

\paragraph{Notation and plan}
In the rest of the paper $p$ is a prime, $d$ a positive integer,
$q=p^d$ and $\F_q$ is the field with $q$ elements. For an elliptic
curve $E$ and a field $\K$ embedded in an algebraic closure
$\clot{\K}$, we note by $E(\K)$ the set of $\K$-rational points and by
$E[m]$ the $m$-torsion subgroup of $E(\clot{\K})$. The group law on
the elliptic curve is noted additively, its zero is the point at
infinity, noted $\0$. For an affine point $P$ we note by $x(P)$ its
abscissa and by $y(P)$ its ordinate. We will restrict ourselves to the
case of ordinary elliptic curves, thus $E[p^k]\isom\Z/p^k\Z$.

Unless otherwise stated, all time complexities will be measured in
number of operations in $\F_p$ and all space complexities in number of
elements of $\F_p$; we do not assume $p$ to be constant. We use the
$O$, $\Theta$ and $\Omega$ notations to state respectively upper
bounds, tight bounds and lower bounds for asymptotic complexities. We
also use the notation $\tildO_x$ that forgets polylogarithmic factors
in the variable $x$, thus $O(xy\log x \log y)\subset\tildO_x(xy\log
y)\subset\tildO_{x,y}(xy)$. We simply note $\tildO$ when the variables
are clear from the context. We let $\Mult:\N\rightarrow\N$ be a
\emph{multiplication function}, such that polynomials of degree at
most $n$ with coefficients in $\F_p$ can be multiplied in $\Mult(n)$
operations, under the conditions of \cite[Ch. 8.3]{vzGG}. Typical
orders of magnitude are $O(n^{\log_23})$ for Karatsuba multiplication
or $O(n\log n\log\log n)$ for FFT multiplication. We let
$2<\omega\le3$ be the exponent of linear algebra, that is an integer
such that $n\times n$ matrices can be multiplied in $n^\omega$
operations.

The plan is the following.\\
Preliminaries on isogenies\\
C2\\
extension to $p=2$\\
complexity of C2\\
C2-AS\\
FAAST\\
interpolation\\
complexity of C2-AS-FI\\
modular composition\\
complexity of C2-AS-FI-MC\\
bounded degree problem\\
complexity of CBD-AS-FI-MC, comparison to other methods\\
implementation\\
benchmarks\\



% Local Variables:
% mode:flyspell
% ispell-local-dictionary:"british"
% mode:TeX-PDF
% TeX-master: "ec-isogeny"
% End:
%
% LocalWords:  Schreier Artin pseudotrace frobenius bivariate Joux Sirvent FFT
% LocalWords:  Couveignes isogenies Schoof isogeny cryptosystems Lercier
% LocalWords:  precomputation arithmetics polylogarithmic Karatsuba

\section{Preliminaries on Isogenies}

Let $E$ be an ordinary elliptic curve over the field $\F_q$, we
suppose it is given to us as the locus of zeroes of an affine
Weierstrass equation
\[y^2 + a_1xy + a_3y = x^3 + a_2x^2 + a_4x + a_6
\qquad a_1,\ldots,a_6\in\F_q\text{.}\]

\paragraph{Simplified forms} If $p>3$ it is well known that the curve
$E$ is isomorphic to a curve in the form
\begin{equation}
  \label{eq:weierstrass>3}
  y^2 = x^3 + ax + b
\end{equation}
and its $j$-invariant is $j(E) = \frac{1728(4a)^3}{16(4a^3 + 27b^2)}$.

When $p=3$, since $E$ is ordinary, it is isomorphic to a curve
\begin{equation}
  \label{eq:weierstrass=3}
  y^2 = x^3 + ax^2 + b
\end{equation}
and its $j$-invariant is $j(E) = -\frac{a^3}{b}$.

Finally, when $p=2$, since $E$ is ordinary, it is isomorphic to a curve
\begin{equation}
  \label{eq:weierstrass=2}
  y^2 + xy = x^3 + ax^2 + b
\end{equation}
and its $j$-invariant is $j(E) = \frac{1}{b}$.

These isomorphism are easy to compute and we will always assume that
the elliptic curves given to our algorithms are in such simplified
forms.

\paragraph{Isogenies}
Elliptic curves are endowed with the classic group structure through
the chord-tangent law. A group morphism having finite kernel and
sending $\0$ over $\0$ is called an \emph{isogeny}. An isogeny whose
kernel has cardinality $\ell$ is called a degree $\ell$ isogeny or
simply an $\ell$-isogeny. Isogenies are algebraic maps, as such they
can be represented by rational functions. An isogeny is said to be
$\K$-rational if it is $\K$-rational as algebraic map.

One important property about isogenies is that they factor the
multiplication-by-$m$ map.

\begin{definition}[Dual isogeny]
  Let $\I : E \rightarrow E'$ be a degree $m$ isogeny. There exists an
  unique isogeny $\hat{\I} : E' \rightarrow E$, called the \emph{dual
    isogeny} such that
  \[\I\circ\hat{\I} = [m]_E \qquad\text{and}\qquad \hat{\I}\circ\I =
  [m]_{E'}\]
\end{definition}

As algebraic maps, isogenies can be separable, inseparable or purely
inseparable. In the case of finite fields, purely inseparable
isogenies are easily understood as powers of the frobenius map. Let
\[E^{(p)} : y^2 + a_1^pxy + a_3^py = x^3 + a_2^px^2 + a_4^px + a_6^p\]
then the map
\begin{align*}
  \frobisog : E &\rightarrow E^{(p)}\\
          (x,y) &\mapsto (x^p,y^p)
\end{align*}
is a degree $p$ purely inseparable isogeny. Any purely inseparable
isogeny is a composition of such frobenius isogenies.

Let $E$ and $E'$ be two elliptic curves defined over $\F_q$, by
finding an \emph{explicit isogeny} we mean to find an
($\F_q$-rational) rational function from $E(\clot{\F_q})$ to
$E'(\clot{\F_q})$ such that the map it defines is an isogeny.

\begin{figure}
  \centering
  \[\xymatrix{
    E \ar[r]^{[m]}\ar@/_1pc/[rrr]_{\I'} & E \ar[r]^\I & E' \ar[r]^{\frobisog^n} & E'^{(p^n)}\\
  }\]
  \caption{Factorization of an isogeny. $\I'$ has kernel $E[m]\oplus\ker\I$.}
\end{figure}

Since an isogeny can be uniquely factored in the product of a
separable and a purely inseparable isogeny, we focus ourselves on the
problem of computing explicit separable isogenies. Furthermore one can
factor out multiplication-by-$m$ maps, thus reducing the problem to
compute explicit separable isogenies with cyclic kernel.

In the rest of this paper, unless otherwise stated, by $\ell$-isogeny
we mean a separable isogeny with kernel isomorphic to $\Z/\ell\Z$.

\paragraph{Vélu formulae}
For any finite subgroup $G \subset E(\clot{\K})$, Vélu formulae
\cite{Vel71} give an elliptic curve $\tilde{E}$ and an explicit
isogeny in a canonical way. The isogeny is $\K$-rational if and only
if the polynomial vanishing on the abscissae of $G$ belongs to
$\K[X]$.

In practice, if $E$ is defined over $\F_q$ and if
\[h(X) = \prod_{\substack{P\in G\\P\ne\0}}(X - x(P)) \in \F_q[X]\]
is known, Vélu formulae compute a rational function
\begin{equation}
  \label{eq:isog}
  \I(x,y) = \left(\frac{g(x)}{h(x)}, \frac{k(x,y)}{l(x)}\right)  
\end{equation}
and a curve $\tilde{E}$ such that $\I : E\rightarrow\tilde{E}$ is an $\F_q$-rational
isogeny of kernel $G$. A consequence of Vélu formulae is
\begin{equation}
  \label{eq:velu-deg}
  \deg g = \deg h + 1 = \card{G}
  \text{.}
\end{equation}

Given two curves $E$ and $E'$, Vélu formulae reduce the problem of
finding an explicit isogeny between $E$ and $E'$ to that of finding
the kernel of an isogeny between them. Once the polynomial $h(X)$
vanishing on $\ker\I$ is found, the explicit isogeny is computed
composing Vélu formulae with the isomorphism between $\tilde{E}$ and
$E'$ as in figure \ref{fig:velu}.

\begin{figure}
  \centering
  \[\xymatrix{
    E \ar[r]^{\tilde{\I}} \ar[rd]^\I & \tilde{E} \ar[d]^{\simeq}\\
    & E'
  }\]
  \caption{Using Vélu formulae to compute an explicit isogeny.}
  \label{fig:velu}
\end{figure}




% Local Variables:
% mode:flyspell
% ispell-local-dictionary:"british"
% mode:TeX-PDF
% TeX-master: "ec-isogeny"
% End:
%
% LocalWords:  Schreier Artin pseudotrace frobenius bivariate Joux Sirvent FFT
% LocalWords:  Couveignes isogenies Schoof isogeny cryptosystems Lercier
% LocalWords:  precomputation arithmetics polylogarithmic Karatsuba
% LocalWords:  endomorphisms

\section{The algorithm C2}
\label{sec:C2}

The algorithm we refer to as C2 was originally proposed in
\cite{Cou96}. It takes as input two elliptic curves $E, E'$ and an
integer $\ell$ prime to $p$ and it returns, if it exists, an
$\F_q$-rational isogeny of degree $\ell$ between $E$ and $E'$. It only
works in odd characteristic.

\paragraph{The idea} Suppose it exists an $F_q$-rational isogeny
$\I:E\rightarrow E'$ of degree $\ell$. Since $\ell$ is prime to $p$
one has $\I(E[p^k]) = E'[p^k]$.

Recall that $E[p^k]$ and $E'[p^k]$ are cyclic groups, C2 iteratively
computes generators $P_k,P_k'$ of $E[p^k]$ and $E'[p^k]$
respectively. Now C2 makes the guess $\I(P_k) = P_k'$, then, if $\I$
is given by rational fractions as in \eqref{eq:isog},
\begin{equation}
  \label{eq:C2:I}
  \frac{g\bigl(x([i]P_k)\bigr)}{h\bigl(x([i]P_k)\bigr)} = x([i]P_k')
  \quad\text{for $i\in\Z/p^k\Z$} 
\end{equation}
and by \eqref{eq:velu-deg} $\deg g = \deg h + 1 = \ell$.

Using \eqref{eq:C2:I} one can compute the rational fraction
$\frac{g(X)}{h(X)}$ through Cauchy interpolation over the points of
$E[p^k]$. C2 takes $p^k > 4\ell - 2$, interpolates the rational
fraction and then checks that it corresponds to the restriction of an
isogeny to the $x$-axis. If this is the case, the whole isogeny is
computed through Vélu formulae and the algorithm terminates. Otherwise
the guess $\I(P_k) = P_k'$ was false, then C2 computes a new generator
for $E'[p^k]$ and starts over again.

We now go through the details of the algorithm.

\paragraph{The $p$-torsion}
The computation of the $p$-torsion points follows from the work of
Gunji \cite{Gun76}. Here we suppose $p\ne2$.

\begin{definition}
  \label{def:hasse}
  Let $E$ have equation $y^2 = f(x)$, one defines the \emph{Hasse
    invariant} of $E$, noted $H_E$ as the coefficient of $X^{p-1}$ in
  $f(X)^{\frac{p-1}{2}}$.
\end{definition}

Gunji shows the following proposition and gives formulae to compute
the $p$-torsion points.

\begin{proposition}
  \label{th:gunji}
  Let $c=\sqrt[p-1]{H_E}$, then the $p$-torsion points of $E$ are
  defined in $\F_q[c]$ and their abscissae are defined in $\F_q[c^2]$.
\end{proposition}


\paragraph{The $p^k$-torsion}
$p^k$-torsion points are iteratively computed via $p$-descent. The
basic idea to split the multiplication map as $[p] = \frobisog\circ V$
and invert each of the components. The purely inseparable isogeny
$\frobisog$ is just a frobenius map and the separable isogeny $V$ can
be computed by Vélu formulae once the $p$-torsion points are
known. Although this is reasonably efficient, pulling $V$ back may
involve factoring polynomials of degree $p$ in some extension field.

A finer way to do the $p$-descent, as suggested in the original paper
\cite{Cou96}, is to use the works of Voloch \cite{Vol90}. Suppose
$p\ne2$, let $E$ and $\widetilde{E}$ have equations respectively
\begin{align*}
  y^2&=f(x)=x^3+a_2x^2+a_4x+a_6 \;\text{,}\\
  \tilde{y}^2&=\tilde{f}(\tilde{x}) = \tilde{x}^3 +
  \sqrt[p]{a_2}\tilde{x}^2 + \sqrt[p]{a_4}\tilde{x} + \sqrt[p]{a_6}
  \;\text{,}
\end{align*}
set
 \begin{equation}
  \label{eq:voloch:cover}
  \tilde{f}(X)^{\frac{p-1}{2}} = \alpha(X) + H_{\widetilde{E}}X^{p-1} + X^p\beta(X)
\end{equation}
with $\deg \alpha < p-1$ and $H_{\widetilde{E}}$ the Hasse invariant
of $\widetilde{E}$. Voloch shows the following.

\begin{proposition}
  \label{th:voloch}
  Let $\tilde{c} = \sqrt[p-1]{H_{\widetilde{E}}}$, the cover of
  $\widetilde{E}$ defined by
  \begin{equation}
    \label{th:voloch:cover}
    C:\; z^p - z = \frac{\tilde{y}\beta(\tilde{x})}{\tilde{c}^p}
  \end{equation}
  is an étale cover of degree $p$ and is isomorphic to $E$ over
  $\F_q[\tilde{c}]$; the isomorphism is given by
  \begin{equation}
    \label{th:voloch:isom}
    \left\{
      \begin{aligned}
        (\tilde{x}, \tilde{y}) &= V(x, y)\\
        z &= -\frac{y}{\tilde{c}^p}\sum_{i=1}^{(p-1)}\frac{1}{x - x([i]P_1)}
      \end{aligned}
    \right.
  \end{equation}
  where $P_1$ is a primitive $p$-torsion point of $E$.
\end{proposition}

The descent is then performed as follows: starting from a point $P$ on
$E$, first pull it back along $\frobisog$, then take one of its
pre-images in $C$ by solving equation \eqref{th:voloch:cover}, finally
use equation \eqref{th:voloch:isom} to land on a point $P'$ in $E$.
The proposition guarantees that $[p]P' = P$. The descent is pictured
in figure \ref{fig:voloch}.

\begin{figure}
  \centering
  \[
  \xymatrix{\widetilde{E}\ar@/^/[r]^{\frobisog} & E\ar@/^/[l]^{V}}
  %%
  \qquad
  %%
  \xymatrix{
    \widetilde{E}\ar@/^/[r]^{\frobisog} & E\ar@/^/@{-->}[l]^{V}\ar[d]_{\simeq}\\
    & C\ar@/^/[ul]
  }
  \]
  
  \caption{Two ways of doing the $p$-descent: standard on the left and via a degree $p$ cover on the right}
  \label{fig:voloch}
\end{figure}


The reason why this is more efficient than a standard descent is the
shape of equation \eqref{th:voloch:cover}: it is an Artin-Schreier
equation and it can be solved by many techniques, the simplest being
linear algebra (as it was suggested in \cite{Cou96}). Once a solution
$z$ to \eqref{th:voloch:cover} is known, solving in $x$ and $y$ the
bivariate polynomial system \eqref{th:voloch:isom} takes just a GCD
computation (explicit formulae were given by Lercier in
\cite[$\S$6.2]{Ler97}, we give some slightly improved ones in Section
\ref{sec:implementation}). Compare this with a generic factoring
algorithm needed by standard descent.

Solving Artin-Schreier equations is the most delicate task of the
descent and we will further discuss it.


\paragraph{Cauchy interpolation}
Interpolation reconstructs a polynomial from the values it takes on
some points, Cauchy interpolation reconstructs a rational
fraction. The Cauchy interpolation algorithm is divided in two phases:
first find the polynomial $P$ interpolating the evaluation points,
then use rational fraction reconstruction to find a rational fraction
congruent to $P$ modulo the polynomial vanishing on the points. The
first phase is carried out through any classical interpolation
algorithm, while the second is an XGCD computation. See
\cite[$\S$5.8]{vzGG} for details.

Cauchy interpolation needs $n+2$ points to reconstruct a degree
$(k,n-k)$ rational fraction. This, together with \eqref{eq:velu-deg},
justifies the choice of $k$ such that $p^k > 4\ell - 2$. Some of our
variants of C2 will interpolate only on the primitive $p^k$-torsion
points, thus requiring the slightly larger bound $\euler(p^k) \ge
4\ell - 2$. This is not very important to our asymptotical analysis
since in both cases $p^k \in O(\ell)$.

\paragraph{Recognising the isogeny}
Once the rational fraction $\frac{g(X)}{h(X)}$ has been computed, one
has to verify that it is indeed an isogeny. The first test is to check
that the degrees of $g$ and $h$ match equation \eqref{eq:velu-deg}, if
they don't, the equation can be discarded right away and the algorithm
can go on with the next trial. Next, one can check that $h$ is indeed
the square of a polynomial (or, if $\ell$ is even, the product of one
factor of the $2$-division polynomial and a square polynomial). This
two tests are usually enough to detect an isogeny, but, should they
lie, one can still check that the resulting rational function is
indeed a group morphism by trying some random points on $E$. We don't
dip into the analysis of the probabilities of success of these tests
as this will be done in detail in Section \ref{sec:bounded}.


\subsection{The case $p=2$}
\label{sec:p=2}
The algorithm as we have presented it only works when $p\ne2$, it is
however an easy matter to generalise it. The only phase that doesn't
work is the computation of the $p^k$-torsion points. For curves in the
form \eqref{eq:weierstrass=2} the only $2$-torsion point is
$(0,\sqrt{b})$.

Voloch formulae are hard to adapt, nevertheless a $2$-descent on the
Kummer surface of $E$ can easily be performed since the doubling
formula reads
\begin{equation}
  x([2]P) = \frac{b}{x(P)^2} + x(P)^2 =
  \frobisog\left(\frac{\sqrt{b} + x(P)^2}{x(P)} \right) = \frobisog\circ V
  \;\text{.}
\end{equation}
Given point $x_P$ on $K_E$, a pull-back along $\frobisog$ gives a
point $\tilde{x}_P$ on $K_{\widetilde{E}}$. Then pulling $V$ back
amounts to solve
\begin{equation}
  \label{eq:2-descent}
  x^2 + \tilde{x}_Px = \sqrt{b}
\end{equation}
and this can be turned in an Artin-Schreier equation through the
change of variables $x \rightarrow x'\tilde{x}_P$.

From the descent on the Kummer surfaces one could deduce a full
$2$-descent on the curves by solving a quadratic equation at each step
in order recover the $y$ coordinate, but this would be too
expensive. Fortunately, the $y$ coordinates are not needed by the
subsequent steps of the algorithm, thus one may simply ignore
them. Observe in fact that even if $K_E$ does not have a group law,
the restriction of scalar multiplication is well defined and can be
computed through Montgomery formulae \cite{Mon87}. This is enough to
compute all the abscissae of the points in $E[p^k]$ once a generator
is known.


\subsection{Complexity analysis}
\label{sec:C2:complexity}
Analysing the complexity of C2 is a delicate matter since the
algorithm relies on some unspecified computer algebra algorithms in
order to deal with finite extensions of $\F_q$. The choice of the
actual algorithms may strongly influence the overall complexity of C2.
In this section we will only give some lower bounds on the complexity
of C2, since a much more accurate complexity analysis will be carried
out in Section \ref{sec:C2-AS}.

\paragraph{$p$-torsion}
Applying Gunji formulae first requires to find $c$ and $c'$, $p-1$-th
roots of $H_E$ and $H_{E'}$, and build the field extension $\F_q[c] =
\F_q[c']$. Independently of the actual algorithm used, observe that in
the worst case $F_q[c]$ is a degree $p-1$ extension of $F_q$, thus
simply representing one of its elements requires $\Theta(pd)$ elements
of $\F_p$.

Subsequently, the main cost in Gunji's formulae is the computation of
the determinant of a $\frac{p-1}{2}\times\frac{p-1}{2}$
quadri-diagonal matrix (see \cite{Gun76}). This takes $\Theta(p^2)$
operations in $\F_q[c]$, that is no less than $\Omega(p^3d)$
operations in $\F_p$.

\paragraph{$p^k$-torsion}
During the $p$-descent, factoring of equations \eqref{th:voloch:cover}
or \eqref{eq:2-descent} may introduce some field extensions over
$\F_q[c]$. Observe that an Artin-Schreier polynomial is either
irreducible or totally split, so at each step of the $p$-descent we
either stay in the same field or we take a degree $p$ extension. This
shows that in the worse case, we have to take an extension of degree
$p^{k-1}$ over $F_q[c]$. The following proposition, which is a
generalisation of \cite[Prop. 26]{Ler97}, states precisely how likely
this case is.

\begin{proposition}
  \label{th:tower}
  Let $E$ be an elliptic curve over $\F_q$, we note $\U_i$ the
  smallest field extension of $\F_q$ such that $E[p^i]\subset
  E(\U_i)$. For any $i\ge1$, either $[\U_{i+1}:\U_i] = p$ or
  $\U_{i+1}=\U_i=\cdots=\U_1$.
\end{proposition}
\begin{proof}
  The proof in \cite{Ler97} is false. Patching it needs some more
  calculations in the $p$-adics.
  \begin{gather}
    t_{i+1} == t_i mod p^i\\
    t_{i+1} = t_i + ap^i\\
    \\
    t_i == t_{i-1} mod p^{i-1}\\
    t_i = t_{i-1} + fp^{i-1}\\
    \\
    t_{i-1}^{p^{k-1}} = 1 + cp^{i-1}\\
    t_i^{p^{k-1}} = 1 + cp^{i-1} + p^{k-1}t_{i-1}^{p^{k-1}-1}fp^{i-1} + ... !== 1 mod p^i\\
    t_i^{p^k} == 1 mod p^i\\
    \\
    t_i^{p^{k-1}} = 1 + gp^{i-1} \\
    t_i^{p^k} = 1 + gp^{i} + ... = 1 + hp^{i}\\
    t_{i+1}^{p^k} = t_i^{p^k} + p^kt_i^{p^k-1}ap^i + O(p^{2i}) == t_i^{p^k} == 1 + hp^i != 1 mod p^{i+1}\\
  \end{gather}
\end{proof}

Thus for any elliptic curve there is an $i_0$ such that $[\U_i:\U_1] =
p^{i-i_0}$ for any $i \ge i_0$. This shows that the worse and the
average case coincide since asymptotically $[\U_k:\U_1] \in
\Theta(p^k)$. In this situation, one needs $\Theta(p^kd)$ elements of
$\F_p$ to store an element of $\U_k$.

Now the last iteration of the $p$-descent needs to solve an
Artin-Schreier equation in $\U_k$. To do this C2 precomputes the
matrix of the $F_q$-linear application $(x^q-x):\U_k\rightarrow\U_k$
and its inverse, plus the matrix of the $F_p$-linear application
$(x_p-x):\F_q\rightarrow\F_q$ and its inverse. The former is the most
expensive one and takes $\Theta(p^{\omega k})$ operations in $\F_q$,
that is $\Omega(p^{\omega k}d) = \Omega(\ell^\omega d)$ operations in
$\F_p$, plus a storage of $\Theta(\ell^2d)$ elements of
$\F_p$. Observe that this precomputation may be used to compute any
other isogeny.

After the precomputation has been done, C2 successively applies the
two inverse matrices; details can be found in
\cite[$\S$2.4]{Cou96}. This costs at least $\Omega(\ell^2d)$.


\paragraph{Interpolation}
The most expensive part of Cauchy interpolation is the polynomial
interpolation phase. In fact, simply representing a polynomial of
degree $p^k-1$ in $\U_k[X]$ takes $\Theta(p^{2k}d)$ elements, thus at
least $\Omega(\ell^2d)$ operations are needed to interpolate unless
special care is taken. Mistakingly, this contribution due to
arithmetics in $\U_k$ had been underestimated in the complexity
analysis of \cite{Cou96}, thus leading to the wrong complexity
estimate of $\Omega(\ell d\log\ell)$ for this phase. We will give more
details on interpolation in Section \ref{sec:C2-AS-FI}.


\paragraph{Recognising the isogeny}
The cost of testing for squareness of the denominator and other tests
is negligible compared to the rest of the algorithm. Nevertheless it
is important to realize that on average half of the $\euler(p^k)$
mappings from $E[p^k]$ to $E'[p^k]$ must be tried before finding the
isogeny, for only one of these mappings corresponds to it. This
implies that the Cauchy interpolation step must be repeated an average
of $\Theta(p^k)$ times, thus contributing a $\Omega(\ell^3d)$ to the
total complexity.

Summing up all the contributions one ends up with the following lower
bound
\begin{equation}
  \label{eq:C2:complexity}
  \Omega(\ell^3d + p^3d)
\end{equation}
plus a precomputation step whose cost is negligible compared to this
one and a space requirement of $\Theta(\ell^2d)$ elements. In the next
sections we will see how to make all this costs drop.


\subsection{The case $(p,\ell)\ne1$}
\label{sec:C2:non-prime}
If we are interested in finding a separable isogeny whose degree is
not prime to $p$, the best way is to compute the curve $\widetilde{E}$
such that $E = \widetilde{E}^{(p)}$, then compute an isogeny of degree
$\ell/p$ between $\widetilde{E}$ and $E'$ and finally compose it with
the separable $p$-isogeny $V$ from $E$ to $\widetilde{E}$.

Observe however that C2 can be easily adapted to directly compute such
an isogeny. In fact let $v=v_p(\ell)$, then $\I(E[p^k]) =
E'[p^{k-v}]$. All one needs to do in this case is to modify the Cauchy
interpolation so that it interpolates the rational function that sends
a generator of $E[p^k]$ over a generator of $E'[p^{k-v}]$ and the
other points accordingly. The maximum number of trials to do before
finding the isogeny is $\euler(p^{k-v})$, thus the overall complexity
is
\begin{equation}
  \label{eq:C2:complexity-non-prime}
  \Omega\left(\frac{\ell^3}{p^v}d + p^3d\right)
  \;\text{.}
\end{equation}

Although this method is less efficient than the first one, it will
come handy in Section \ref{sec:bounded}.



% Local Variables:
% mode:flyspell
% ispell-local-dictionary:"british"
% mode:TeX-PDF
% TeX-master: "ec-isogeny"
% End:
%
% LocalWords:  Schreier Artin pseudotrace frobenius bivariate Joux Sirvent FFT
% LocalWords:  Couveignes isogenies Schoof isogeny cryptosystems Lercier
% LocalWords:  precomputation arithmetics polylogarithmic Karatsuba precomputes
% LocalWords:  endomorphisms asymptotical

\section{The algorithm C2-AS}
\label{sec:C2-AS}

One of the most expensive steps of C2 is the resolution of an
Artin-Schreier equation in an extension field $\U_i$. In \cite{Cou00}
Couveignes gives an approach alternative to linear algebra to solve
this problem. First it builds the whole tower $(\U_1=\F_q[c], \ldots,
\U_k)$ of intermediate extensions, then it solves an Artin-Schreier
equation in $\U_i$ recursively by reducing it to another
Artin-Schreier equation in $\U_i$. Details are in \cite{Cou00} and
\cite{DFS09}.

To solve the final Artin-Schreier equation in $\U_1=\F_q[c]$ one
resorts to linear algebra, thus precomputing the inverse matrix of the
$\F_p$-linear application $(x^p-x):\U_1\rightarrow\U_1$.


\subsection{Complexity analysis}
\label{sec:C2-AS:complexity}
How effective this method is depends on the way algebra is performed
in the tower $(\U_1,\ldots,\U_k)$. The present author and Schost
\cite{DFS09} recently presented a new construction based on
Artin-Schreier theory that allows to do most arithmetic operations in
the tower in quasi-linear time. Assuming this construction is used, we
can now give precise bounds for each step of C2-AS.

\paragraph{$p$-torsion}
The construction of $\F_q[c]$ may be done in many ways. The only
requirements of \cite{DFS09} are
\begin{enumerate}
\item that its elements have a representation as elements of
  $F_p[X]/Q_1(X)$ for some irreducible polynomial $Q_1$,
\item that either $(d,p)=1$ or $\deg Q_1' + 2 = \deg Q_1$.
\end{enumerate}
Selecting a random polynomial $Q_1$ and testing for irreducibility is
usually enough to meet these conditions. This costs
$O\bigl(pd\Mult(pd)\log (pd)\log(p^2d)\bigr)$ according to
\cite[Th. 14.42]{vzGG}.

Now we need to compute the embedding $\F_q\subset\F_q[c]$. Supposing
$\F_q$ is represented as $\F_p[X]/Q_0(X)$, we factor $Q_0$ in
$\F_q[c]$, which costs $O\bigl(pd\Mult(pd^2)\log d\log p\bigr)$ using
\cite[Coro. 14.16]{vzGG}. Then the most naive technique to express the
embedding is linear algebra. This requires the computation of $pd$
elements of $\F_q[c]$ at the expense of $\Theta\bigl(pd\Mult(pd)\bigr)$
operations in $\F_p$, then the inversion of the matrix holding such
elements, at a cost of $\Theta\bigl((pd)^\omega\bigr)$ operations. This is
certainly not optimal, yet this phase will have negligible cost
compared to the rest of the algorithm.

Now we can compute $c$ and $c'$ by factoring the polynomials
$Y^{p-1}-H_E$ and $Y^{p-1}-H_{E'}$ in $\F_p[X]/Q_1(X)$. This costs
\[O\bigl((p\ModComp(pd) + \ModComp(p)\Mult(pd) + \Mult(p)\Mult(pd)\log
p)(\log^2 p+\log d)\bigr)\]
using \cite[Section 3]{KS97}.

Finally, computing the determinants needed by Gunji's formulae takes
$\Theta(p^2)$ multiplications in $\F_q[c]$, that is
$\Theta\bigl(p^2\Mult(pd)\bigr)$.

Letting out logarithmic factors, the overall cost of this phase is
\begin{equation}
  \label{eq:gunji-complexity}
  \tildO\bigl(p^2d^3 + p\ModComp(pd) + \ModComp(p)pd + (pd)^\omega \bigr)
\end{equation}


\paragraph{$p^k$-torsion}
Application of Voloch formulae requires at each of the levels
$\U_2,\ldots,\U_k$
\begin{enumerate}
\item to solve equation \eqref{th:voloch:cover} by factoring an
  Artin-Schreier polynomial,
\item to solve the system \eqref{th:voloch:isom}.
\end{enumerate}
If we assume the worst case $[\U_2:\U_1] = p$, according to
\cite[Th. 21]{DFS09}, at each level $i$ the first step costs
\begin{gather*}
  O\bigl((pd)^\omega i + {\sf PT}(i-1) + \Mult(p^{i+1}d)\log p\bigr)\\[0.3cm]
  \begin{aligned}
    \text{where}&&
    {\sf PT}(i) &= O\bigl((pi + \log(d))i{\sf L}(i) + p^i\ModComp(pd)\log^2(pd)\bigr)\\
    \text{and}&&
    {\sf L}(i) &= O\bigl(p^{i+2}d\log_p^2{p^{i+1}d} + p\Mult(p^{i+1}d)\bigr)
    \text{ ;}
  \end{aligned}
\end{gather*}
while the second takes the GCD of two degree $p$ polynomials in
$\U_i[X]$ for each $i$ (see Section \ref{sec:implementation}), at a
cost of $O\bigl(\Mult(p^{i+1}d)\log p\bigr)$ operations using a fast
algorithm \cite[$\S$11.1]{vzGG}.

Summing up over $i$, the total cost of this phase up to logarithmic
factors is
\begin{equation}
  \label{eq:C2-AS:complexity:p^k}
  \tildO_{p,d,\log\ell}\left((pd)^\omega \log_p^2\ell + p^2\ell d\log_p^4\ell +
  \frac{\ell}{p}\ModComp(pd)\right)
  \;\text{.}  
\end{equation}
Also notice that there is no more need to store a $p^{k-1}d\times
p^{k-1}d$ matrix to solve the Artin-Schreier equation, thus the space
requirements are not anymore quadratic in $\ell$.


\paragraph{Interpolation}
The interpolation phase is not essentially changed: one needs first to
interpolate a degree $p^k-1$ polynomial with coefficients in $\U_k$,
then use \cite[\alg{Push-down}]{DFS09} to obtain the corresponding
polynomial in $\F_q[X]$ and finally do a rational fraction
reconstruction.

The first step costs $O\bigl(\Mult(p^{2k}d)\log p^k\bigr)$ using fast techniques
as \cite[$\S$10.2]{vzGG}, then converting to $\F_q[c][X]$ takes
$O\bigl(p^k{\sf L}(k-1)\bigr)$ by \cite{DFS09} and further converting to
$\F_q[X]$ takes $\Theta\bigl((pd)^2\bigr)$ by linear algebra. The rational function
reconstruction then takes $O\bigl(\Mult(p^kd)\log p^k\bigr)$ using
fast GCD techniques \cite[$\S$11.1]{vzGG}.

The overall complexity of one interpolation is then
\begin{equation}
  \label{eq:C2-AS:complexity:interp}
  O\bigl(\Mult(\ell^2d)\log_p\ell + \ell{\sf L}(k-1) + (pd)^2\bigr)
  \;\text{.}
\end{equation}
Remember that this step has to be repeated an average number of
$\euler(p^k)/4$ times, thus the dependency of C2-AS in $\ell$ is still cubic.



% Local Variables:
% mode:flyspell
% ispell-local-dictionary:"british"
% mode:TeX-PDF
% TeX-master: "ec-isogeny"
% End:
%
% LocalWords:  Schreier Artin pseudotrace Frobenius bivariate Joux Sirvent FFT
% LocalWords:  Couveignes isogenies Schoof isogeny cryptosystems Lercier
% LocalWords:  precomputation arithmetics polylogarithmic Karatsuba
% LocalWords:  irreducibility

\section{The algortihm C2-AS-FI}
\label{sec:C2-AS-FI}

The most expensive step of C2-AS is the polynomial interpolation step
which is part of the Cauchy interpolation. If we use a standard
interpolation algorithm, its input consists in a list of $\Theta(p^k)$
pairs $\bigl(P, \I(P)\bigr)$ with $P\in\U_k$, thus a lower bound for
any such algorithm is $\Omega(p^{2k}d)$. Notice however that the
output is a polynomial of degree $\Theta(p^k)$ in $\F_q[X]$, hence if
supplied it with a shorter input an \emph{ad hoc} algorithm could
reach the bound $\Omega(p^kd)$.

In this section we give an algorithm that reaches this bound up to
some logarithmic factors. It realizes the polynomial interpolation on
the primitive points of $E[p^k]$, thus its output is a degree
$\euler(p^k)/2-1$ polynomial in $\F_q[X]$. Using the Chinese reminder
theorem, it is straightforward to generalise this to an algorithm
having the same asymptotic complexity realizing the polynomial
interpolation on all the points of $E[p^k]$. We call C2-AS-FI the
variant of C2-AS resulting from applying this new algorithm.

We set some notation. Let $i_0$ be the largest index such that
$\U_{i_0} = \U_1$ and let $\frac{p-1}{2r} = [\F_q[c^2]:\F_q]$. For
notational convenience, we set $\U_0=\F_q$.

We note $T(X)$ the polynomial vanishing on the primitive points of
$E[p^k]$ and

\begin{equation}
  \label{eq:T}
  T = \prod T_j^{(i)}
\end{equation}
its factorisation over $\U_i$; we remark that all the $T_j^{(0)}$'s
have degree $\frac{\euler(p^{k-i_0+1})}{2r}$. We also note $A(X)$ the goal
polynomial and
\begin{equation}
  \label{eq:A}
  A_j^{(i)} = A \bmod T_j^{(i)}
  \;\text{.}
\end{equation}

It was already pointed out in \cite[$\S$2.3]{Cou96} that if all the
$A_j^{(0)}$'s are known one can recover $A$ using the Chinese reminder
theorem. If we chose any point $P$ such that
$T_j^{(0)}\bigl(x(P)\bigr)=0$ and fix the embedding
\begin{equation}
  \label{eq:embed}
  \xymatrix{
    ^{\F_q[X]}/_{T_j^{(0)}(X)} \ar@{^{(}->}[r]^-\iota & \U_k
  }
\end{equation}
given by $\iota(X) = x(P)$, then it is evident that
$\iota\bigl(A_j^{(0)}(X)\bigr) = x\bigl(\I(P)\bigr)$, thus in order to
compute $A_j^{(0)}$ one just needs to compute
$\iota^{-1}\bigl(x(\I(P))\bigr)$.

Unfortunately, the information needed to compute $\iota$ was lost in
the $p$-descent, for we don't even know the $T_j^{(i)}$'s. None of the
algorithms of \cite{DFS09} helps us to compute such information and
straightforward computation of it would be to expensive. The solution
is to decompose $\iota$ as a chain of morphisms and invert them
one-by-one going down in the tower $(\U_0,\U_1,\ldots,\U_k)$, this is
similar to the way \cite{Cou00} solves an Artin-Schreier equation by
moving it down from $\U_k$ to $\U_1$.

\paragraph{The moduli}
We first need to compute $T_0^{(i)}\in\U_i[X]$ for any $i$. For this
we fix a primitive point $P\in E[p^k]$ and we reorder the indices in
\eqref{eq:T} so that $T_0^{(i)}$ is the minimal polynomial of $x(P)$
over $\U_i$.

The first minimal polynomial is simply
\begin{equation}
  \label{eq:T_0^k}
  T_0^{(k)}(X) = X - x(P)
  \;\text{.}
\end{equation}
Now suppose we know $T_0^{(i+1)}$, then a generator $\sigma$ of
$\Gal(\U_{i+1}/\U_i)$ acts on the roots of $T_0^{(i+1)}$ sending them
on the roots of some $T_j^{(i+1)}$. Then the
minimal polynomial of $x(P)$ over $\U_i$ is
\begin{equation}
  \label{eq:T_0^i}
  T_0^{(i)} = \prod_{\sigma\in\Gal(\U_{i+1}/\U_i)} \sigma\left(T_0^{(i+1)}\right)
  \;\text{.}
\end{equation}
Some care has to be taken when computing $T_0^{(0)}$: in fact the
abscissae of the points may be counted twice if
$c\not\in\F_q[c^2]$. In this case only a subgroup of index $2$ of
$\Gal(\U_1/\U_0)$ must be used instead of the whole group.


\paragraph{The interpolation}
The computation of $A_0^{(i)}$ is done in the same recursive way. Fix
the same point $P$ used to compute the $T_0^{(i)}$'s and fix the chain
of embeddings
\begin{equation}
  \xymatrix{
    ^{\U_0[X_0]}/_{T_0^{(0)}(X_0)} \ar@{^{(}->}[r]^-{\iota_0} &
    \;\cdots\; \ar@{^{(}->}[r]^-{\iota_{k-1}} &
    ^{\U_k[X_k]}/_{T_0^{(k)}(X_k)} \ar@{^{(}->}[r]^-{\iota_k} &
    \U_k
  }
\end{equation}
given by $\iota_k\circ\cdots\circ\iota_i(X_i) = x(P)$ for any $i$.

We compute $A_0^{(i)}$ by inverting the chain: inverting $\iota_k$
simply gives
\begin{equation}
  \label{eq:A_0^k}
  A_0^{(k)} = x\bigl(\I(P)\bigr)
  \;\text{.}
\end{equation}
Then suppose we know $A_0^{(i+1)}$ and decompose the embedding
$\iota_i$ as
\begin{equation}
  \xymatrix{
    ^{\U_i[X_i]}/_{T_0^{(i)}(X_i)} \ar@{^{(}->}[r]^-{\iota_i} \ar@{^{(}->}[d]^{\varepsilon} &
    ^{\U_{i+1}[X_{i+1}]}/_{T_0^{(i+1)}(X_{i+1})} \\
    ^{\U_{i+1}[Y]}/_{T_0^{(i)}(Y)} \ar@{^{(}->>}[r]^-{\gamma} &
    \bigoplus_j {}^{\U_{i+1}[Y_{j}]}/_{T_j^{(i+1)}(Y_{j})} \ar@{->>}[u]_{\pi}
  }
\end{equation}
where $\varepsilon$ is the canonical injection extending
$\U_i\subset\U_{i+1}$, $\gamma$ is the Chinese reminder isomorphism
and $\pi$ is projection onto the first coordinate.

To invert $\pi$ observe that any $\sigma\in\Gal(\U_{i+1}/\U_i)$ leaves
$A_0^{(i)}$ invariant while it permutes the moduli $T_j^{(i+1)}$, thus
\begin{equation}
  A_0^{(i)} \equiv \sigma\left(A_0^{(i+1)}\right)
  \bmod \sigma\left(T_0^{(i+1)}\right)
  \;\text{;}
\end{equation}
Hence we can obtain all the $A_j^{(i+1)}$ through the action of
$\Gal(\U_{i+1}/\U_i)$ on $A_0^{(i+1)}$.

Then we can invert $\gamma$ through a Chinese reminder algorithm
\cite[$\S$10.3]{vzGG} and $\varepsilon$ by converting coefficients from
$\U_{i+1}$ to $\U_i$.

As for the moduli, a special treatment is needed for $\iota_0$ if
$c\not\in\F_q[c^2]$.


\subsection{Complexity analysis}
\label{sec:C2-AS-FI:complexity}

The two algorithms for computing the $T_{0}^{(i)}$'s and the
$A_{0}^{(i)}$'s are very similar and run in parallel. We can merge
them in one unique algorithm, at each level $i\ge i_0$ it does the
following

\begin{enumerate}
\item for $\sigma \in \Gal(\U_{i+1}/\U_i)$,
  \begin{enumerate}
  \item\label{alg:T:gal} $T_{\bar\sigma(0)}^{(i+1)} =
    \sigma\left(T_0^{(i+1)}\right)$ and,
  \item\label{alg:A:gal} $A_{\bar\sigma(j)}^{(i+1)} =
    \sigma\left(A_0^{(i+1)}\right)$ using
    \cite[\alg{IterFrobenius}]{DFS09},
  \end{enumerate}
\item\label{alg:T:prod} compute $T_0^{(i)}$ through a subproduct tree
  as in \cite[Algo. 10.3]{vzGG},
\item\label{alg:A:CRA} compute $A_0^{(i)}$ through Chinese Reminder
  Algorithm \cite[Algo. 10.16]{vzGG},
\item\label{alg:T:push} convert $T_0^{(i)}$ and $A_0^{(i)}$ into
  elements of $\U_i[X]$ using \cite[\alg{Push-down}]{DFS09}.
\end{enumerate}

Steps \ref{alg:T:gal} and \ref{alg:A:gal} are identical. Both are
repeated $p$ times, each iteration taking $O\bigl(p^{k-i}{\sf
  L}(i-i_0)\bigr) \subset O\bigl({\sf L}(k-i_0)\bigr)$ by an easy
refinement of \cite[Th. 11]{DFS09} \todo one day we'll publish the
long version \todo

Step \ref{alg:T:prod} takes $O\bigl(\Mult(p^{k-i_0+1}d/r)\log p\bigr)$
by \cite[Lemma 10.4]{vzGG} and step \ref{alg:A:CRA} has the same
complexity by \cite[Coro. 10.17]{vzGG}.

Step \ref{alg:T:push} takes $O\bigl(p^{k-i+1}{\sf L}(i-i_0)\bigr)
\subset O\bigl(p{\sf L}(k-i_0)\bigr)$.

When $i=0$ and $\U_1\ne\F_q$ the algorithm is identical but steps
\ref{alg:T:gal} and \ref{alg:A:gal} must be computed through a generic
frobenius algorithm (applying a $q$-power, for example) and step
\ref{alg:T:push} must use the implementation of $F_q[c]$ to make the
conversion (for example, linear algebra). In this case steps
\ref{alg:T:gal} and \ref{alg:A:gal} cost
$\Theta\bigl(\frac{p^{k-i_0}}{r}\Mult(pd)\log q \bigr)$ and step
\ref{alg:T:push} costs $\Theta\bigl(p^{k-i_0}(pd)^2\bigr)$.

The total cost of the algorithm is then
\begin{equation*}
  \label{eq:T:complexity}
  O\left(\bigl(k-i_0\bigr)\bigl(p{\sf L}(k-i_0) + \Mult(p^{k-i_0+1}d/r)\log p\bigr) +
    \frac{p^{k-i_0}}{r}\bigl(\Mult(pd)\log q + r(pd)^2\bigr) \right)
  \;\text{.}
\end{equation*}

After all, the whole algorithm looks a lot like fast interpolation
\cite[$\S$10]{vzGG} and it is indeed a modified version of it. A
similar algorithm was already given in \cite{EnMo03}.


\paragraph{The complete interpolation}
We compute all the $A_j^{(0)}$'s using this algorithm; there's
$p^{i_0-1}r$ of them. We then recombine them through a Chinese
reminder algorithm at a cost of $O\bigl(\Mult(p^kd)\log
p^{i_0-1}r\bigr)$. The total cost of the whole interpolation phase is
then
\begin{equation*}
  O\left(\bigl(k-i_0\bigr) \bigl(p{\sf L}(k) + \Mult(p^kd)\log p\bigr) +
    p^{k-1}\Mult(pd)\log q + p^k(pd)^2 + i_0\Mult(p^kd)\log p
  \right)
  \;\text{,}
\end{equation*}
that is
\begin{equation}
  \label{eq:interp}
  O\left(p{\sf L}(k)\log\left(\frac{\ell}{p^{i_0}}\right) + 
    \Mult(\ell d)\log\ell\log p +
    \frac{\ell d}{p}\Mult(pd)\log p +
    \ell(pd)^2
  \right)
  \;\text{.}
\end{equation}

Alternatively, once $A_0^{(0)}$ is known, one could compute the other
$A_j^{(0)}$'s using modular composition with the multiplication maps
of $E$ and $E'$ as suggested in \cite{Cou96}. However this approach
doesn't give a better asymptotic complexity because in the worst case
$A_0^{(0)}=A$. From a practical point of view, though, Brent's and
Kung's algorithm for modular composition \cite{BrKu78}, despite having
a worse asymptotic complexity, could perform faster for some set of
parameters. We will discuss this matter in Section
\ref{sec:implementation}.

If more than $\euler(p^k)/2$ points are needed, but less than
$\frac{p-1}{2}$, one can use the previous algorithm to compute all the
polynomials $A_i$ interpolating respectively over the $p^i$-torsion
points of $E$ and $E'$. They can then be recombined through a Chinese
reminder algorithm at a cost of $O\bigl(\Mult(p^kd)\log p^k\bigr)$,
which doesn't change the overall complexity of C2-AS-FI.


Putting together the complexity estimates of C2-AS and C2-AS-FI, we
have the following.

\begin{theorem}
  \label{th:complexity}
  Assuming $\Mult(n) = n\log n\log\log n$, the algorithm C2-AS-FI has
  worst case complexity
  \begin{equation*}
    \tildO_{p,d,\log\ell}\left(
      p^2d^3 +
      p^3d^2 +
      (pd)^\omega\log\ell +
      p^3\ell^2 d\log^3\ell + 
      p^2\ell^2 d^2
    \right)
    \;\text{,}
  \end{equation*}
  or
  \begin{equation*}
    \tildO\left(
      p^2d^3 +
      p^3d^2 +
      (pd)^\omega +
      p^3\ell^2 d + 
      p^2\ell^2 d^2
    \right)
    \;\text{.}
  \end{equation*}
\end{theorem}



% Local Variables:
% mode:flyspell
% ispell-local-dictionary:"british"
% mode:TeX-PDF
% TeX-master: "ec-isogeny"
% End:
%
% LocalWords:  Schreier Artin pseudotrace frobenius bivariate Joux Sirvent FFT
% LocalWords:  Couveignes isogenies Schoof isogeny cryptosystems Lercier moduli
% LocalWords:  precomputation arithmetics polylogarithmic Karatsuba embeddings
% LocalWords:  irreducibility

\section{Smallest degree isogeny}
\label{sec:bounded}

We now present an extension to Couveignes algorithm that could be
useful in cryptographic application. It is well known that two curves
having the same number of points over a finite field are isogenous,
however this doesn't say anything on the degree of the isogeny
connecting them. Given two elliptic curves $E$ and $E'$ defined over
$\F_q$ and having the same number of points, we want to find the
smallest degree isogeny between them.

The simplest solution is to take any algorithm computing a fixed
degree isogeny and try all the degrees until an isogeny is found. If
$\ell$ is the degree of the smallest isogeny, this of course adds a
factor $\ell$ to the complexity of any polynomial time algorithm.

Couveignes' algorithm can be easily adapted to solve this problem at
no additional cost. We call this algorithm C2SD and we will only
discuss its most efficient variant C2SD-AS-FI.

Observe that, apart for the choice of $k$, the computation of $E[p^k]$
and the polynomial interpolation step do not depend at all on
$\ell$. The degree of the isogeny only comes into play in the last
part of the Cauchy interpolation, that is the rational function
reconstruction. We study more in detail this last step.


\paragraph{Rational Function Reconstruction}
Rational function reconstruction takes as input a degree $n$
polynomial $T$, a polynomial $A$ of degree less than $n$ and a target
degree $m\le n$ and outputs the unique rational function such that
\begin{equation*}
  A \equiv \frac{R}{V} \bmod T
\end{equation*}
and $\deg R < m$, $\deg V \le n-m$. This is done by computing a Bezout
relation $AV + TU = R$ with the expected degrees via an XGCD
algorithm. If a classical XGCD algorithm is used, one simply computes
all the lines
\begin{equation}
  \label{eq:XGCD}
  \begin{aligned}
    R_0 &= T, & U_0 &= 1, & V_0 &= 0,\\
    R_1 &= A, & U_1 &= 0, & V_1 &= 1,\\
    R_{i-1} &= Q_iR_i + R_{i+1}, & U_{i+1} &= U_{i-1}-Q_iU_i, & V_{i+1} &= V_{i-1}-Q_iV_i
  \end{aligned}
\end{equation}
and stops as soon as a remainder $R_{i+1}$ with $\deg R_{i+1}<m$ is
found. If a fast XGCD algorithm as \cite[Algo. 11.4]{vzGG} is used,
one directly aims at the two lines
\begin{equation}
  \label{eq:FastGCD}
  \begin{aligned}
    R_{h-2} &= Q_{h-1}R_{j-1} + R_h\\
    R_{h-1} &= Q_hR_h + R_{h+1}
  \end{aligned}
\end{equation}
such that $\deg R_{h+1} < m \le \deg R_h$ without computing the
intermediate lines.

When looking for an $\ell$-isogeny, one simply sets
$m=\ell+1$. Observe that if the algorithm doesn't return a rational
fraction $\frac{R}{V}$ with $\deg R = \ell$ and $\deg V = \ell -1 $,
then no such fraction congruent to $A$ modulo $T$ exists.

If $\ell$ is not \emph{a priori} known, we can still use the fact that
a separable isogeny with cyclic kernel must have $\deg R = \deg V +
1$. In fact if we suppose $R = R_i$ and $V = V_i$, then
\begin{align*}
  \deg T &= \deg V_{i+1} + \deg R_i,\\
  \deg R_i - \deg V_i &= \deg R_{i-1} - \deg V_{i+1}
\end{align*}
implies
\begin{equation*}
  \deg T + 1 = \deg R_{i-1} + \deg R_i  
  \;\text{.}
\end{equation*}
Hence, if $A$ is congruent to an $\ell$-isogeny with $\ell =
\left\lfloor\frac{\deg T}{2}\right\rfloor - t$ for some $t\ge0$, then
\begin{equation}
  \label{eq:degseq}
  \deg R_{i-1} =
  \left\lceil\frac{\deg T}{2}\right\rceil + t + 1 >
  \left\lfloor\frac{\deg T}{2}\right\rfloor - t = \deg R_i
  \;\text{.}
\end{equation}
Thus we can recover any isogeny having degree less than
$\left\lfloor\frac{\deg T}{2}\right\rfloor$ using either a classical
or a fast XGCD algorithm setting $m = \left\lceil\frac{\deg
    T}{2}\right\rceil + 1$.


\paragraph{Recognising an isogeny}
Once we have a rational fraction with the required degree, we have to
test if it really is an isogeny. In order to understand how often we
have to make this test, we introduce some more terminology. Let $n_i =
\deg R_i$, we call $(n_0,\ldots,n_r)$ the \emph{degree sequence} of
$A$ and $T$; a degree sequence is said \emph{normal} if $n_i = n_{i+1}
+ 1$ for any $i$.

\begin{proposition}
  \label{th:normseq}
  Let $f,g\in\F_q[X]$ be uniformly chosen random polynomials of
  respective degrees $n_0>n_1>0$ and let $(n_0, n_1, \ldots, n_r)$ be
  their degree sequence. For $0\le i < n_1$ define the binary random
  variables $X_i = 1 \Leftrightarrow i\in(n_0,n_1,\ldots,n_r)$, then
  the $X_i$ are independent random variables and $\mathrm{Prob}(X_i=0) =
  \frac{1}{q}$.
\end{proposition}
\begin{proof}
  Pairs of polynomials $f,g$ are in bijection with the GCD-sequence
  $(R_r, Q_r, \ldots, Q_1)$ constituted by their GCD and the quotients
  of the GCD algorithm. To each such sequence is associated a degree
  sequence
  \begin{equation*}
    (n_0,n_1,\ldots,n_r) =
    \left(\deg R_r + \sum_{i=1}^r\deg Q_i, \ldots, \deg R_r + \sum_{i=1}^1\deg Q_i, \deg R_r\right)
    \;\text{,} 
  \end{equation*}
  thus for any given degree sequence there are
  \begin{equation*}
    (q-1)q^{n_0-n_1}\cdot(q-1)q^{n_1-n_2}\cdot\cdots\cdot(q-1)q^{n_r} =
    (q-1)^{r+1}q^{n_0}
  \end{equation*}
  GCD-sequences.
 
  Let $I$ and $O$ be two disjoints subsets of $\{X_i\}$, the number of
  GCD-sequences such that $X\in I \Rightarrow X=1$ and $X\in O
  \Rightarrow X=0$,
  \begin{equation*}
     \sum_{s=0}^{n_1-\card{I}-\card{O}}\binom{n_1-\card{I}-\card{O}}{s}(q-1)^{s+2+\card{I}}q^{n_0} =
    (q-1)^{2+\card{I}}q^{n_0}q^{n_1-\card{I}-\card{O}}
    \;\text{.}
  \end{equation*}
  There are $(q-1)^2q^{n_0}q^{n_1}$ pairs of polynomials of degrees
  $n_0,n_1$, thus
  \begin{equation}
   \label{th:normseq:prob}
    \mathrm{Prob}\bigl(\{X = 1 \mid X\in I\},
    \{X=0\mid X\in O\}\bigr) = \left(\frac{q-1}{q}\right)^{\card{I}}\left(\frac{1}{q}\right)^{\card{O}}
    \;\text{.}
  \end{equation}
  The claim follows.
\end{proof}

Degree sequences associated to isogenies are in general not normal, in
fact if $\ell\le\left\lfloor\frac{\deg T}{2}\right\rfloor-t$, equation
\eqref{eq:degseq} shows that there must be at least a gap of degree
$2c$ in the degree sequence. Heuristically, we can expect that if the
polynomial $A$ doesn't correspond to an isogeny, then $A$ and $T$ act
like random polynomials, thus, by the proposition above, the
probability that $A$ looks like an isogeny of degree
$\ell\le\left\lfloor\frac{\deg T}{2}\right\rfloor-t$ is less than $\frac{1}{q^{2t}}$.

Therefore, by choosing an appropriate $t\in O(\log_q p^k)$, C2SD can
find any isogeny of degree less than $\frac{p^k-1}{4}-t$ at the same
cost of one run of C2. Also notice that C2SD is not restricted to
isogenies of degree prime to $p$ as it was already mentioned in
Section \ref{sec:C2:non-prime}.

No other method for computing isogenies is known to have a similar
generalisation, this makes C2SD-AS-FI interesting for practical
applications.




% Local Variables:
% mode:flyspell
% ispell-local-dictionary:"british"
% mode:TeX-PDF
% TeX-master: "ec-isogeny"
% End:
%
% LocalWords:  Schreier Artin pseudotrace Frobenius bivariate Joux Sirvent FFT
% LocalWords:  Couveignes isogenies Schoof isogeny cryptosystems Lercier
% LocalWords:  precomputation arithmetics polylogarithmic Karatsuba precomputes
% LocalWords:  endomorphisms  isogenous

\section{Implementation}
\label{sec:implementation}

We implemented C2-AS-FI-MC as \texttt{C++} programs using the
libraries \texttt{NTL}~\cite{NTL} for finite field arithmetics,
\texttt{gf2x}~\cite{gf2x} for fast arithmetics in characteristic $2$
and \texttt{FAAST}~\cite{DFS09} for fast arithmetics in Artin-Schreier
towers.

This section mainly deals with some tricks we implemented in order to
speed up the computation. At the end of the section we briefly discuss
the implementation we made in Magma~\cite{Magma} of the algorithm
in~\cite{LeSi09}.

\subsection{Building $E[p^k]$ and $E'[p^k]$}
\label{sec:impl:torsion}

\paragraph{$p$-torsion}
For $p\ne2$, C2 and its variants require to build the extension
$\F_q[c]$ where $c$ is a $p-1$-th root of $H_E$. In order to deal with
the lowest possible extension degree, it is a good idea to modify the
curve so that $[\F_q[c]:\F_q]$ is the smallest possible.

$[\F_q[c]:\F_q]$ is invariant under isomorphism, but taking a twist
can save us a quadratic extension. Let $u=c^{-2}$, the curve
\begin{equation*}
  \bar{E} : y^2 = x^3 + a_2ux^2 + a_4u^2x + a_6u^3
\end{equation*}
is defined over $\F_q[c^2]$ and is isomorphic to $E$ over $\F_q[c]$
via $(x,y)\mapsto(\sqrt{u}^2x,\sqrt{u}^3y)$. Its Hasse invariant is
$H_{\bar{E}} = (u)^{\frac{p-1}{2}}H_E = 1$, thus its $p$-torsion
points are defined over $\F_q[c^2]$.

In order to compute the $p^k$-torsion points of $E$ we build
$\F_q[c^2]$, we compute $\bar{P}$ a $p^k$-torsion points of $\bar{E}$
using $p$-descent, then we invert the isomorphism to compute the
abscissa of $P\in E[p^k]$. Since the Cauchy interpolation only needs
the abscissae of $E[p^k]$, this is enough to complete the
algorithm. Scalar multiples of $P$ can be computed without knowledge
of $y(P)$ using Montgomery formulae \cite{Mon87}.

Remark that for $p=2$ we use the same construction in an implicit way
since we do a $p$-descent on the Kummer surface.


\paragraph{$p^k$-torsion points}
For $p\ne2$ we use Voloch's $p$-descent to compute the $p^k$-torsion
points iteratively as described in Section \ref{sec:C2}. To factor the
Artin-Schreier polynomial \eqref{th:voloch:cover}, we use the
algorithms from \cite{Cou00} and \cite{DFS09} that were analysed in
Section \ref{sec:C2-AS}. All these algorithms were provided by the
library \texttt{FAAST}.

To solve system \eqref{th:voloch:isom} we first compute
\begin{equation*}
  V(x,y) = \left(\frac{g(x)}{h^2(x)}, 
    sy\left(\frac{g(x)}{h^2(x)}\right)'\right)
\end{equation*}
through Vélu formulae.\footnote{Vélu formulae compute this isogeny up
  to an indeterminacy on the sign of the ordinate, the actual value of
  $s$ must be determined by composing $V$ with $\frobisog$ and
  verifying that it corresponds to $[p]$ by trying some random
  points.} Recall that we work on a curve having Hasse invariant $1$,
system \eqref{th:voloch:isom} can then be rewritten
\begin{equation*}
  \left\{
    \begin{aligned}
      X &= \frac{g(x)}{h^2(x)}\\
      Y &= sy\left(\frac{g(x)}{h^2(x)}\right)'\\
      Z &= -2y\frac{h'(x)}{h(x)}
    \end{aligned}
  \right.
\end{equation*}
where $(X,Y,Z)$ is the point on the cover $C$ that we want to pull
back. After some substitutions this is equivalent to
\begin{equation*}
  \left\{
    \begin{aligned}
      Xh^2(x) - g(x) &= 0\\
      \left(Xh^2(x) - g(x) - \frac{Y}{sZ}h^2(x)\right)' &= 0
    \end{aligned}
  \right.
\end{equation*}
Then a solution to this system is given by the GCD of the two
equations. Remark that proposition \ref{th:voloch} ensures there is
one unique solution. This formulae are slightly more efficient than
the ones in \cite[$\S$6.2]{Ler97}.

For $p=2$ we use the library \texttt{FAAST} (for solving
Artin-Schreier equations) on top of \texttt{gf2x} (for better
performance). There is nothing special to remark about the
$2$-descent.


\subsection{Cauchy interpolation and loop}
\label{sec:impl:cauchy}
The polynomial interpolation step is done as described in Section
\ref{sec:C2-AS-FI}. As a result of this implementation, the polynomial
interpolation algorithm was added to the library \texttt{FAAST}.

The rational fraction reconstruction is implemented using a fast XGCD
algorithm on top of \texttt{NTL} and \texttt{gf2x}. This algorithm was
added to \texttt{FAAST} too.

The loop uses modular composition as in Section~\ref{sec:C2-AS-FI-MC}
in order to minimise the number of interpolations. The timings in the
next section clearly show that this non-asymptotically-optimal variant
performs much faster in practice.

To check that the rational fractions are isogenies we test their
degrees, that their denominator is a square and that they act as group
morphisms on a fixed number of random points. All these checks take a
negligible amount of time compared to the rest of the algorithm.


\subsection{Parallelisation of the loop}
\label{parallel}

The most expensive step of C2-AS-FI-MC, in theory as well as in
practice, is the final loop over the points of $E'[p^k]$. Fortunately,
this phase is very easy to parallelise with very few overhead.

Let $n$ be the number of processors we wish to parallelise on, suppose
that $[\U_k:\F_q]$ is maximal, then we make only one interpolation
followed by $\euler(p^k)/2$ modular compositions.\footnote{If
  $[\U_k:\F_q]$ is not maximal, the parallelisation is straightforward
  as we simply send one interpolation to each processor in turn.} We
set $m=\left\lfloor\frac{\euler(p^k)}{2n}\right\rfloor$ and we compute
the action of $\frob^{m}$ on $E[p^k]$ as in
Section~\ref{sec:C2-AS-FI-MC}:
\begin{equation*}
  F^{(m)}(X) = F(X) \circ \cdots \circ F(X) \bmod T(X)
  \;\text{,}
\end{equation*}
this can be done with $\Theta(\log m)$ modular compositions via a
binary square-and-multiply approach as in~\cite[Algorithm
5.2]{vzGS92}.

Then we compute the $n$ polynomials
\begin{equation*}
  A_{mi}(X) = A_{m(i-1)}(X) \circ F^{(m)}(X) \bmod T(X)
\end{equation*}
and distribute them to the $n$ processors so that they each work on a
separate slice of the $A_i$'s. The only overhead is $\Theta(\log
(\ell/n))$ modular compositions with coefficients in $\F_q$, this is
acceptable in most cases.


\subsection{Implementation of~\cite{LeSi09}}
In order to compare our implementation with the state-of-the-art
algorithms, we implemented a Magma prototype of~\cite{LeSi09}; in what
follows, we will refer to this algorithm as LS. The algorithm
generalises~\cite{BoMoSaSc08} by lifting the curves in the $p$-adics
to avoid divisions by zero. Given two curves $E$ and $E'$ and an
integer $\ell$, it performs the following steps
\begin{enumerate}
\item Lift $E$ to $\bar{E}$ in $\Q_q$,
\item Lift the modular polynomial $\Phi_\ell$ to $\bar{\Phi}_\ell$ in
  $\Q_q$,
\item Find a root in $\Q_q$ of $\bar{\Phi}(X,j_{\bar{E}})$ that
  reduces to $j_{E'}$ in $\F_q$,
\item Apply~\cite{BoMoSaSc08} in $\Q_q$ to find an isogeny between
  $\bar{E}$ and $\bar{E}'$,
\item Reduce the isogeny to $\F_q$.
\end{enumerate}

We implemented this algorithm using Magma support for the
$p$-adics. Instead of the classical modular polynomials $\Phi_\ell$ we
used Atkin's canonical polynomials $\Phi^\ast_\ell$ since they have
smaller coefficients and degree; this does not change the other steps
of the algorithm. The modular polynomials were taken from the tables
precomputed in Magma.

The bottleneck of the algorithm is the use of the modular polynomial
as its bit size is $O(\ell^3)$, thus LS is asymptotically worse in
$\ell$ than C2. However the next section will show that LS is more
practical than C2 in many circumstances.



% Local Variables:
% mode:flyspell
% ispell-local-dictionary:"british"
% mode:TeX-PDF
% TeX-master: "ec-isogeny"
% End:
%
% LocalWords:  Schreier Artin pseudotrace Frobenius bivariate Joux Sirvent FFT
% LocalWords:  Couveignes isogenies Schoof isogeny cryptosystems Lercier Hasse
% LocalWords:  precomputation arithmetics polylogarithmic Karatsuba precomputes
% LocalWords:  endomorphisms 

\section{Benchmarks}
\label{sec:benchmarks}



% Local Variables:
% mode:flyspell
% ispell-local-dictionary:"british"
% mode:TeX-PDF
% TeX-master: "ec-isogeny"
% End:
%
% LocalWords:  Schreier Artin pseudotrace frobenius bivariate Joux Sirvent FFT
% LocalWords:  Couveignes isogenies Schoof isogeny cryptosystems Lercier
% LocalWords:  precomputation arithmetics polylogarithmic Karatsuba precomputes
% LocalWords:  endomorphisms 


\scriptsize

\begin{thebibliography}{10}

\bibitem{Atk91}
  A.O.L.~Atkin.
  \newblock \emph{The number of points on an elliptic curve modulo a prime}.
  \newblock Email on the Number Theory Mailing List, 1991.

\bibitem{BK78}R.P.~Brent and H.T.~Kung, Fast algorithms for
  manipulating formal power series.  \emph{J. ACM} 25(4):581--595, 1978.

\bibitem{BlSeSm99}
  I.~Blake, G.~Seroussi, and N.~Smart.
  \newblock {\em Elliptic curves in cryptography}.
  \newblock Cambridge University Press, 1999.

\bibitem{BoMoSaSc08}
  A.~Bostan, F.~Morain, B.~Salvy and É.~Schost
  \newblock Fast algorithms for computing isogenies between elliptic curves.
  \newblock \emph{Math. Comp.} 77, 263:1755-1778, 2008.

\bibitem{Magma}
  W.~Bosma, J.~Cannon, C.~Playoust.
  \newblock The Magma algebra system. I. The user language.
  \newblock \emph{J. Symb. Comp.}, 24(3-4):235-265, 1997.
  
\bibitem{gf2x}
  R.~Brent, P.~Gaudry, E.~Thom{\'e}, P.~Zimmermann.
  \newblock Faster multi\-plication in GF$(2)[x]$.
  \newblock In {\em ANTS'08}, 153-166. Springer, 2008.

\bibitem{Cou94}
  J.-M.~Couveignes.
  \newblock \emph{Quelques calculs en théorie des nombres}.
  \newblock PhD thesis. 1994.

\bibitem{Cou96}
  J.-M.~Couveignes. 
  \newblock Computing {$\ell$}-isogenies using the {$p$}-torsion.
  \newblock in {\em ANTS'II}, 59--65. Springer, 1996.

\bibitem{Cou00}
  J.-M.~Couveignes. 
  \newblock Isomorphisms between {A}rtin-{S}chreier towers. 
  \newblock \emph{Math. Comp.} 69(232): 1625--1631, 2000.
  
\bibitem{DFS09}
  L.~De~Feo and É.~Schost
  \newblock Fast Arithmetics in Artin Schreier Towers.
  \newblock \emph{ISSAC '09}, 2009.

\bibitem{Elk91}
  N.D.~Elikes
  \newblock \emph{Explicit Isogenies}.
  Draft, 1991.

\bibitem{EnMo03}
  A.~Enge and F.~Morain,
  \newblock Fast decomposition of polynomials with known Galois group.
  \newblock in {\em AAECC-15}, 254--264. Springer, 2003.

\bibitem{JL06}A.~Joux, R. Lercier.
  \newblock Counting points on elliptic curves in medium characteristic.
  \newblock \emph{Cryptology ePrint Archive} 2006/176, 2006.

\bibitem{vzGG} 
  J.~von~zur~Gathen and J.~Gerhard. 
  \newblock \emph{Modern Computer Algebra}. 
  \newblock Cambridge University Press, 1999.

\bibitem{Gun76}H.~Gunji,
  \newblock The Hasse Invariant and $p$-division Points of an Elliptic Curve.
  \newblock \emph{Archiv der Mathematik} 27(2), Springer, 1976.

\bibitem{Ler96}
  R.~Lercier.
  \newblock Computing isogenies in GF($2^n$).
  \newblock In {\em ANTS-II}, LNCS vol 1122, pages 197--212. Springer, 1996.

\bibitem{Ler97}
  R.~Lercier.
  \newblock Algorithmique des courbes elliptiques dans les corps finis. 
  \newblock Ph.D. Thesis, {\'E}cole polytechnique, 1997.

\bibitem{LeSi09}
  R.~Lercier, T.~Sirvent.
  \newblock On Elkies subgroups of $\ell$-torsion points in curves defined over
  a finite field.
  \newblock To appear in {\em J. Th\'eor. Nombres Bordeaux}.

\bibitem{RoSt06}
  A.~Rostovtsev and A.~Stolbunov.
  \newblock Public-key cryptosystem based on isogenies.
  \newblock Cryptology ePrint Archive, Report 2006/145.
  
\bibitem{Sat00}
  T.~Satoh.
  \newblock The canonical lift of an ordinary elliptic curve over a nite eld and its point counting.
  \newblock \emph{Journal of the Ramanujan Mathematical Society}, 2000.

\bibitem{Sch95}
  R.~Schoof.
  \newblock Counting points on elliptic curves over finite fields.
  \newblock \emph{J. Théorie des Nombres de Bordeaux} 7:219--254, 1995.

\bibitem{NTL}
  V.~Shoup.
  \newblock {NTL}: A library for doing number theory.
  \newblock \url{http://www.shoup.net/ntl/}.

\bibitem{Tes06}
  E.~Teske.
  \newblock An elliptic trapdoor system.
  \newblock {\em Journal of Cryptology}, 19(1):115--133, 2006.

\bibitem{Vel71}J.~Vélu,
  \newblock Isogénies entre courbes elliptiques.
  \newblock \emph{Comptes Rendus de l'Académie des Sciences de Paris} 273,
  Série A, 238--241, 1971.

  
  
\end{thebibliography}


\end{document}


% Local Variables:
% mode:flyspell
% ispell-local-dictionary:"british"
% mode:TeX-PDF
% End:
%
% LocalWords:  Schreier Artin pseudotrace frobenius bivariate memoization
% LocalWords:  Couveignes isogenies
