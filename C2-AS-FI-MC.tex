\section{The algorithm C2-AS-FI-MC}
\label{sec:C2-AS-FI-MC}

However asymptotically fast, the polynomial interpolation step is
quite expensive for reasonably sized data. Instead of repeating it
$\frac{\euler(p^k)}{2}$ times, one can use composition with the
Frobenius endomorphism $\frobisog_E$ in order to reduce the number of
interpolations in the final loop.

\subsection{The algorithm}
Suppose we have computed, by the algorithm of the previous Section,
the polynomial $T$ vanishing on the abscissae of $E[p^k]$ and an
interpolating polynomial $A_0\in\F_q[X]$ such that
\begin{equation*}
  A_0\bigl(x\bigl([n]P\bigr)\bigr) = x\bigl([n]P'\bigr)
  \quad\text{for any $n$.}
\end{equation*}
The group $\Gal(\U_k/\F_q) = \langle\frob\rangle$ acts on $E'[p^k]$
permuting its points and preserving the group structure. Thus, the
polynomial
\begin{equation*}
  A_1 = A_0\circ\frob = \frob\circ A_0
\end{equation*}
is an interpolating polynomial such that
\begin{equation*}
  A_1\bigl(x\bigl([n]P\bigr)\bigr) = x\bigl([n]\frobisog_{E'}(P')\bigr)
  \quad\text{for any $n$,}
\end{equation*}
where $\frobisog_{E'}$ is the Frobenius endomorphism of $E'$.  Since
$\frobisog_{E'}(P')$ is a generator of $E'[p^k]$, $A_1$ is one of the
polynomials that the algorithm C2 tries to identify to an isogeny. By
iterating this construction we obtain $[\U_k:\F_q]/2$ different
polynomials $A_i$ for the algorithm C2 with only one interpolation.

To compute the $A_i$'s, we first compute $F\in\F_q[X]$
\begin{equation}
  \label{eq:frob}
  F(X) = X^q \bmod T(X)
  \text{,}
\end{equation}
then for any $1\le i<[\U_k:\F_q]/2$
\begin{equation}
  \label{eq:modcomp}
  A_i(X) = A_{i-1}(X)\circ F(X) \bmod T(X)\text{.}
\end{equation}

If $\frac{\euler(p^k)}{[\U_k:\F_q]} = p^{i_0-1}r$, we must compute
$p^{i_0-1}r$ polynomial interpolations and apply this algorithm to
each of them in order to deduce all the polynomials needed by C2.


\subsection{Complexity analysis}
We compute \eqref{eq:frob} via square-and-multiply, this costs
$\Theta(d\Mult(p^kd)\log p)$ operations. Each application of
\eqref{eq:modcomp} is done via a \emph{modular composition}, the cost
is thus $O(\ModComp(p^k))$ operations in $\F_q$, that is
$O(\ModComp(p^k)\Mult(d))$ operations in $\F_p$. Using the algorithm
of~\cite{KeUm08} for modular composition, the complexity of
C2-AS-FI-MC wouldn't be essentially different from the one of
C2-AS-FI; however, in practice the fastest algorithm for modular
composition is~\cite{BrKu78}, and in particular the variant
in~\cite[Lemma 3]{KS98}, which has a worse asymptotic complexity, but
performs better on the instances we treat in
Section~\ref{sec:benchmarks}.

Notice that a similar approach could be used inside the polynomial
interpolation step (see Section \ref{sec:C2-AS-FI}) to deduce
$A_k^{(0)}$ from $A_0^{(0)}$ using modular composition with the
multiplication maps of $E$ and $E'$ as described in
\cite[$\S$2.3]{Cou96}. This variant, though, has an even worse
complexity because of the cost of computing multiplication maps.




% Local Variables:
% mode:flyspell
% ispell-local-dictionary:"british"
% mode:TeX-PDF
% TeX-master: "ec-isogeny"
% End:
%
% LocalWords:  Schreier Artin pseudotrace Frobenius bivariate Joux Sirvent FFT
% LocalWords:  Couveignes isogenies Schoof isogeny cryptosystems Lercier moduli
% LocalWords:  precomputation arithmetics polylogarithmic Karatsuba embeddings
% LocalWords:  irreducibility
