\section{Implementation}
\label{sec:implementation}

We implemented C2-AS-FI and C2SD-AS-FI as \texttt{C++} programs using
the libraries \texttt{NTL} for finite field arithmetics, \texttt{gf2x}
for fast arithmetics in characteristic $2$ and \texttt{FAAST} for fast
arithmetics in Artin-Schreier towers.

This section mainly deals with some tricks we implemented in order to
speed up the computation.

\subsection{Building $E[p^k]$ and $E'[p^k]$}
\label{sec:impl:torsion}

\paragraph{$p$-torsion}
For $p\ne2$, C2 and its variants require to build the extension
$\F_q[c]$ where $c$ is a $p-1$-th root of $H_E$. In order to deal with
the lowest possible extension degree, it is a good idea to modify the
curve so that $[\F_q[c]:\F_q]$ is the smallest possible.

$[\F_q[c]:\F_q]$ is invariant under isomorphism, but taking a twist
can save us a quadratic extension. Let $u=c^{-2}$, the curve
\begin{equation*}
  \bar{E} : y^2 = x^3 + a_2ux^2 + a_4u^2x + a_6u^3
\end{equation*}
is defined over $\F_q[c^2]$ and is isomorphic to $E$ over $\F_q[c]$
via $(x,y)\mapsto(\sqrt{u}^2x,\sqrt{u}^3y)$. Its Hasse invariant is
$H_{\bar{E}} = (u)^{\frac{p-1}{2}}H_E = 1$, thus its $p$-torsion
points are defined over $\F_q[c^2]$.

In order to compute the $p^k$-torsion points of $E$ we build
$\F_q[c^2]$, we compute $\bar{P}$ a $p^k$-torsion points of $\bar{E}$
using $p$-descent, then we invert the isomorphism to compute the
abscissa of $P\in E[p^k]$. Since the Cauchy interpolation only needs
the abscissae of $E[p^k]$, this is enough to complete the
algorithm. Scalar multiples of $P$ can be computed without knowledge
of $y(P)$ using Montgomery formulae \cite{Mon87}.

Remark that for $p=2$ we use the same construction in an implicit way
since we do a $p$-descent on the Kummer surface.


\paragraph{$p^k$-torsion points}
For $p\ne2$ we use Voloch's $p$-descent to compute the $p^k$-torsion
points iteratively as described in Section \ref{sec:C2}. To factor the
Artin-Schreier polynomial \eqref{th:voloch:cover}, we use the
algorithms from \cite{Cou00} and \cite{DFS09} that were analysed in
Section \ref{sec:C2-AS}. All these algorithms were provided by the
library \texttt{FAAST}.

To solve system \eqref{th:voloch:isom} we first compute
\begin{equation*}
  V(x,y) = \left(\frac{g(x)}{h^2(x)}, 
    sy\left(\frac{g(x)}{h^2(x)}\right)'\right)
\end{equation*}
through Vélu formulae.\footnote{Vélu formulae compute this isogeny up
  to an indeterminacy on the sign of the ordinate, the actual value of
  $s$ must be determined by composing $V$ with $\frobisog$ and
  verifying that it corresponds to $[p]$ by trying some random
  points.} Recall that we work on a curve having Hasse invariant $1$,
system \eqref{th:voloch:isom} can then be rewritten
\begin{equation*}
  \left\{
    \begin{aligned}
      X &= \frac{g(x)}{h^2(x)}\\
      Y &= sy\left(\frac{g(x)}{h^2(x)}\right)'\\
      Z &= -2y\frac{h'(x)}{h(x)}
    \end{aligned}
  \right.
\end{equation*}
where $(X,Y,Z)$ is the point on the cover $C$ that we want to pull
back. After some substitutions this is equivalent to
\begin{equation*}
  \left\{
    \begin{aligned}
      Xh^2(x) - g(x) &= 0\\
      \left(Xh^2(x) - g(x) - \frac{Y}{sZ}h^2(x)\right)' &= 0
    \end{aligned}
  \right.
\end{equation*}
Then a solution to this system given by the GCD of the two
equations. Remark that proposition \ref{th:voloch} ensures there is
one unique solution. This formulae are slightly more efficient than
the ones in \cite[$\S$6.2]{Ler97}.

For $p=2$ we use the library \texttt{FAAST} (for solving
Artin-Schreier equations) on top of \texttt{gf2x} (for better
performance). There is nothing special to remark about the
$2$-descent.


\subsection{Cauchy interpolation and loop}
\label{sec:impl:cauchy}
The polynomial interpolation step is done as described in Section
\ref{sec:C2-AS-FI}. As a result of this implementation, the polynomial
interpolation algorithm was added to the library \texttt{FAAST}.

The rational fraction reconstruction is implemented using a fast XGCD
algorithm on top of \texttt{NTL} and \texttt{gf2x}. This algorithm was
added to \texttt{FAAST} too.

To check that the rational fractions are isogenies we test their
degrees, that their denominator is a square and that they act as group
morphisms on a fixed number of random points. All these checks take a
negligible amount of time compared to the rest of the algorithm.

However asymptotically fast, the polynomial interpolation step is
quite expensive for reasonably sized data. Instead of repeating it
$\frac{\euler(p^k)}{2}$ times, we used composition with the frobenius
endomorphism $\frobisog_E$ in order to reduce the number of
interpolations in the final loop. This works as follows. Suppose we
have computed the polynomial $A_0\in\F_q[X]$ such that
\begin{equation*}
  A_0\bigl(x\bigl([n]P\bigr)\bigr) = x\bigl([n]P'\bigr)
  \quad\text{for any $n$,}
\end{equation*}
we view $A_0$ as the restriction to $K_E[p^k]$ of an $\F_q$-rational
algebraic map\footnote{Notice this is not necessarily a group
  morphism.}  $\mathcal{A}_0:E\rightarrow E'$.

Let $T$ be the polynomial vanishing on the abscissae of $E[p^k]$
computed by the algorithm in Section \ref{sec:C2-AS-FI}, compute
$F\in\F_q[X]$, the restriction of $\frobisog_E$ to $K_E[p^k]$, this is
\begin{equation}
  \label{eq:frob}
  F(X) = X^q \bmod T(X)
  \text{.}
\end{equation}

Define the algebraic maps
\begin{equation*}
  \mathcal{A}_i := \mathcal{A}_0\circ\frobisog_E^i = \frobisog_{E'}^i\circ\mathcal{A}_0
\end{equation*}
where the equality follows from the fact that $\mathcal{A}_0$ is
$\F_q$-rational. Then their restrictions to $K_E[p^k]$ are given by
\begin{equation}
  \label{eq:modcomp}
  A_i(X) = A_{i-1}(X)\circ F(X) \bmod T(X)
\end{equation}
and 
\begin{equation*}
  A_i\bigl(x\bigl([n]P\bigr)\bigr) = x\bigl([n]\frobisog_{E'}^i(P')\bigr)
  \quad\text{for any $n$.}
\end{equation*}
Since $\frob_{E'}^i(P')$ is a primitive element of $E'[p^k]$, for any
$0 < i < [\U_k:\F_q]$ we get a new interpolating polynomial as
required by C2 and we only have to do the rational fraction
reconstruction step of the Cauchy interpolation. If
$\frac{2[\U_k:\F_q]}{\euler(p^k)} = p^{i_0-1}r > 1$, we must compute $p^{i_0-1}r$
polynomial interpolations and apply this algorithm to each of them in
order to deduce all the polynomials needed by C2.

We compute \eqref{eq:frob} via square-and-multiply, this costs
$\Theta(d\Mult(p^kd)\log p)$ operations. Each application of
\eqref{eq:modcomp} is done via a \emph{modular composition}. The best
algorithm known for this problem is \cite{Uma09} and using it would
yield an asymptotic complexity similar\footnote{\cite{Uma09} has a
  slightly better dependency in $p$ and $d$ and a slightly worse
  dependency in $\ell$.} to the one of C2-AS-FI; however, \cite{Uma09}
is not useful on practical instances of the problem, thus we have to
resort to \cite{BrKu78} which has a much worse asymptotic complexity,
but performs much faster then our interpolation algorithm in practice.

A similar approach could be used inside the polynomial interpolation
step\footnote{See Section \ref{sec:C2-AS-FI}.} to deduce $A_k^{(0)}$
from $A_0^{(0)}$ using modular composition with the multiplication
maps of $E$ and $E'$ as described in \cite[$\S$2.3]{Cou96}. We didn't
implement this variant, though.


\subsection{Parallelisation of the loop}
\label{parallel}

The most expensive step of C2-AS-FI and C2SD-AS-FI, in theory as well
as in practice, is the final loop over the points of
$E'[p^k]$. Fortunately, this phase is very easy to parallelise with
very few overhead.

Let $n$ be the number of processors we wish to parallelise on,
suppose\footnote{Otherwise the parallelisation is straightforward.}
that $[\U_k:\F_q] = \frac{\euler(p^k)}{2}$ and set
$m=\left\lfloor\frac{\euler(p^k)}{2n}\right\rfloor$. We compute the
restriction of $\frobisog_E^{m}$ to $K_E[p^k]$ as
\begin{equation*}
  F^{(m)}(X) = F(X) \circ \cdots \circ F(X) \bmod T(X)
  \;\text{,}
\end{equation*}
this can be done with $\Theta(\log m)$ modular compositions via a
binary square-and-multiply-like approach.

Then compute the $n$ polynomials
\begin{equation*}
  A_{mi}(X) = A_{m(i-1)}(X) \circ F^{(m)}(X) \bmod T(X)
\end{equation*}
and distribute them to the $n$ processors so that they each work on a
separate slice of the $A_i$'s.



% Local Variables:
% mode:flyspell
% ispell-local-dictionary:"british"
% mode:TeX-PDF
% TeX-master: "ec-isogeny"
% End:
%
% LocalWords:  Schreier Artin pseudotrace frobenius bivariate Joux Sirvent FFT
% LocalWords:  Couveignes isogenies Schoof isogeny cryptosystems Lercier Hasse
% LocalWords:  precomputation arithmetics polylogarithmic Karatsuba precomputes
% LocalWords:  endomorphisms 
