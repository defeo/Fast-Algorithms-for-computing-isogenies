\documentclass[10pt]{beamer}

\usepackage[francais]{babel}
\usepackage[utf8]{inputenc}
\usepackage{amssymb}
\usepackage{amsthm}
%\usepackage[all]{xy}
\usepackage{graphicx}
%\usepackage{tikz}
%\usetikzlibrary{trees}
\usepackage[style=beamer,doi=false,isbn=false,eprint=false,maxnames=10]{biblatex}
\bibliography{defeo}

%% Stuff
\renewcommand{\le}{\leqslant}
\renewcommand{\ge}{\geqslant}  % comme François le demande...
\newcommand{\blue}[1]{\textcolor{blue}{#1}}  % colouring
%% Algèbre
\newcommand{\clot}[1]{\bar{#1}}  % clôture algèbrique
\newcommand{\card}[1]{\# #1}  % cardinalité d'un ensemble
\DeclareMathOperator{\car}{char}  % caractéristique d'un corps
\DeclareMathOperator{\Frac}{Frac}  % corps des fractions
\newcommand{\Z}{\mathbb{Z}}  % les entiers
\newcommand{\K}{\mathbb{K}}  % un corps
\newcommand{\LK}{\mathbb{L}}  % encore un corps
\newcommand{\U}{\mathbb{U}}  % encore un corps
\newcommand{\F}{\mathbb{F}}  % un corps fini
\newcommand{\Q}{\mathbb{Q}}  % les rationnels
\newcommand{\R}{\mathbb{R}}  % les réels
\newcommand{\C}{\mathbb{C}}  % les complexes
\newcommand{\isom}{\cong}  % isomorphisme de corps
\newcommand{\frob}{\varphi}  % fröbenius
\DeclareMathOperator{\Gal}{Gal}  % groupe de Galois
\DeclareMathOperator{\Tr}{Tr}  % trace
\DeclareMathOperator{\PTr}{PTr}  % pseudotrace
\DeclareMathOperator{\Norm}{N} % norme
\newcommand{\euler}{\phi}  % indicatrice d'Euler
\DeclareMathOperator{\ord}{ord}  % l'ordre d'un élément
\newcommand{\AS}[1]{\mathcal{#1}}  % la police des polynômes d'AS
\DeclareMathOperator{\rev}{rev}  % le reverse d'un polynôme
%% Courbes
\DeclareMathOperator{\Jac}{Jac}  % la jacobienne
\newcommand{\Proj}{\mathbb{P}}  % espace projectif
\newcommand{\0}{\mathcal{O}}  % point de base d'une courbe
\newcommand{\ecpoint}[3]{[#1:#2:#3]}  % un point d'une courbe
\newcommand{\isog}[1]{\mathcal{#1}}  % la police des isogénies
\newcommand{\I}{\isog{I}}  % une isogénie I
\newcommand{\Hasse}{H}  % l'invariant de Hasse
\newcommand{\divpol}{f}  % polynôme de division
%% Autre
\newcommand{\tildO}{\tilde{O}}  % la notation O~ qui oublie les log
\newcommand{\Mint}{\mathrm{\sf M}_\text{int}}  % fonction de multiplication
\newcommand{\Mpol}[1][]{\mathrm{\sf M}_\text{pol}^{#1}}  % fonction de multiplication
\newcommand{\Mult}[1][]{\mathrm{\sf M}_{#1}}  % fonction de multiplication
\newcommand{\Push}{\mathrm{\sf P}}  % fonction de push-down
\newcommand{\Lift}{\mathrm{\sf L}}  % fonction de lift-up
\newcommand{\Trace}{\mathrm{\sf T}}  % fonction de trace
\newcommand{\Frob}{\mathrm{\sf F}}  % fonction de frobenius itéré
\newcommand{\Ptr}{\mathrm{\sf PT}}  % fonction de pseudo-trace
\newcommand{\ModComp}{\mathrm{\sf C}}  % fonction de composition modulaire
\newcommand{\alg}[1]{\textsf{#1}}  % la police des algorithmes
\newcommand{\wrt}{\dashv}  % appartenance forte, a\wrt A signifie que a est représenté comme un élément de A
\DeclareMathOperator{\op}{op}  % une opération

% beamer-specific
%\setbeamertemplate{theorem begin}{
%  \begin{\inserttheoremblockenv}{
%      Théorème
%      \ifthenelse{\equal{\inserttheoremaddition}{}}
%		 {}
%		 {(\inserttheoremaddition)}    
		 %  \ifx\inserttheoremaddition\@empty\else\ (\inserttheoremaddition)\fi
%    }
%}

\mode<presentation>{%
  \usetheme[]{Madrid}
  \usefonttheme[onlymath]{serif}
  \usecolortheme{crane}
% \usecolortheme{rose}
}


\title{Calcul rapide d'isogénies en genre $1$}
\author{Luca~De~Feo}
\institute[INRIA Saclay]{INRIA Saclay, Projet TANC}
\date[ENST, Paris, 15 Octobre 2010]{15 Octobre 2010\\Séminaire BAC, ENST, Paris}


\AtBeginSection[]
{
  \begin{frame}<beamer>
    \frametitle{Plan}
    \tableofcontents[currentsection]
  \end{frame}
}


\begin{document}

\begin{frame}
  \titlepage
\end{frame}

%%

\section{Quoi?}

\begin{frame}
  \frametitle{Isogénies entre courbes elliptiques}
  
  \vspace{-2mm}

  {\large \[\I:E\to E'\]}
  \textbf{Isogénie (separable)}: morphisme rationnel (separable)
  non-constant préservant le point à l'infini.
  
  \begin{block}{Propriétés}
    \begin{itemize}
    \item Noyau fini, surjective (dans $\clot{\K}$), 
    \item définie par des fractions rationnelles avec un pôle à l'infini,
    \item $\card{E(\F_{q^n})} = \card{E'(\F_{q^n})}$ pour tout $n$,
    \item Isogénie \emph{duale}: $\;[m] = \I\circ\hat{\I}$.
    \end{itemize}
  \end{block}

  \vspace{-1mm}

  \begin{block}{
	\begin{overprint}
	\onslide<1> Multiplication	
	\onslide<2> Endomorphisme de Frobenius
	\onslide<3> Isogénie separable, degré impair (simplified Weierstrass model)
	\end{overprint}
      }
    \begin{overprint}
      \onslide<1>
      \[\begin{aligned}
	{}[m] : E(\clot{\K}) &\rightarrow E(\clot{\K})\\
	                   P &\mapsto [m]P
      \end{aligned}\]
      $\ker\I = E[m], \quad\deg\I = m^2$.

      \onslide<2>
      \[\begin{aligned}
	\frob : E(\clot{\K}) &\rightarrow E(\clot{\K})\\
	               (X,Y) &\mapsto (X^q,Y^q)
      \end{aligned}\]
      $\ker\frob = \{\0\}, \quad\deg\I = q$.

      \onslide<3>
      \[\quad\I(X,Y) = \left(\frac{g(X)}{h^2(X)},
      cY\left(\frac{g(X)}{h^2(X)}\right)'\right)\]
      $\;\ell\;=\;\deg\I\;=\;
      \card{\ker\I} \;=\; 2\deg h + 1\;$ odd.
    \end{overprint}
  \end{block}
\end{frame}

%%

\begin{frame}
  \frametitle{S{\scriptsize choof}E{\scriptsize lkies}A{\scriptsize tkin}}

  \begin{block}{Schoof}
    \begin{itemize}
    \item $\card{E(\F_q)} = q + 1 - t\;$ avec $t$ la \emph{trace} du Frobenius:
      $\frob^2 - [t]\circ\frob + [q] = 0$;
    \item Calculer $\;t\bmod\ell\;$ pour les premiers $\;\ell<O(\log q)$;
    \item Équivalent à calculer l'action de $\;\frob\;$ sur $\;E[\ell]\isom\Z/\ell^2\Z$;
    \item \textbf{En pratique:} calculs modulo le \emph{polynôme de
        division} (\alert{degré $\sim O(\ell^2)$}).
    \end{itemize}
  \end{block}

  \begin{block}{Elkies}
    \begin{itemize}
    \item Par le théorème de l'isogénie duale $\;[\ell] = \I\circ\hat{\I}$;
    \item $E[\ell]\;$ contient le sous-groupe $\;\ker\I\isom\Z/\ell\Z$;
    \item Si $\;\ker\I\;$ est défini sur $\;\K$:
      \begin{enumerate}
      \item Trouver la \alert<2>{courbe isogène} $\;E/\ker\I$,
      \item calculer l'action de  $\;\frob\;$ sur $\;\ker\I$,
      \item \textbf{En pratique:} calculs modulo le
        \alert<2>{dénominateur} de l'isogénie (\alert{degré $\sim
          O(\ell^2)$}).
      \end{enumerate}
    \item Marche en moyenne pour la moitié des nombres premiers.
    \end{itemize}
  \end{block}
  
  Atkin\dots
\end{frame}

%%

\begin{frame}
  \frametitle{Cryptographie moderne}
  
  \begin{block}{L'attaque de \cite{gaudry+hess+smart02}}
    \begin{itemize}
    \item Calcule une isogénie $\I:E/\F_{q^d}\to H/\F_q\;$ par
      descente de Weil;
    \item $H\;$ hyperelliptique de genre $\sim d$: log discret sur
      $\;H\;$ plus facile pour certains jeux de paramètres.
    \end{itemize}
  \end{block}

  \vspace{-1mm}

  \begin{block}{Le cryptosystème de \cite{teske06}}
    Pas toutes les courbes dans une classe d'isogénie sont vulnérables
    à GHS, cela permet d'obtenir un cryptosystème \emph{à trappe}:
    \begin{enumerate}
    \item Sélectionner $\;E\;$ GHS-vulnérable;
    \item Obtenir $\;E'\;$ non-GHS-vulnérable par une marche
      aléatoire dans le graphe d'isogénies;
    \item Utiliser $\;E'\;$ pour le matériel publique, donner
      $\;E\;$ à une autorité de confiance.
    \end{enumerate}
  \end{block}

  \vspace{-1mm}

  \begin{block}{Autres protocoles utilisant des isogénies}
    \begin{itemize}
    \item Le protocole \emph{à la Diffie-Hellman} de \cite{rostovtsev+stolbunov06},
    \item Fonctions de hachage: \cite{charles+lauter+goren09}.
    \end{itemize}
  \end{block}
\end{frame}

%%

\begin{frame}
  \frametitle{Quelles entrées, quelles sorties?}

  \begin{block}{Entrées}
    \begin{enumerate}
    \item Donnés $E$ et un sous-groupe $H$ de cardinal $\ell$;
    \item Donnés $E, \ell$;
    \item Donnés $E, E'$;
    \item Donnés $E, E', \ell$;
    \end{enumerate}
  \end{block}

  \begin{block}{Sorties}
    \begin{enumerate}
    \item Existence: dire s'il existe $\I:E\to E'$ (de noyau $H$, ou de degré $\ell$);
    \item Calculer: des expressions rationnelles en $x$ et $y$ pour $\I$.
    \end{enumerate}
  \end{block}
\end{frame}

%%
%%

\section{Comment?}

%%

\begin{frame}
  \frametitle{Formules de Vélu}
  
  \begin{block}{Formules de Vélu (corps algébriquement clos)}
    Étant donné le noyau $H$
    \begin{align*}
      &\I(\0_E) = \I(\0_{E/H})\\
      &\begin{aligned}
        \I(P) = \Biggl( &x(P) + \sum_{Q\in H - \{\0_E\}}x(P+Q) - x(Q) \quad,\\
        &y(P) + \sum_{Q\in H - \{\0_E\}}y(P+Q) - y(Q) \Biggr) \text{.}
      \end{aligned}
    \end{align*}
  \end{block}

  \begin{block}{Pour $p\ge 3$, étant donné $h(x)$ s'annulant sur $H$}
    {\footnotesize
      \[
      y^2 = f(x)\text{,}
      \qquad
      t = \sum_{Q\in G^\ast} f'(Q)\text{,}
      \quad
      u = \sum_{Q\in G^\ast} 2f(Q)\text{,}
      \quad
      w = u + \sum_{Q\in G^\ast} x(Q)f'(Q)\text{,}\]}
    \[\alert{\I(x,y) = \left(\frac{g(x)}{h(x)}, y\left(\frac{g(x)}{h(x)}\right)'\right)}
    \quad\text{avec}\quad
    \frac{g(x)}{h(x)} = x + t\frac{h'(x)}{h(x)} - u\left(\frac{h'(x)}{h(x)}\right)'\]
  \end{block}
\end{frame}

%%

\begin{frame}
  \frametitle{Polynôme modulaire}

  \begin{center}
    $\Phi_\ell(X,Y)$, le polynôme minimal sur $\C$ de la fonction
    modulaire $j(\ell\tau)$
  \end{center}

  \begin{block}{Propriétés}
    \begin{itemize}
    \item Les racines de $\Phi_\ell(X,j(E))$ sont les $j$-invariants
      des courbes $\ell$-isogènes à $E$;
    \item Symétrique en $X$ et $Y$, degré $\ell+1$;
    \item Coefficients entiers de taille $O(\ell)$.
    \end{itemize}
  \end{block}

  \begin{block}{En pratique}
    \begin{itemize}
    \item On utilise plutôt d'autres invariants modulaires, mais la théorie ne change pas;
    \item Plusieurs tables de polynômes modulaires précalculés (Enge-Morain, Magma, \dots);
    \item Calcul en \alert{$\tildO(\ell^3)$} par \cite{sutherland10:modpol};
    \item et aussi: calcul de $\Phi_\ell \bmod m$ en \alert{$\tildO(\ell^2\log^2m)$}.
    \end{itemize}
  \end{block}
\end{frame}

%%

\begin{frame}
  \frametitle{\cite{bostan+morain+salvy+schost08}}

  \begin{center}
    \Large
    Donnés $E, E', \ell$, calculer $\I:E\to E'$
  \end{center}

  Par les formules de Vélu:
  $\I(x,y) = \left(\frac{g(x)}{h(x)}, y\left(\frac{g(x)}{h(x)}\right)'\right)$,
  d'où
  \[(x^3 + ax + b){\left(\frac{g(x)}{h(x)}\right)'}^2 =
  \left(\frac{g(x)}{h(x)}\right)^3 + a'\frac{g(x)}{h(x)} + b'\]
  
  \begin{block}{Algorithme}
    \begin{enumerate}
    \item Changement de variables $ S(x) =
      \sqrt{\frac{h(1/x^2)}{g(1/x^2)}} \quad\Leftrightarrow\quad
      \frac{g(x)}{h(x)} = \frac{1}{S(1/\sqrt{x})^2}$;
    \item Solution série de l'équa diff $\;(bx^6 + ax^4 +
      1){S'}^2 = 1 + a'S^4 + b'S^6$;
    \item Inverser le changement de variables, reconstruire une
      fraction rationnelle.
    \end{enumerate}
  \end{block}
\end{frame}

%%

\begin{frame}
  \frametitle{SEA}

  Pour tout $\ell\sim\log q$ premier d'Elkies
  \begin{enumerate}
  \item Trouver une racine $j'$ de $\;\Phi_\ell(X,j(E))$;
  \item \alert<2>{En déduire une courbe $\;E'$, $\;\ell$-isogène à $E$};
  \item Utiliser BMSS pour calculer $\;\I:E\to E'$;
  \item En déduire $\;t\bmod\ell$.
  \end{enumerate}
\end{frame}

%%

\begin{frame}
  \frametitle{Isogénies canoniques}

  En général pour le modèle de Weierstrass $\I(x,y) =
  \left(\frac{g(x)}{h(x)},
    \alert{c}y\left(\frac{g(x)}{h(x)}\right)'\right)$,

  \begin{itemize}
  \item Quand $c=1$ on dit que le modèles de $E$ et $E'$ sont
    \emph{canoniques} ou \emph{strictes} ou \emph{normalisés};
  \item Dans ce cas, on peut appliquer BMSS directement: complexité \alert{$\tildO(\ell)$};
  \item Sinon il faut passer par le polynôme canonique: complexité \alert{$\tildO(\ell^2)$}.
  \end{itemize}
\end{frame}

%%

\begin{frame}
  \frametitle{Lercier-Sirvent}

  \begin{center}
    \textbf{Idée:} Lifter dans les $p$-adiques

    \textbf{Problème:} Comment maintenir des modèles normalisés?
  \end{center}

  \begin{block}{Algorithme}
    \begin{enumerate}
    \item Lifter $\Phi_\ell$ dans $\Q_q$;
    \item Lifter $j$ et $j'$ en maintenant $\tilde{\Phi}_\ell(j,j')=0$;
    \item Lifter les modèles normalisés;
    \item Appliquer BMSS dans $\Q_q$.
    \end{enumerate}
    Précision $p$-adique requise $O(\log^2\ell)$, complexité totale $\tildO(\ell^2)$.
  \end{block}
\end{frame}

%%

\begin{frame}
  \frametitle{SEA revisité}
  \cite{silverman:elliptic}  
\end{frame}

%%

\begin{frame}
  \frametitle{Les algorithmes de Couveignes}
  
\end{frame}

%%

\begin{frame}
  \frametitle{La complexité de Couveignes II}
  
\end{frame}

%%

\begin{frame}
  \frametitle{Isogénies de degré inconnu}
  
\end{frame}

%%
%%

\section{Et maintenant?}

\begin{frame}
  \frametitle{Quelques temps d'exécution}
  
\end{frame}

%%

\begin{frame}
  \frametitle{Marathon d'implantations}

\end{frame}

%%

\begin{frame}
  \frametitle{En quête de la complexité quasi-linéaire}
  
\end{frame}

%%

\begin{frame}
  \frametitle{Isogénies de degré inconnu}
  
\end{frame}

%%

\begin{frame}
  \frametitle{Z'en voulez plus?}
  
\end{frame}

%%
%%

\begin{frame}[allowframebreaks]
  \frametitle{References}

  \beamertemplatebookbibitems
  \printbibliography[type=book]
  \framebreak
  \beamertemplatearticlebibitems
  \printbibliography[nottype=book]
\end{frame}

\end{document}


% Local Variables:
% mode:flyspell
% ispell-local-dictionary:"francais"
% mode:TeX-PDF
% End:
%

% LocalWords:  Isogeny abelian isogenies hyperelliptic supersingular Frobenius
% LocalWords:  isogenous
