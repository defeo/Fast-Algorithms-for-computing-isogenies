\documentclass[10pt]{beamer}

\usepackage[francais]{babel}
\usepackage[utf8]{inputenc}
\usepackage{amsmath}
\usepackage{amssymb}
\usepackage{amsthm}
\usepackage[all]{xy}
\usepackage{graphicx}
\usepackage[style=beamer,doi=false,isbn=false,eprint=false,maxnames=10]{biblatex}
\bibliography{defeo}

%% Stuff
\renewcommand{\le}{\leqslant}
\renewcommand{\ge}{\geqslant}  % comme François le demande...
\newcommand{\blue}[1]{\textcolor{blue}{#1}}  % colouring
%% Algèbre
\newcommand{\clot}[1]{\bar{#1}}  % clôture algèbrique
\newcommand{\card}[1]{\# #1}  % cardinalité d'un ensemble
\DeclareMathOperator{\car}{char}  % caractéristique d'un corps
\DeclareMathOperator{\Frac}{Frac}  % corps des fractions
\newcommand{\Z}{\mathbb{Z}}  % les entiers
\newcommand{\K}{\mathbb{K}}  % un corps
\newcommand{\LK}{\mathbb{L}}  % encore un corps
\newcommand{\U}{\mathbb{U}}  % encore un corps
\newcommand{\F}{\mathbb{F}}  % un corps fini
\newcommand{\Q}{\mathbb{Q}}  % les rationnels
\newcommand{\R}{\mathbb{R}}  % les réels
\newcommand{\C}{\mathbb{C}}  % les complexes
\newcommand{\isom}{\cong}  % isomorphisme de corps
\newcommand{\frob}{\varphi}  % fröbenius
\DeclareMathOperator{\Gal}{Gal}  % groupe de Galois
\DeclareMathOperator{\Tr}{Tr}  % trace
\DeclareMathOperator{\PTr}{PTr}  % pseudotrace
\DeclareMathOperator{\Norm}{N} % norme
\newcommand{\euler}{\phi}  % indicatrice d'Euler
\DeclareMathOperator{\ord}{ord}  % l'ordre d'un élément
\newcommand{\AS}[1]{\mathcal{#1}}  % la police des polynômes d'AS
\DeclareMathOperator{\rev}{rev}  % le reverse d'un polynôme
%% Courbes
\DeclareMathOperator{\Jac}{Jac}  % la jacobienne
\newcommand{\Proj}{\mathbb{P}}  % espace projectif
\newcommand{\0}{\mathcal{O}}  % point de base d'une courbe
\newcommand{\ecpoint}[3]{[#1:#2:#3]}  % un point d'une courbe
\newcommand{\isog}[1]{\mathcal{#1}}  % la police des isogénies
\newcommand{\I}{\isog{I}}  % une isogénie I
\newcommand{\Hasse}{H}  % l'invariant de Hasse
\newcommand{\divpol}{f}  % polynôme de division
%% Autre
\newcommand{\tildO}{\tilde{O}}  % la notation O~ qui oublie les log
\newcommand{\Mint}{\mathrm{\sf M}_\text{int}}  % fonction de multiplication
\newcommand{\Mpol}[1][]{\mathrm{\sf M}_\text{pol}^{#1}}  % fonction de multiplication
\newcommand{\Mult}[1][]{\mathrm{\sf M}_{#1}}  % fonction de multiplication
\newcommand{\Push}{\mathrm{\sf P}}  % fonction de push-down
\newcommand{\Lift}{\mathrm{\sf L}}  % fonction de lift-up
\newcommand{\Trace}{\mathrm{\sf T}}  % fonction de trace
\newcommand{\Frob}{\mathrm{\sf F}}  % fonction de frobenius itéré
\newcommand{\Ptr}{\mathrm{\sf PT}}  % fonction de pseudo-trace
\newcommand{\ModComp}{\mathrm{\sf C}}  % fonction de composition modulaire
\newcommand{\alg}[1]{\textsf{#1}}  % la police des algorithmes
\newcommand{\wrt}{\dashv}  % appartenance forte, a\wrt A signifie que a est représenté comme un élément de A
\DeclareMathOperator{\op}{op}  % une opération

% beamer-specific
%\setbeamertemplate{theorem begin}{
%  \begin{\inserttheoremblockenv}{
%      Théorème
%      \ifthenelse{\equal{\inserttheoremaddition}{}}
%		 {}
%		 {(\inserttheoremaddition)}    
		 %  \ifx\inserttheoremaddition\@empty\else\ (\inserttheoremaddition)\fi
%    }
%}

\mode<presentation>{%
  \usetheme[]{Madrid}
  \usefonttheme[onlymath]{serif}
  \usecolortheme{crane}
% \usecolortheme{rose}
}


\title{Calcul rapide d'isogénies en genre $1$}
\author{Luca~De~Feo}
\institute[INRIA Saclay]{INRIA Saclay, Projet TANC}
\date[Caen, 21 Octobre 2010]{21 Octobre 2010\\Séminaire Crypto, Université de Caen, Paris}

%%%%%%% This talk originated from the slides for this other talk !
% \title{Calcul rapide d'isogénies en genre $1$}
% \author{Luca~De~Feo}
% \institute[INRIA Saclay]{INRIA Saclay, Projet TANC}
% \date[ENST, Paris, 15 Octobre 2010]{15 Octobre 2010\\Séminaire BAC, ENST, Paris}


\AtBeginSection[]
{
  \begin{frame}<beamer>
    \frametitle{Plan}
    \tableofcontents[currentsection]
  \end{frame}
}


\begin{document}

\begin{frame}
  \titlepage
\end{frame}

%%

\section{Quoi?}

\begin{frame}
  \frametitle{Isogénies entre courbes elliptiques}
  
  \vspace{-2mm}

  {\large \[\I:E\to E'\]}
  \textbf{Isogénie (separable)}: morphisme rationnel (separable)
  non-constant préservant le point à l'infini.
  
  \begin{block}{Propriétés}
    \begin{itemize}
    \item Noyau fini, surjective (dans $\clot{\K}$), 
    \item définie par des fractions rationnelles avec un pôle à l'infini,
    \item $\card{E(\F_{q^n})} = \card{E'(\F_{q^n})}$ pour tout $n$,
    \item Isogénie \emph{duale}: $\;[m] = \I\circ\hat{\I}$.
    \end{itemize}
  \end{block}

  \vspace{-1mm}

  \begin{block}{
	\begin{overprint}
	\onslide<1> Multiplication	
	\onslide<2> Endomorphisme de Frobenius
	\onslide<3> Isogénie separable, degré impair (modèle de Weierstrass simplifié)
	\end{overprint}
      }
    \begin{overprint}
      \onslide<1>
      \[\begin{aligned}
	{}[m] : E(\clot{\K}) &\rightarrow E(\clot{\K})\\
	                   P &\mapsto [m]P
      \end{aligned}\]
      $\ker\I = E[m], \quad\deg\I = m^2$.

      \onslide<2>
      \[\begin{aligned}
	\frob : E(\clot{\K}) &\rightarrow E(\clot{\K})\\
	               (X,Y) &\mapsto (X^q,Y^q)
      \end{aligned}\]
      $\ker\frob = \{\0\}, \quad\deg\I = q$.

      \onslide<3>
      \[\quad\I(X,Y) = \left(\frac{g(X)}{h^2(X)},
      cY\left(\frac{g(X)}{h^2(X)}\right)'\right)\]
      $\;\ell\;=\;\deg\I\;=\;
      \card{\ker\I} \;=\; 2\deg h + 1\;$ impair.
    \end{overprint}
  \end{block}
\end{frame}

%%

\begin{frame}
  \frametitle{S{\scriptsize choof}E{\scriptsize lkies}A{\scriptsize tkin}}

  \begin{block}{\cite{schoof85}}
    \begin{itemize}
    \item $\card{E(\F_q)} = q + 1 - t\;$ avec $t$ la \emph{trace} du Frobenius:
      $\frob^2 - [t]\circ\frob + [q] = 0$;
    \item Calculer $\;t\bmod\ell\;$ pour les premiers $\;\ell<O(\log q)$;
    \item Équivalent à calculer l'action de $\;\frob\;$ sur $\;E[\ell]\isom(\Z/\ell\Z)^2$;
    \item \textbf{En pratique:} calculs modulo le \emph{polynôme de
        division} (\alert{degré $\sim O(\ell^2)$}).
    \end{itemize}
  \end{block}

  \begin{block}{Elkies (\cite{elkies98,schoof95})}
    \begin{itemize}
    \item Par le théorème de l'isogénie duale $\;[\ell] = \I\circ\hat{\I}$;
    \item $E[\ell]\;$ contient le sous-groupe $\;\ker\I\isom\Z/\ell\Z$;
    \item Si $\;\I\;$ est définie sur $\;\K$:
      \begin{enumerate}
      \item Trouver la \alert<2>{courbe isogène} $\;E/\ker\I$,
      \item calculer l'action de  $\;\frob\;$ sur $\;\ker\I$,
      \item \textbf{En pratique:} calculs modulo le
        \alert<2>{dénominateur} de l'isogénie (\alert{degré $\sim
          O(\ell)$}).
      \end{enumerate}
    \item Marche en moyenne pour la moitié des nombres premiers.
    \end{itemize}
  \end{block}
  
  \cite{atkin88}\dots
\end{frame}

%%

\begin{frame}
  \frametitle{Cryptographie moderne}
  
  \begin{block}{L'attaque de \cite{gaudry+hess+smart02}}
    \begin{itemize}
    \item Calcule une isogénie $\I:E/\F_{q^d}\to H/\F_q\;$ par
      descente de Weil;
    \item $H\;$ hyperelliptique de genre $\sim d$: log discret sur
      $\;H\;$ plus facile pour certains jeux de paramètres.
    \end{itemize}
  \end{block}

  \vspace{-1mm}

  \begin{block}{Le cryptosystème de \cite{teske06}}
    Pas toutes les courbes dans une classe d'isogénie sont vulnérables
    à GHS, cela permet d'obtenir un cryptosystème \emph{à trappe}:
    \begin{enumerate}
    \item Sélectionner $\;E\;$ GHS-vulnérable;
    \item Obtenir $\;E'\;$ non-GHS-vulnérable par une marche
      aléatoire dans le graphe d'isogénies;
    \item Utiliser $\;E'\;$ pour le matériel publique, donner
      $\;E\;$ à une autorité de confiance.
    \end{enumerate}
  \end{block}

  \vspace{-1mm}

  \begin{block}{Autres protocoles utilisant des isogénies}
    \begin{itemize}
    \item Le protocole \emph{à la Diffie-Hellman} de \cite{rostovtsev+stolbunov06},
    \item Fonctions de hachage: \cite{charles+lauter+goren09}.
    \end{itemize}
  \end{block}
\end{frame}

%%

\begin{frame}
  \frametitle{Quelles entrées, quelles sorties?}

  \setbeamertemplate{enumerate items}[circle]
  \begin{block}{Entrées}
    \begin{enumerate}
    \item Donnés $E$ et un sous-groupe $H$ de cardinal $\ell$;
    \item Donnés $E, \ell$;
    \item Donnés $E, E'$;
    \item Donnés $E, E', \ell$;
    \end{enumerate}
  \end{block}

  \begin{block}{Sorties}
    \renewcommand{\insertenumlabel}{\Alph{enumi}}
    \begin{enumerate}
    \item Existence: dire s'il existe $\I:E\to E'$ (de noyau $H$, ou de degré $\ell$);
    \item Calculer: des expressions rationnelles en $x$ et $y$ pour $\I$.
    \end{enumerate}
  \end{block}

  \begin{itemize}
  \item \textbf{1A, 2A:} Toujours vrai;
  \item \textbf{3A:} $\card{E(\F_q)} = \card{E'(\F_q)}$?
  \end{itemize}
\end{frame}

%%
%%

\section{Comment?}

%%

\begin{frame}
  \frametitle{Formules de Vélu}
  
  \begin{block}{\cite{velu71} (corps algébriquement clos)}
    Étant donné le noyau $H$, calcule $\;\I : E\to E/H\;$ donnée par 
    \begin{align*}
      &\I(\0_E) = \I(\0_{E/H})\text{,}\\
      &\begin{aligned}
        \I(P) = \Biggl(x(P) + \sum_{Q\in H^\ast}x(P+Q) - x(Q),
        y(P) + \sum_{Q\in H^\ast}y(P+Q) - y(Q) \Biggr) \text{.}
      \end{aligned}
    \end{align*}
  \end{block}

  \begin{block}{Pour $p\ge 3$, étant donné $h(x)$ s'annulant sur $H$}
    {\footnotesize
      \[
      y^2 = f(x) = x^3 + a_2x^2 + a_4x + a_6\text{,}
      \quad
      t = \!\!\sum_{Q\in H^\ast} \!\!\!f'(Q)\text{,}
      \quad
      u = \!\!\sum_{Q\in H^\ast} \!\!\!2f(Q)\text{,}
      \quad
      w = u + \!\!\sum_{Q\in H^\ast} \!\!\!x(Q)f'(Q)\text{,}\]}
    \[\alert{\I(x,y) = \left(\frac{g(x)}{h(x)}, y\left(\frac{g(x)}{h(x)}\right)'\right)}
    \quad\text{avec}\quad
    \frac{g(x)}{h(x)} = x + t\frac{h'(x)}{h(x)} - u\left(\frac{h'(x)}{h(x)}\right)'\]
    \[E/H : y^2 = x^3 + a_2x^2 + (a_4 - 5t)x + a_6 - 4a_2t - 7w\]
  \end{block}
\end{frame}

%%

\begin{frame}
  \frametitle{Polynôme modulaire}

  \begin{center}
    $\Phi_\ell(X,Y)$, le polynôme minimal sur $\C$ de la fonction
    modulaire $j(\ell\tau)$
  \end{center}

  \begin{block}{Propriétés}
    \begin{itemize}
    \item Les racines de $\Phi_\ell(X,j(E))$ sont les $j$-invariants
      des courbes $\ell$-isogènes à $E$;
    \item Symétrique en $X$ et $Y$, degré $\ell+1$;
    \item Coefficients entiers de taille $O(\ell)$.
    \end{itemize}
  \end{block}

  \begin{block}{En pratique}
    \begin{itemize}
    \item On utilise plutôt d'autres invariants modulaires, mais la théorie ne change pas;
    \item Plusieurs tables de polynômes modulaires précalculés (Enge-Morain, Magma, \dots);
    \item Calcul en \alert{$\tildO(\ell^3)$} par \cite{sutherland10:modpol};
    \item et aussi: calcul de $\Phi_\ell \bmod m$ en \alert{$\tildO(\ell^2\log m)$}.
    \end{itemize}
  \end{block}
\end{frame}

%%

\begin{frame}
  \frametitle{\cite{bostan+morain+salvy+schost08}}

  \begin{center}
    \Large
    Donnés $E, E', \ell$, calculer $\I:E\to E'$
  \end{center}

  Par les formules de Vélu:
  $\I(x,y) = \left(\frac{g(x)}{h(x)}, \alert<2>{c}y\left(\frac{g(x)}{h(x)}\right)'\right)$,
  d'où
  \[\alert<2>{c^2}(x^3 + ax + b){\left(\frac{g(x)}{h(x)}\right)'}^2 =
  \left(\frac{g(x)}{h(x)}\right)^3 + a'\frac{g(x)}{h(x)} + b'\]
  
  \begin{block}{Algorithme BMSS}
    \begin{enumerate}
    \item Changement de variables $ S(x) =
      \sqrt{\frac{h(1/x^2)}{g(1/x^2)}} \quad\Leftrightarrow\quad
      \frac{g(x)}{h(x)} = \frac{1}{S(1/\sqrt{x})^2}$;
    \item Solution série de l'équa diff $\;\alert<2>{c^2}(bx^6 + ax^4 +
      1){S'}^2 = 1 + a'S^4 + b'S^6$;
    \item Inverser le changement de variables, reconstruire une
      fraction rationnelle.
    \end{enumerate}
  \end{block}
\end{frame}
%%

\begin{frame}
  \frametitle{Quelles entrées, quelles sorties?}

  \setbeamertemplate{enumerate items}[circle]
  \begin{block}{Entrées}
    \begin{enumerate}
    \item Donnés $E$ et un sous-groupe $H$ de cardinal $\ell$;
    \item Donnés $E, \ell$;
    \item Donnés $E, E'$;
    \item Donnés $E, E', \ell$;
    \end{enumerate}
  \end{block}

  \begin{block}{Sorties}
    \renewcommand{\insertenumlabel}{\Alph{enumi}}
    \begin{enumerate}
    \item Existence: dire s'il existe $\I:E\to E'$ (de noyau $H$, ou de degré $\ell$);
    \item Calculer: des expressions rationnelles en $x$ et $y$ pour $\I$.
    \end{enumerate}
  \end{block}

  \begin{itemize}
  \item \textbf{1B:} Formules de Vélu;
  \item \textbf{4A:} $\Phi_\ell(j(E),j(E')) = 0$?
  \item \textbf{4B $\Rightarrow$ 2B:} Factoriser $\Phi_\ell(X,j(E))$;
  \item \textbf{4B:} BMSS, si possible\dots
  \item \textbf{3B:} ???
  \end{itemize}
\end{frame}


%%

\begin{frame}
  \frametitle{SEA}

  Pour tout $\ell\sim\log q$ premier d'Elkies
  \begin{enumerate}
  \item Trouver une racine $j'$ de $\;\Phi_\ell(X,j(E))$;
  \item \alert<2>{En déduire une courbe \[E':y^2=x^3+ax+b\quad\text{avec}\quad j(E')=j'\text{,}\] $\ell$-isogène à $E$};
  \item \alert<2>{Utiliser BMSS pour calculer $\;\I:E\to E'$};
  \item En déduire $\;t\bmod\ell$.
  \end{enumerate}
\end{frame}

%%

\begin{frame}
  \frametitle{Isogénies canoniques}

  En général pour le modèle de Weierstrass $\I(x,y) =
  \left(\frac{g(x)}{h(x)},
    \alert{c}y\left(\frac{g(x)}{h(x)}\right)'\right)$,

  \begin{itemize}
  \item Quand $c=1$ on dit que le modèles de $E$ et $E'$ sont
    \emph{canoniques} ou \emph{strictes} ou \emph{normalisés};
  \item Dans le cadre de SEA, \cite{elkies98} obtient un modèle
    normalisé $E''\isom E'$ en évaluant les dérivées partielles de
    $\Phi_\ell$.
  \end{itemize}

  \begin{itemize}
  \item \textbf{1B:} Formules de Vélu\dotfill$O(\ell)$;
  \item \textbf{4A:} $\Phi_\ell(j(E),j(E')) = 0$?\dotfill$\tildO(\ell^2)$;
  \item \textbf{2B:} Factoriser $\Phi_\ell(X,j(E))$ + \cite{elkies98} + BMSS\dotfill$\tildO(\ell^2)$;
  \item \textbf{4B:} BMSS, seulement dans le cas \emph{canonique}\dotfill$\tildO(\ell)$;
  \item \textbf{2B $\Rightarrow$ 4B:} \cite{elkies98} + BMSS + isomorphisme\dotfill$\tildO(\ell^2)$;
  \item \textbf{3B:} ???
  \end{itemize}
\end{frame}

%%
%%

\section{$p$-Comment?}

\begin{frame}
  \frametitle{\cite{lercier+sirvent08}}

  \begin{center}
    \textbf{Idée:} Lifter dans les $p$-adiques

    \textbf{Problème:} Comment maintenir des modèles normalisés?
  \end{center}

  \begin{block}<2>{Algorithme}
    \begin{enumerate}
    \item Lifter $j$ et $j'$ en maintenant $\Phi_\ell(\tilde{j},\tilde{j}')=0$;
    \item Lifter $E$;
    \item Calculer un modèle $\ell$-normalisé pour $\tilde{j}'$ par \cite{elkies98};
    \item Appliquer BMSS dans $\Q_q$.
    \end{enumerate}
    Précision $p$-adique requise $O(\log^2\ell)$, complexité totale
    $\tildO(\ell^2)$,\\ \alert{même dans le cas \emph{canonique}!}
  \end{block}

  \[\xymatrix{
    \tilde{E}\alt<2>{\ar[r]^{\tilde{\I}}}{\ar[rr]^{???}} & \only<2>{\tilde{E''}\ar@{<->}[r]^\isom} & \tilde{E'}\\
    E\ar[rr]^\I\ar@{-->}[u] & & E'\ar@{-->}[u]
  }\]
\end{frame}

%%

\begin{frame}
  \frametitle{SEA revisité}

  Pour tout $\ell\sim\log q$ premier d'Elkies
  \begin{enumerate}
  \item Lifter $E$ dans $\Q_q$;
  \item Trouver une racine $\tilde{j}'$ dans $\Q_q$ de $\;\Phi_\ell(X,j(\tilde{E}))$;
  \item Par \cite{elkies98}, obtenir une
    courbe \[\tilde{E}':y^2=x^3+ax+b\quad\text{avec}\quad
    j(\tilde{E}')=\tilde{j}'\text{,}\] $\ell$-isogène à $\tilde{E}$;
  \item Utiliser BMSS dans $\Q_q$ pour calculer $\;\tilde{\I}:\tilde{E}\to \tilde{E}'$;
  \item Réduire $\tilde{I}$ dans $\F_q$;
  \item En déduire $\;t\bmod\ell$.
  \end{enumerate}
\end{frame}

%%

{
\setbeamertemplate{navigation symbols}{}
\begin{frame}
  \frametitle{Les algorithmes de Couveignes}
  
  \begin{center}
    \textbf{Idée:} Envoyer $E[p^k]$ sur $E'[p^k]$
  \end{center}

  \begin{columns}[t]
    \begin{column}{0.5\textwidth}
      \centering\cite{couveignes94}

      \begin{uncoverenv}<2-> 
        \begin{itemize}
        \item Passer dans le groupe formel $\mathcal{E}$ de $E$: un
          \emph{point formel} est une série en un paramètre formel
          $\;\tau$;
        \item Fixer une précision \emph{assez grande} pour
          $\;\F_q[[\tau]]$ ($\sim\log_p4\ell$);
        \item Calculer un morphisme $\mathcal{U}(\tau) : \mathcal{E}
          \to \mathcal{E}'$;
        \item Reconstruire une fraction rationnelle
          $\;\frac{g(X)}{h(X)} = \frac{1}{\mathcal{U}(1/X)}$;
        \item Si $\frac{g}{h}$ est une isogénie, fini; sinon choisir
          un autre $\mathcal{U}$.
        \item<3> $\mathcal{U}$ est uniquement déterminé par son action
          sur $\;\mathcal{E}[p^k]\;$ pour tout $\;k$.
        \end{itemize}
      \end{uncoverenv}
    \end{column}
    \begin{column}{0.5\textwidth}
      \centering\cite{couveignes96}
      
      \begin{itemize}
      \item Calculer itérativement les extensions $\U_i/\F_q$
        t.q. $E[p^i]$ est défini dans $\U_i$;
      \item Sélectionner un $\;k\;$ \emph{assez grand} ($k\sim\log_p4\ell$);
      \item Calculer $\;P$, générateur de $\;E[p^k]$;
      \item Calculer $\;P'$, générateur de $\;E'[p^k]$;
      \item Calculer le polynôme $\;T\;$ s'annulant sur $\;E[p^k]$;
      \item Interpoler $\;A : x(P) \mapsto x(P')$;
      \item Reconstruire une fraction rationnelle $\;\frac{g}{h}\equiv A \bmod T$;
      \item Si $\frac{g}{h}$ est une isogénie, fini; sinon choisir un autre $P'$.
      \end{itemize}
    \end{column}
  \end{columns}
\end{frame}
}

%%

\begin{frame}
  \frametitle{Comment reconnaître une isogénie?}

  \begin{itemize}
  \item \textbf{Degré}: $\frac{g}{h}$ avec $\deg g=\ell$, $\deg h = \ell-1$;\hfill\alert{$O(1)$}
  \item \textbf{Facteur carré:} $h = \prod_{Q\in H^\ast}(X- x(Q)) = f^2\;$ si $\ell$ impair;\hfill$\tildO(\ell)$
  \item \textbf{Action de groupe:} Tester avec des points au hasard;\hfill$O(\ell)$
  \item \textbf{Facteur du polynôme de $\ell$-division:} Calculer $\phi_\ell\bmod h$.\hfill$\tildO(\ell)$
  \end{itemize}
\end{frame}

%%

\begin{frame}
  \frametitle{Comment reconnaître une isogénie?}
  
  \[AU_i + TV_i = R_i  \qquad\Leftrightarrow\qquad  A\equiv \frac{R_i}{U_i} \bmod T\]
  \[\ell = 11\]
  \pause
  \begin{center}
  \begin{tabular}{c | c}
    $\deg R_i$ & $\deg U_i$ \\
    \pause
    $3141592653589793238462643$ & 0 \\
    \pause
    \alt<7>{\blue{$3141592653589793238462642$}}{$3141592653589793238462642$} & 1 \\
    \pause
    $3141592653589793238462641$ & \alt<7>{\blue{$2$}}{$2$} \\
    \pause
    \vdots & \vdots\\
    $3141592653589793238462634$ & $9$ \\
    \pause\pause
    \Huge\alert{$11$} & \Huge\alert{$10$}\\
    \pause
    $10$ & $3141592653589793238462633$\\
    \vdots & \vdots
  \end{tabular}
  \end{center}
\end{frame}

%%

\begin{frame}
  \frametitle{Isogénies de degré inconnu}
  
  \begin{itemize}
  \item Ce \emph{pattern} est extrêmement rare.
  \item Ceci est la seule partie des algorithmes de Couveignes qui dépend de $\ell$.
  \item<2-> \large En fait, cela ne dépend pas vraiment de $\ell$, mais juste de
    la présence d'un \emph{saut}.
  \item<2-> \large Si $\ell$ n'est pas connu à l'avance, il suffit d'attendre
    un \emph{saut}.
  \item<2-> \large Ainsi, toutes les isogénies de degré $\ll p^k$ peuvent être
    obtenues avec une seule exécution des algorithmes de Couveignes.
  \item<3-> \Large Mais pourquoi? Et ça sert à quoi?
  \end{itemize}  
\end{frame}

%%
%%

\section{Et maintenant?}

\begin{frame}
  \frametitle{Quelques temps d'exécution}

  \begin{figure}
  \centering
  \includegraphics[height=0.5\textwidth]{../C2-LS}
   \includegraphics[height=0.5\textwidth]{../C2-LS2}
   \caption{Comparaison des temps d'exécution de \cite{couveignes96}
     (C2, seul le temps moyen est dessiné) et \cite{lercier+sirvent08}
     (LS) sur plusieurs courbes et corps de base. Processeur Intel
     Xeon E5520 (Nehalem) @2.26GHz. Échelle logarithmique.}
  \label{fig:comp}
\end{figure}

\end{frame}

%%

\begin{frame}
  \frametitle{Marathon d'implantations}

  \begin{itemize}
  \item \texttt{FAAST}, pour l'arithmétique dans les tours
    d'Artin-Schreier, implantée en \texttt{C++} et distribuée sous
    GPL: \url{http://www.lix.polytechnique.fr/~defeo/FAAST}
  \item Implantation de \cite{couveignes96} basée sur FAAST, pas encore distribuée.
  \item Lercier-Sirvent en Magma, en attendant une version en \texttt{C++}.
  \item Lercier ($p=2$) en \texttt{C++} par F.~Morain.
  \item Librairie de calcul d'isogénies en développement, effort joint
    avec F.~Morain, É.~Schost.
  \item Portage pour SAGE\dots un jour?
  \end{itemize}
\end{frame}

%%

\begin{frame}
  \frametitle{Perspectives}
  
  \begin{block}{En quête de la complexité quasi-linéaire}
    \begin{itemize}
    \item Weierstrass a un défaut de canonicité: autres paramétrisations?
    \item Comment obtenir de l'information \emph{locale} sur le
      comportement de l'isogénie? (par exemple, son action sur $E[p]$)
    \end{itemize}
  \end{block}

  \begin{block}{Isogénies de degré inconnu}
    \begin{itemize}
    \item Testé deux courbes sur $\F_{2^{161}}$ isogènes de degré
      inconnu, prises de \cite{teske06};
    \item Certifié en 258 heures-cpu qu'il n'y a aucune isogénie de degré
      $2^c\ell$ pour $c$ quelconque et $\ell<2^{11}$;
    \item Certifié en 1195 heures-cpu qu'il n'y a aucune isogénie de
      degré inférieur à $2^{12}$.
    \item Les courbes ont une isogénie de degré (très friable) $\sim
      2^{1050}$. Prouver qu'aucune isogénie de degré plus petit existe
      est actuellement hors porté.
    \end{itemize}
  \end{block}
\end{frame}

%%

\begin{frame}
  \frametitle{Z'en voulez plus?}
  
  \begin{center}
    \Large Fast Algorithms for Towers of Finite Fields and Isogenies

    \bigskip

    \large 13 décembre, École Polytechnique

    \normalsize heure et amphi à préciser
  \end{center}
\end{frame}

%%
%%

\begin{frame}[allowframebreaks]
  \frametitle{References}

  \defbibfilter{books}{\type{book} \or \type{booklet} \or \type{thesis}
    \or \type{report} \or \type{collection} \or \type{manual}
    \or \type{periodical} \or \type{proceedings}}
  \defbibfilter{articles}{\not \(\type{book} \or \type{booklet} \or \type{thesis}
    \or \type{report} \or \type{collection} \or \type{manual}
    \or \type{periodical} \or \type{proceedings}\)}

  \beamertemplatebookbibitems
  \printbibliography[filter=books]
  \beamertemplatearticlebibitems
  \printbibliography[filter=articles]
\end{frame}

\end{document}


% Local Variables:
% mode:flyspell
% ispell-local-dictionary:"francais"
% mode:TeX-PDF
% End:
%

% LocalWords:  Isogeny abelian isogenies hyperelliptic supersingular Frobenius
% LocalWords:  isogenous
