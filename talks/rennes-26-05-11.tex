\documentclass{article}

\usepackage[american]{babel}
\usepackage[utf8]{inputenc}
\usepackage{mysymbols}
\usepackage{amsmath}
\usepackage{amssymb}
\usepackage{amsthm}
\usepackage{amsfonts}
%\usepackage{tikz}
%\usetikzlibrary{}

\newcommand{\mf}{\mathfrak}

\newtheorem{theorem}{Theorem}
\newtheorem{definition}[theorem]{Definition}
\newtheorem{proposition}[theorem]{Proposition}

\title{Notes on complex multiplication}
\author{Luca De Feo}

\begin{document}
\maketitle

\section{Introduction}


\section{Class field theory}

Let $K$ be a number field, we denote by $\0_K$ its \emph{ring of
  integers}, that is the subring of $K$ made of the numbers that
satisfy a monic equation with integer coefficients.

\begin{theorem}
  $\0_K$ is a Dedekind domain. In particular
  \begin{itemize}
  \item it is Noetherian,
  \item it is integrally closed in $K$,
  \item every non-zero prime ideal is maximal.
  \end{itemize}
\end{theorem}

The most important property of Dedekind domains is that any ideal
$\mf{a}\subset\0_K$ factors in a product of prime ideals
\begin{equation}
  \label{eq:4}
  \mf{a} = \mf{p}_1\cdots\mf{p}_r
\end{equation}
and the decomposition is unique up to reordering.

Prime ideals of $\0_K$ are also called the \emph{primes} of $K$.

\begin{definition}[Fractional ideal]
  A \emph{fractional ideal} is a non-zero finitely generated $\0_K$
  submodule of $K$.
\end{definition}

\begin{proposition}
  Any fractional $\0_K$-ideal $\mf{a}$
  \begin{itemize}
  \item is invertible, i.e.\ there exist a fractional ideal $\mf{b}$
    such that $\mf{a}\mf{b}=\0_K$;
  \item can be uniquely written as $\mf{a} =
    \alpha\prod_i^r\mf{p}_i^{r_i}$ with $\alpha\in\K$ and $\mf{p}_i$
    distinct prime ideals of $\0_K$.
  \end{itemize}
\end{proposition}

Hence fractional ideals form a group, denoted by $I_K$. 

\begin{definition}[Ideal class group]
  A fractional ideal is \emph{principal} if it is of the form
  $\alpha\0_K$ for some $\alpha\in K$. The principal fractional ideals
  form a subgroup $P_K\subset I_K$. The \emph{ideal class group}
  $C(\0_k)$ of $K$ is $P_K/I_K$.
\end{definition}

Using the \emph{Artin map}, we can prove the following.

\begin{theorem}[Hilbert class field]
  Given a number field $K$, there is a finite Galois extension $L/K$
  such that
  \begin{itemize}
  \item $L$ is an unramified Abelian extension of $K$,
  \item any unramified Abelian extension of $K$ lies in $L$,
  \item there is a canonical isomorphism $\Gal(L/K)\isom C(\0_k)$.
  \end{itemize}
  Such an $L$ is called the \emph{Hilbert class field} of $K$.
\end{theorem}

The degree $[L:K] = \card{C(\0_K)}$ is called the \emph{class number}
of $K$ and is denoted by $h(\0_K)$.


\section{Imaginary quadratic fields}

A quadratic field $K$ can be written as $K=\Q(\sqrt{N})$ where $N$ is
a squarefree integer; when $N<0$, $K$ is said to be
\emph{imaginary}. To a quadratic field $K$ is associated a
\emph{discriminant} $d_K$, which is defined to be
\begin{equation}
  \label{eq:1}
  d_k =
  \begin{cases}
    N &\text{if $N=1\bmod 4$,}\\
    4N & \text{otherwise.}
  \end{cases}
\end{equation}

Obviously $K=\Q(\sqrt{d_K})$ and its ring of integers is
\begin{equation}
  \label{eq:2}
  \0_K =
  \begin{cases}
    \Z\left[\frac{1+\sqrt{N}}{2}\right] & \text{if $N=1\bmod 4$,}\\
    \Z[\sqrt{N}] & \text{otherwise.}
  \end{cases}
\end{equation}
This can more conveniently be written
\begin{equation}
  \label{eq:3}
  \0_K = \Z\left[\frac{d_K+\sqrt{d_K}}{2}\right].
\end{equation}

\begin{definition}[Orders]
  An \emph{order} $\0$ in a quadratic field $K$ is a subring of $K$
  that is finitely generated as $\Z$-module and that contains a
  $\Q$-basis of $K$.
\end{definition}

In particular $\0\subset\0_K$ and $\0_k$ is also called the
\emph{maximal order}.

\begin{proposition}[Conductor]
  Any order $\0$ has finite index in $\0_K$. Its index $f=[\0_k:\0]$
  is called the \emph{conductor of $\0$} and we have
  \[\0 = \Z + f\0_K.\]
\end{proposition}

\begin{definition}[Discriminant]
  The \emph{discriminant} of an order $\0$ with conductor $f$ is
  \[D = f^2d_K.\]
\end{definition}

We can define fractional $\0$-ideals as we did for $\0_K$. These
ideals need not be invertible, but if we restrict to invertible ones,
we can define

\begin{definition}[Ideal class group]
  Let $\0\subset\0_K$ be an order, $I(O)$ the group of invertible
  fractional $\0$-ideals, and $P(\0)$ the group of principal
  fractional $\0$-ideals (which are obviously invertible). The
  \emph{ideal class group} of $\0$ is
  \[C(\0) = I(\0)/P(\0).\]
\end{definition}

Every ideal class group $C(\0)$ of an order of $K$ arises as the
Galois group of an Abelian extension of $K$ ramified at primes above
the conductor, called the \emph{ring class field} of $\0$.


\section{Elliptic functions}

In the meantime, people in complex analysis were trying to solve
elliptic integrals
\begin{equation}
  \label{eq:5}
  \int\frac{dt}{\sqrt{t(t-1)(t-\lambda)}}.
\end{equation}

The solutions are better expressed in terms of their inverse
functions: the elliptic functions.

\begin{definition}[Lattice]
  A \emph{lattice} $\Lambda\subset\C$ is a discrete additive subgroup
  of $\C$ that contains a basis of the $\R$-vector space $\C$. Two
  lattices $\Lambda_1,\Lambda_2$ are said to be \emph{homothetic} if
  $\Lambda_1=\alpha\Lambda_2$ for some $\alpha\in\C^{\ast}$.
\end{definition}

As a group, a lattice is isomorphic to $\Z\times\Z$. Two elements
$\omega_1,\omega_2\in\Lambda$ such that
\begin{equation}
  \label{eq:135}
  \Lambda = \omega_1\Z + \omega_2\Z
\end{equation}
are called a \emph{basis} of $\Lambda$. The quotient $\C/\Lambda$ is
called a \index{complex torus}\emph{complex torus}.

\begin{definition}[Elliptic function]
  Let $\Lambda$ be a lattice. An \emph{elliptic function} on $\Lambda$
  is a meromorphic function $f$ on $\C$ such that
  \begin{equation}
    \label{eq:141}
    f(z+\omega) = f(z)
  \end{equation}
  for any $\omega\in\Lambda$.  The set of elliptic functions on
  $\Lambda$ is denoted by $\C(\Lambda)$.
\end{definition}

An example of elliptic function is the \emph{Weierstrass
  $\wp$-function}
\begin{equation}
  \label{eq:142}
  \wp(z;\Lambda) = \frac{1}{z^2} + \sum_{\omega\in\Lambda\diffset\{0\}}\frac{1}{(z-\omega)^2}-\frac{1}{\omega^2}
  \text{.}
\end{equation}
\begin{theorem}
  Let $\Lambda$ be a lattice, then $\C(\Lambda) = \C(\wp(z),\wp'(z))$.
\end{theorem}

For $k>1$, we define the \emph{Eisenstein series} $G_{2k}$ as
\begin{equation}
  \label{eq:143}
  G_{2k}(\Lambda) = \sum_{\omega\in\Lambda\diffset\{0\}}\frac{1}{\omega^2}
  \text{.}
\end{equation}

\begin{theorem}
  The Laurent series expansion of $\wp(z)$ at $0$ is
  \begin{equation}
    \label{eq:144}
    \wp(z) = \frac{1}{z^2} + \sum_{k=1}^\infty (2k+1)G_{2k+2}z^{2k}
    \text{.}
  \end{equation}
  At any $z\not\in\Lambda$, the function $\wp(z)$ satisfies the
  differential equation
  \begin{equation}
    \label{eq:145}
    {\wp'}^2 = 4\wp^3 - 60G_4\wp - 140G_6
    \text{.}
  \end{equation}
\end{theorem}

We set $g_2(\Lambda)=60G_4(\Lambda)$ and $g_3(\Lambda)=140G_6(\Lambda)$.

A lattice uniquely determines its Weierstrass function, which in turn
can be used to express any other elliptic function. Scaling the axes
does not substantially change the $\wp$-function associated to a
lattice, and we have
\begin{equation}
  \label{eq:6}
  \wp(\lambda z; \lambda\Lambda) = \lambda^{-2}\wp(z;\Lambda).
\end{equation}

Thus it is natural to consider lattices up to homothety. The
$j$-function allows to do such a classification.

\begin{definition}[$j$-invariant]
  The $j$-invariant $j(\Lambda)$ of $\Lambda$ is
  \[j(\Lambda) = \frac{g_2(\Lambda)^3}{g_2(\Lambda)^3 - 27g_3(\Lambda)^2}.\]
\end{definition}

\begin{theorem}
  $j(\Lambda)=j(\Lambda')$ if and only if $\Lambda$ and $\Lambda'$ are
  homothetic.
\end{theorem}


\section{Complex multiplication}

It is straightforward to realise that an order $\0$ in an imaginary
quadratic field $K\subset\C$ is a lattice. More generally, any
invertible fractional $\0$-ideal can be viewed as a subset of $\C$ and
thus as a lattice.

If $\mf{a}$ is an invertible $\0$-ideal, we write $j(\mf{a})$ for the
$j$-invariant of the lattice associated to $\mf{a}$. Since we classify
lattices up to homothety, it is natural to classify fractional ideals
up to scaling too: it is then evident that two ideals are homothetic
if and only if they lie in the same ideal class.

Thus, if $\bar{\mf{a}}\in C(\0)$ is the class of $\mf{a}$,
$j(\bar{\mf{a}})$ is well-defined and we have that for any given order
$\0$ of $K$ there is a finite number of $j$-invariants arising from
it, that is exactly $h(\0)$.

But complex multiplication reaches farther, because it gives an action
of $C(\0)$ on the $j$-invariants arising from $\0$.

\begin{theorem}[Main theorem of complex multiplication]
  Let $\0$ be an order in an imaginary quadratic field $K$ and let
  $\mf{a}$ be an invertible fractional $\0$-ideal. The $j$-invariant
  $j(\mf{a})$ is an algebraic integer and $K(j(\mf{a}))$ is the ring
  class field of $\0$.
\end{theorem}

In particular, all the $j$-invariants arising from $\0$ are conjugate,
and the class group $C(\0)$ acts faithfully and transitively on them
(as the Galois group of $K(j(\mf{a}))/K$).

The polynomial
\begin{equation}
  \label{eq:7}
  H_{\0}(X) = \prod_{\bar{\mf{a}}\in C(\0)} \left(X - j(\bar{\mf{a}})\right) \in \Z[X]
\end{equation}
is the minimal polynomial of the $j$-invariants arising from $\0$ and
is called the (Hilbert) class polynomial.


\section{Elliptic curves}

\begin{theorem}
  Let $E$ be the curve 
  \begin{equation}
    \label{eq:146}
    E\;:\: y^2=4x^3-g_2(\Lambda)x-g_3(\Lambda)
    \text{.}
  \end{equation}
  The map
  \begin{equation}
    \label{eq:147}
    \begin{aligned}
      \phi: \C/\Lambda &\ra E(\C)\\
      z &\mapsto
      \begin{cases}
        \0 &\text{if $z=0$,}\\
        (\wp(z), \wp'(z)) &\text{otherwise,}
      \end{cases}
    \end{aligned}
  \end{equation}
  is an isomorphism of Riemann surfaces and a group homomorphism. 
\end{theorem}

If $\Lambda$ is a lattice, we denote by $E_\Lambda$ the elliptic curve
corresponding to it as in Eq.~\eqref{eq:146}.

\begin{theorem}
  Let $\Lambda_1,\Lambda_2$ be lattices and let $a\in\C^\ast$ such that
  $a\Lambda_1\subset\Lambda_2$. The map
  \begin{equation}
    \label{eq:148}
    \begin{aligned}
      \phi_a:\C/\Lambda_1&\ra\C/\Lambda_2\\
      z&\mapsto az
    \end{aligned}
  \end{equation}
  is holomorphic. The map $E_{\Lambda_1}\ra E_{\Lambda_2}$ induced by
  $\phi_a$ is an isogeny.
\end{theorem}

The correspondences we just defined are actually equivalences.

\begin{theorem}
  The category of elliptic curves over $\C$ with isogenies as maps is
  equivalent to the category of lattices up to homothety with maps
  $z\mapsto az$ such that $a\Lambda_1\subset\Lambda_2$.
\end{theorem}

Thus to any elliptic curve there is an unique lattice $\Lambda$
associated up to homothety; the addition law on $\C/\Lambda$ is just
addition in $\C$, and an isogeny can be obtained by an $a\in\C$ such
that $a\Lambda_1\subset\Lambda_2$.

Now, the group law of $E$ defines algebraic endomorphisms $[n]:E\to E$
for any scalar $n\in\Z$. It is natural to ask whether are other
algebraic endomorphisms exist. The answer is yes, and when this
happens we say that $E$ \emph{has complex multiplication}.

\begin{theorem}
  The endomorphism ring of an elliptic curve over $\C$ is either $\Z$
  or an order in a quadratic imaginary field. If $\alpha\in\End(E)$ is
  such an endomorphism, then
  \begin{itemize}
  \item $\wp(\alpha z)$ is a rational function in $\wp(z)$,
  \item $\alpha\Lambda \subset \Lambda$.
  \end{itemize}
\end{theorem}

A similar theorem holds for isogenies.

\section{Applications}

\begin{itemize}
\item Elliptic curves over $\F_q$.
\item CM method.
\item Isogeny volcanoes.
\end{itemize}



\end{document}


% Local Variables:
% mode:TeX-PDF
% mode:reftex
% mode:flyspell
% ispell-local-dictionary:"british"
% End:
