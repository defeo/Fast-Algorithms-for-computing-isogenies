\documentclass[10pt]{beamer}

\usepackage[english]{babel}
\usepackage[utf8]{inputenc}
\usepackage{amssymb}
\usepackage{amsthm}
\usepackage[all]{xy}
\usepackage{graphicx}
\usepackage{tikz}
\usetikzlibrary{trees}

%% Stuff
\renewcommand{\le}{\leqslant}
\renewcommand{\ge}{\geqslant}  % comme François le demande...
\newcommand{\blue}[1]{\textcolor{blue}{#1}}  % colouring
%% Algèbre
\newcommand{\clot}[1]{\bar{#1}}  % clôture algèbrique
\newcommand{\card}[1]{\# #1}  % cardinalité d'un ensemble
\DeclareMathOperator{\car}{char}  % caractéristique d'un corps
\DeclareMathOperator{\Frac}{Frac}  % corps des fractions
\newcommand{\Z}{\mathbb{Z}}  % les entiers
\newcommand{\K}{\mathbb{K}}  % un corps
\newcommand{\LK}{\mathbb{L}}  % encore un corps
\newcommand{\U}{\mathbb{U}}  % encore un corps
\newcommand{\F}{\mathbb{F}}  % un corps fini
\newcommand{\Q}{\mathbb{Q}}  % les rationnels
\newcommand{\R}{\mathbb{R}}  % les réels
\newcommand{\C}{\mathbb{C}}  % les complexes
\newcommand{\isom}{\cong}  % isomorphisme de corps
\newcommand{\frob}{\varphi}  % fröbenius
\DeclareMathOperator{\Gal}{Gal}  % groupe de Galois
\DeclareMathOperator{\Tr}{Tr}  % trace
\DeclareMathOperator{\PTr}{PTr}  % pseudotrace
\DeclareMathOperator{\Norm}{N} % norme
\newcommand{\euler}{\phi}  % indicatrice d'Euler
\DeclareMathOperator{\ord}{ord}  % l'ordre d'un élément
\newcommand{\AS}[1]{\mathcal{#1}}  % la police des polynômes d'AS
\DeclareMathOperator{\rev}{rev}  % le reverse d'un polynôme
%% Courbes
\DeclareMathOperator{\Jac}{Jac}  % la jacobienne
\newcommand{\Proj}{\mathbb{P}}  % espace projectif
\newcommand{\0}{\mathcal{O}}  % point de base d'une courbe
\newcommand{\ecpoint}[3]{[#1:#2:#3]}  % un point d'une courbe
\newcommand{\isog}[1]{\mathcal{#1}}  % la police des isogénies
\newcommand{\I}{\isog{I}}  % une isogénie I
\newcommand{\Hasse}{H}  % l'invariant de Hasse
\newcommand{\divpol}{f}  % polynôme de division
%% Autre
\newcommand{\tildO}{\tilde{O}}  % la notation O~ qui oublie les log
\newcommand{\Mint}{\mathrm{\sf M}_\text{int}}  % fonction de multiplication
\newcommand{\Mpol}[1][]{\mathrm{\sf M}_\text{pol}^{#1}}  % fonction de multiplication
\newcommand{\Mult}[1][]{\mathrm{\sf M}_{#1}}  % fonction de multiplication
\newcommand{\Push}{\mathrm{\sf P}}  % fonction de push-down
\newcommand{\Lift}{\mathrm{\sf L}}  % fonction de lift-up
\newcommand{\Trace}{\mathrm{\sf T}}  % fonction de trace
\newcommand{\Frob}{\mathrm{\sf F}}  % fonction de frobenius itéré
\newcommand{\Ptr}{\mathrm{\sf PT}}  % fonction de pseudo-trace
\newcommand{\ModComp}{\mathrm{\sf C}}  % fonction de composition modulaire
\newcommand{\alg}[1]{\textsf{#1}}  % la police des algorithmes
\newcommand{\wrt}{\dashv}  % appartenance forte, a\wrt A signifie que a est représenté comme un élément de A
\DeclareMathOperator{\op}{op}  % une opération

\newenvironment{algorithm}[3]{\begin{center}\begin{minipage}{0.85\textwidth}
      \sf
      \rule{\textwidth}{0.2pt}\\
      \makebox[\textwidth][c]{\textbf{#1}}\\
      \rule[0.5\baselineskip]{\textwidth}{0.2pt}\\
      \textbf{Input~~} #2\\
      \textbf{Output} #3
      \smallskip
      \begin{enumerate}
}{\end{enumerate}
      \rule{\textwidth}{0.2pt}
\end{minipage}\end{center}}

% beamer-specific
%\setbeamertemplate{theorem begin}{
%  \begin{\inserttheoremblockenv}{
%      Théorème
%      \ifthenelse{\equal{\inserttheoremaddition}{}}
%		 {}
%		 {(\inserttheoremaddition)}    
		 %  \ifx\inserttheoremaddition\@empty\else\ (\inserttheoremaddition)\fi
%    }
%}


\title{Couveignes' surprise}
\author[L.~De~Feo]{L.~De~Feo}
\institute[TANC, LIX]{INRIA Projet TANC \& LIX École Polytechnique}
\date[ANTS IX, Nancy, July 20, 2010]{July 20, 2010\\ANTS IX, LORIA, Nancy}


% \AtBeginSection[]
% {
%   \begin{frame}<beamer>
%     \frametitle{Plan}
%     \tableofcontents[currentsection]
%   \end{frame}
% }


\begin{document}

\begin{frame}
  \titlepage
\end{frame}

%%

\begin{frame}
  \frametitle{Couveignes' algorithm}
  
  \begin{itemize}
  \item<1-> Compute an isogeny
    \[\xymatrix{E \ar[r]^{\I} & E'}\]
    $\I$ has degree $\ell$, $p$ is small.
  \item<2-> Interpolate (the abscissae)
    \[\xymatrix{E[p^k] \ar[r]^{\I} & E'[p^k]}\]
    for $k$ big enough \alert{($p^k \gg \ell$)}.
  \item<3-> How? Pick up $P\in E[p^k]$ and $P'\in P[p^k]$ and make a guess
    \[\xymatrix{P \ar@{|->}[r]^{\I} & P'}\]
  \item<4-> If it doesn't work, pick another
    $P$'. \uncover<5->{\alert{Again and again...}}
  \end{itemize}  
\end{frame}

\begin{frame}
  \frametitle{How to know when a polynomial is an isogeny?}
  
  \[T = \prod_n (X - [n]P)\]
  \[A_{P'} : [n]P \mapsto [n]P'\]

  \begin{itemize}
  \item<2-> If $A_{P'}$ corresponds to an isogeny
    \[A_{P'}(X) \equiv \alert<3>{\frac{g(X)}{h(X)}} \bmod T(X)\] with
    \alert{$\deg g = \ell$} and \alert{$\deg h = \ell-1$} and $h$ is a
    square.
  \item<3-> Compute $\frac{g}{h}$ by rational fraction
    reconstruction. This is just
    \[\alert{\mathrm{XGCD}(A_{P'},T)}\]
  \end{itemize}
\end{frame}

\begin{frame}
  \frametitle{What happens when you do the XGCD for an isogeny?}

  \[AU_i + TV_i = R_i  \qquad\Leftrightarrow\qquad  A\equiv \frac{R_i}{U_i} \bmod T\]

  \pause

  \begin{center}
  \begin{tabular}{c | c}
    $\deg R_i$ & $\deg U_i$ \\
    \pause
    $n$ & 0 \\
    \pause
    \alt<7>{\huge $n-1$}{$n-1$} & 1 \\
    \pause
    $n-2$ & \alt<7>{\huge $2$}{$2$} \\
    \pause
    \vdots & \vdots\\
    $n-\ell+2$ & $\ell-2$ \\
    \pause\pause
    \Huge\alert{$\ell$} & \Huge\alert{$\ell-1$}\\
    \pause
    $\ell-1$ & $n-\ell+1$\\
    \vdots & \vdots
  \end{tabular}
  \end{center}
\end{frame}

\begin{frame}
  \frametitle{Getting all the isogenies}

  \begin{itemize}
  \item<1-5> This pattern is extremely rare.
  \item<2> \large This is the only part of the computation that depends on $\ell$.
  \item<3> \Large It actually doesn't really depend on $\ell$, just on the
    \alert{leap}.
  \item<4> \huge If we don't know $\ell$ in advance, we just wait for
    a leap.
  \item<5-> \alt<6>{\Huge But Why?\\[2ex] And What is this useful
      for?}{\Huge Actually, we can just look at all the gaps and we
      get all the isogenies with degree $\ll p^k$.}
  \end{itemize}
\end{frame}

\end{document}


% Local Variables:
% mode:flyspell
% ispell-local-dictionary:"british"
% mode:TeX-PDF
% End:
%

% LocalWords:  Isogeny abelian isogenies hyperelliptic supersingular Frobenius
% LocalWords:  isogenous
