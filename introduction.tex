\section{Introduction}

The problem of computing an explicit degree $\ell$ isogeny between two
given elliptic curves over $\F_q$ was originally motivated by point
counting methods based on Schoof's algorithm \cite{Atk91},
\cite{Elk91}, \cite{Sch95}. A review of the most efficient algorithms
to solve this problem is given in \cite{BoMoSaSc08} together with a
new quasi-optimal algorithm; however, all the algorithms presented in
\cite{BoMoSaSc08} compute the power series expansion of the
Weierstrass functions of the curves up to a certain precision, thus
they are limited to the case $\ell\ll p$ where $p$ is the
characteristic of $\F_q$. This is satisfactory for cryptographic
applications where one takes $p=q$ --and in this case Schoof's
algorithm needs $\ell\in O(\log p)$-- or alternatively $p=2$ --and in
this latter case $p$-adic methods based on \cite{Sat00} are preferred
to Schoof's algorithm--.

Nevertheless, the problem of computing explicit isogenies in the case
where $p$ is small compared to $\ell$ stays of theoretical interest
and can find practical applications in newer cryptosystems such as
\cite{Tes06}, \cite{RoSt06}. The first algorithm to solve this problem
was given by Couveignes using formal groups \cite{Cou94}; it uses
$\tildO(\ell^3\log q)$ operations in $\F_p$ assuming $p$ is constant,
however it has an exponential complexity in $\log p$. Another
algorithm by Lercier specific to $p=2$ uses some linear properties of
the problem to build a linear system from whose solution the isogeny
can be deduced \cite{Ler96}; its complexity is conjectured to be
$\tildO(\ell^3\log q)$ operations in $\F_p$, but with a much better
constant factor than \cite{Cou94}. At the present moment the latter
algorithm is by many orders of magnitude the fastest algorithm to
solve practical instances of the problem when $p=2$, thus being the
\emph{de facto} standard for cryptographic use.

$p$-adic methods have been used to solve the isogeny problem by Joux
and Lercier \cite{JL06} and Lercier and Sirvent \cite{LeSi09}. The
former has complexity $\tildO(\ell^2(1 + \ell/p)\log q)$ operations in
$\F_p$, which makes it well adapted to the case where $p\sim\log
q$. The latter has complexity $\tildO(\ell^3 + \ell\log q^2)$
operations in $\F_p$, making it the best algorithm to our knowledge
for the case where $p$ is not constant.

\subsection{The algorithm C2 and its variants}

Finally, the algorithm having the best asymptotic complexity in $\ell$
was proposed again by Couveignes in \cite{Cou96}, we will refer to
this basic version as ``C2''. Its complexity --supposing $p$ is
fixed-- was estimated in \cite{Cou96} as being $\tildO(\ell^2\log q)$
operations in $F_p$, but with a precomputation step requiring
$\tildO(\ell^3\log q)$ and large memory requirements. However this
estimate was wrong as we will argue in Section \ref{sec:}
and C2 has an overall asymptotic complexity of $\tildO(\ell^3\log q)$
operations.

Subsequent work by Couveignes used Artin-Schreier theory to avoid the
precomputation step of C2, drop the memory requirements to
$\tildO(\ell\log q + \log^2 q)$ elements of $\F_p$ \cite{Cou00} and
solve the issue that made the estimate for the complexity of C2
false. In conclusion this variant has complexity $\tildO(\ell^2\log
q)$, we refer to it as ``C2-AS''.

However it has been recently shown in \cite{DFS09} that \cite{Cou00}
laid over some false assumptions on how fast arithmetics can be
performed in a tower of Artin-Schreier extensions. \cite{DFS09} fixes
the issue by giving asymptotically fast algorithms for Artin-Schreier
towers, but this is not enough to patch C2-AS as we will argue in
Section \ref{sec:}, bringing back C2-AS in the complexity class
$\tildO(\ell^3\log q)$.

The aim of this paper is to give a complete review of Couveignes'
algorithm and its variants, provide the missing piece to make it have
a quadratic dependence in $\ell$ and prove the accurate complexity
estimate of $\tildO()$ operations in $\F_p$. We also \dots

\subsection{Notation and plan}

In the rest of the paper $p$ is a prime, $d$ a positive integer,
$q=p^d$ and $\F_q$ is the field with $q$ elements. For an elliptic
curve $E$ and a field $\K$ with algebraic closure $\clot{\K}$, we note
by $E(\K)$ the set of $K$-rational points and by $E[m]$ the
$m$-torsion subgroup of $E(\clot{\K})$. The group law on the elliptic
curve is noted additively, for a point $P$ we note by $x(P)$ its
abscissa and by $y(P)$ its ordinate, we note by $\0$ the point at
infinity. We will restrict ourselves to the case of ordinary elliptic
curves, thus $E[p^k]\isom\Z/p^k\Z$.

Unless otherwise stated, all time complexities will be measured in
number of operations in $\F_p$ and all space complexities in number of
elements of $\F_p$. We use the notation $\tildO_x$ that forgets
polylogarithmic factors in the variable $x$, thus $O(xy\log x \log
y)\subset\tildO_x(xy\log y)$. We simply note $\tildO$ when the
variable $x$ is clear from the context. We note by $\Mult(d)$ the
complexity of multiplying two polynomials of degree at most $d$ with
coefficients in $\F_p$, typical values are $O(d^2)$ for naive
multiplication or $O(d\log d\log\log d)$ for FFT multiplication.

The plan is the following.



% Local Variables:
% mode:flyspell
% ispell-local-dictionary:"british"
% End:
%
% LocalWords:  Schreier Artin pseudotrace frobenius bivariate Joux Sirvent FFT
% LocalWords:  Couveignes isogenies Schoof isogeny cryptosystems Lercier
% LocalWords:  precomputation arithmetics polylogarithmic
