\section{The algortihm C2-AS-FI}
\label{sec:C2-AS-FI}

The most expensive step of C2-AS is the polynomial interpolation step
which is part of the Cauchy interpolation. If we use a standard
interpolation algorithm, its input consists in a list of $\Theta(p^k)$
pairs $\bigl(P, \I(P)\bigr)$, with $P$ having coordinates in $\U_k$,
thus a lower bound for any such algorithm is $\Omega(p^{2k}d)$. Notice
however that the output is a polynomial of degree $\Theta(p^k)$ in
$\F_q[X]$, hence, if supplied with a shorter input, an \emph{ad hoc}
algorithm could reach the bound $\Omega(p^kd)$.

In this section we give an algorithm that reaches this bound up to
some logarithmic factors. It realizes the polynomial interpolation on
the primitive points of $E[p^k]$, thus its output is a degree
$\euler(p^k)/2-1$ polynomial in $\F_q[X]$. Using the Chinese remainder
theorem it is straightforward to generalise this to an algorithm,
having the same asymptotic complexity, that realizes the polynomial
interpolation on all the points of $E[p^k]$. We call C2-AS-FI the
variant of C2-AS resulting from applying this new algorithm.


\subsection{The algorithm}
Let $P\in E[p^k]$ and $P'\in E'[p^k]$ be primitive $p^k$ torsion
points. We want to compute the smallest degree polynomial
$A\in\F_q[X]$ such that
\begin{equation}
  A\bigl(x\bigl([n]P\bigr)\bigr) = x\bigl([n]P'\bigr)
  \quad\text{for any $n\in\left(\Z/p^k\Z\right)^\ast$.}
\end{equation}
To be more precise, we want to compute the canonical representative of
$A$ in $\F_q[X]/T(X)$, where
\begin{equation}
  T(X) = \prod_{n\in\left(\Z/p^k\Z\right)^\ast} \bigl(X - x([n]P)\bigr)
  \text{.}
\end{equation}
 
There are two equivalent ways to look at this problem. The first one
is as the interpolation problem we just stated. The second one is as
an isomorphism of finite fields problem. Both viewpoints will be
important.

For notational convenience, we set $\U_0=\F_q$.  Let

\begin{equation}
  \label{eq:T}
  T = \prod T^{(j)}
\end{equation}
be the factorisation of $T$ over $\U_0$, and set
\begin{equation}
  \label{eq:A}
  A^{(j)} = A \bmod T^{(j)}
  \;\text{.}
\end{equation}

It was already pointed out in \cite[$\S$2.3]{Cou96} that, knowing the
factorisation of $T$ over $\U_0$ and all the $A^{(j)}$'s, we can
recover $A$ using the Chinese remainder theorem. Thus we will focus on
computing, say, $A^{(0)}$.

$T^{(0)}$ is irreducible over $\F_q$. Chose any root of $T^{(0)}$,
without loss of generality we can take $x(P)$ to be such a root.  Fix
the $\F_q$-linear embedding of finite fields
\begin{equation}
  \label{eq:embed}
  \xymatrix{
    ^{\F_q[X]}/_{T^{(0)}(X)} \ar@{^{(}->}[r]^-\iota & \U_k
  }
\end{equation}
given by $\iota(X) = x(P)$. It is evident that
\begin{equation}
  \iota\bigl(A^{(0)}(X)\bigr) = A^{(0)}\left(\iota(X)\right)=x\bigl(P'\bigr)
  \text{,}
\end{equation}
thus in order to compute $A^{(0)}$ one just needs to compute
$\iota^{-1}\bigl(x(P')\bigr)$.

This is a classic problem in computer algebra: given an algebraic
extension $\LK/\K$ and elements $x\in\LK$ and $y\in\K[x]$, find the
minimal polynomial $Q$ of $x$ over $\K$, identify $\K[X]/Q(X)$ to
$\K[x]$ and find the canonical image of $y$ in $\K[X]/Q(X)$. The
fastest techniques available are \cite{Shoup99,PS06}, which are largely
used in \cite{DFS09}. However, they both require to solve a
\emph{power projection} problem, that is, given a $\K$-linear form
$\ell$, compute
\begin{equation}
  \ell(1), \ell(x), \ell(x^2), \dots, \ell(x^n)
\end{equation}
up to some bound $n$. As explained in \cite{Shoup99}, by the
\emph{transposition principle}, solving the power projection has the
same complexity as computing $g(x)$ given $g\in\K[X]$.

But, in our specific case, the extension we work with is $\U_k/\F_q$ and
we know (for the moment) no other method to evaluate a polynomial
$g\in\F_q[X]$ on a point of $\U_k$, other than lift $g$ in $\U_k[X]$
and evaluate by Horner rule. Unfortunately, this is too expensive,
thus we will study an alternative approach that amounts to decompose
$\iota$ as a chain of morphisms and invert them one-by-one going down
in the tower $(\U_0,\U_1,\ldots,\U_k)$. This is similar to the way
\cite{Cou00} solves an Artin-Schreier equation by moving it down from
$\U_k$ to $\U_1$.


\paragraph{Interpolation in towers of extensions}
We switch back to the interpolation viewpoint. The algorithm we give
here can be applied in any tower of cyclic extensions, provided the
action of the Galois groups can be computed. However we will present
it only for our specific tower $(\U_0,\dots,\U_k)$, to avoid adding
unnecessary notation.

Consider the following problem: given elements $x,y\in\U_k$ such that
$x$ generates $\U_k$ over $\F_q$, find a polynomial $A\in\F_q[X]$ such
that
\begin{equation}
  \label{eq:affine-minimal}
  A(x) = y
  \text{.}
\end{equation}
Let $A$ be such a polynomial and let $T$ be the minimal polynomial of
$x$ over $\F_q$, then it is evident that $A+T$ satisfies
\eqref{eq:affine-minimal}. Hence, we can look for a representative of
minimal degree of the class of $A$ in $\F_q[X]/T(X)$. If one such
class exists, then it is unique, because otherwise $x$ would be root
of a polynomial in $\F_q[X]$ of degree smaller than $T$. 

Let $A$ be a polynomial satisfying \eqref{eq:affine-minimal} it is
clear that $A(\sigma(x)) = \sigma(y)$ for any
$\sigma\in\Gal(\U_k/\F_q)$. Conversely, the polynomial interpolating
$\sigma(x)$ over $\sigma(y)$ for any $\sigma$ is invariant under
$\Gal(\U_k/\F_q)$, thus it has coefficients in $\F_q$. Hence we can
construct $A$ by interpolation.

A fast interpolation algorithm as in \cite[10.1-2]{vzGG} would compute
$T$ via a binary subproduct tree, and then interpolate $A$ recursively
applying the Chinese remainder theorem along the branches of the
tree. However this is too expensive. We can do better by using a
non-binary subproduct tree on which the tower of Galois groups
associated to $(\U_0,\ldots,\U_k)$ acts.

First we need to compute $T$. Let $T_i$ be the minimal polynomial of
$x$ over $\U_i$, it is computed recursively as
\begin{align}
  T_k &= (X - x)\text{,}\\
  \label{eq:minprod}
  T_i &= \prod_{\sigma\in\Gal(\U_{i+1}/\U_i)}T_{i+1}^\sigma\text{.}
\end{align}
Then $T=T_0$. Observe that, rather than computing a whole subproduct
tree of $T$, we have only computed one branching as shown in figure
\ref{fig:tree}.

\begin{figure}[tb]
  \centering
  
  \begin{tikzpicture}
    \begin{scope}
      [level distance=1cm]
      \node{$\U_0$}[grow'=up]
      child {node {$\U_1$}
        child {node {$\U_2$}
          child {node {$\U_3$}}
        }
      };
    \end{scope}    
  \end{tikzpicture}
  % 
  \hfill
  %
  \begin{tikzpicture}
    \begin{scope}
      [level distance=1cm,
      level/.style={sibling distance=6cm/#1},
      nc/.style={gray}]
      \node{$T$}[grow'=up]
      child {node {$T_1$}
        child {node {$T_2$}
          child {node {$T_3$}}
          child {node {$T_3^{\sigma^4}$}}
        }
        child {node {$T_2^{\sigma^2}$}
          child[nc] foreach \l in {2,6} {node {$T_3^{\sigma^{\l}}$} edge from parent[dashed]}
        }
      }
      child {node {$T_1^\sigma$}
        child[nc] {node {$T_2^\sigma$} edge from parent[dashed]
          child[nc] foreach \l in {,5} {node {$T_3^{\sigma^{\l}}$} edge from parent[dashed]}
        }
        child[nc] {node {$T_2^{\sigma^3}$} edge from parent[dashed]
          child[nc] foreach \l in {3,7} {node {$T_3^{\sigma^{\l}}$} edge from parent[dashed]}
        }
      };
    \end{scope}
  \end{tikzpicture}
  
  \caption{The subproduct tree of $T$, in the case of a tower of
    quadratic extensions. Any generator of $\Gal(\U_3/\U_0)$ can be
    taken as $\sigma$. We gray out the nodes that the algorithm does
    not compute.}
  \label{fig:tree}
\end{figure}



Now we compute recursively the polynomials in $A_i\in\U_i[X]$ such
that $A_i(x)=y$. We start from $A_k=y$. Suppose $A_{i+1}$ is known,
then we apply the Chinese remainder algorithm of \cite[$\S$10.3]{vzGG}
to compute the polynomial $P\in\U_{i+1}[X]/T_i(X)$ such that
\begin{equation}
  \label{eq:crt}
  P \equiv A_{i+1}^\sigma \bmod T_{i+1}^\sigma
  \qquad\text{for any $\sigma\in\Gal(\U_{i+1}/\U_i)$.}
\end{equation}
It is clear that $P$ is invariant under $\Gal(\U_{i+1}/\U_i)$, hence
$P\in\U_i[X]/T_i(X)$ and by \eqref{eq:crt} it is evident that
$P(x)=A_{i+1}(x)=y$, thus $P=A_i$.

We have thus succeeded in interpolating $A=A_0$, without having to
build the whole subproduct tree. A similar algorithm was already given
in \cite{EnMo03}.


\paragraph{Back to our problem}
It is easy to realise that, on inputs $x(P)$ and $x(P')$, the
algorithm we just gave computes $A^{(0)}$. In fact, $T^{(0)}$ is
the minimal polynomial of $x(P)$ over $\F_q$ and $A^{(0)}$ is the
unique polynomial in $\F_q[X]/T^{(0)}(X)$ that satisfies
\eqref{eq:affine-minimal}.

This can be viewed as decomposing $\iota$ as the chain of
$\F_q$-linear isomorphisms
\begin{equation}
  \xymatrix{
    ^{\U_0[X_0]}/_{T_0(X_0)} \ar@{^{(}->}[r]^-{\iota_0} &
    \;\cdots\; \ar@{^{(}->}[r]^-{\iota_{k-1}} &
    ^{\U_k[X_k]}/_{T_k(X_k)} \ar@{^{(}->}[r]^-{\iota_k} &
    \U_k
  }
\end{equation}
defined by $\iota_k\circ\cdots\circ\iota_i(X_i) = x(P)$ for any $i$,
and then finding the preimage of $x(P')$ by inverting them one by
one.

Then, the Chinese remainder step we applied in \eqref{eq:crt} amounts
to invert $\iota_i$ by descending the lower path in the diagram below
\begin{equation}
  \xymatrix{
    ^{\U_i[X_i]}/_{T_i(X_i)} \ar@{^{(}->}[r]^-{\iota_i} \ar@{^{(}->}[d]^{\varepsilon} &
    ^{\U_{i+1}[X_{i+1}]}/_{T_{i+1}(X_{i+1})} \\
    ^{\U_{i+1}[Y]}/_{T_i(Y)} \ar@{^{(}->>}[r]^-{\gamma} &
    \bigoplus_\sigma {}^{\U_{i+1}[Y_{j}]}/_{\left(T_{i+1}\right)^\sigma(Y_{j})} \ar@{->>}[u]_{\pi}
  }
\end{equation}
where $\varepsilon$ is the canonical injection extending
$\U_i\subset\U_{i+1}$, $\gamma$ is the Chinese remainder isomorphism
and $\pi$ is projection onto the first coordinate.

Some care must be taken when $x(P)$ does not generate $\U_k$, but only
a subfield of index $2$. This happens when $c\not\in\F_q[c^2]$, and in
this case $\iota_0$ is not a field isomorphism. It is not to
difficult, however, to handle this case, as one only needs to take a
subgroup of index $2$ of $\Gal(\U_1/\U_0)$, instead of the whole
group, in the interpolation algorithm given above.

Observe that we could have used a different approach: after $T^{(0)}$
has been computed by \eqref{eq:minprod}, a polynomial $g\in\F_q[X]$
can efficiently be evaluated at $x(P)$ by successively lifting in
$\U_i$ and reducing modulo $T_i$ for $i=1,\ldots,k$. Thus, as noted
before, by the transposition principle we also have an efficient
algorithm to compute the power projection on $x(P)$; hence we can
apply \cite{PS06} to efficiently find $A^{(0)}$. However, this
approach cannot improve the overall complexity as it will be clear
in the next section.


\subsection{Complexity analysis}
\label{sec:C2-AS-FI:complexity}

The two algorithms for computing $T^{(0)}$ and $A^{(0)}$ are very
similar and run in parallel. We can merge them in one unique
algorithm.

We set some notation. Let $i_0$ be the largest index such that
$\U_{i_0} = \U_1$ and let $\frac{p-1}{2r} = [\F_q[c^2]:\F_q]$.  Remark
that all the $T^{(j)}$'s have degree $\frac{\euler(p^{k-i_0+1})}{2r}$.
At each level $i\ge i_0$, it does the following

\begin{enumerate}
\item for $\sigma \in \Gal(\U_{i+1}/\U_i)$, compute
  \begin{enumerate}
  \item\label{alg:T:gal} $\left(T_{i+1}\right)^\sigma$ and
  \item\label{alg:A:gal} $\left(A_{i+1}\right)^\sigma$ using
    \cite[\alg{IterFrobenius}]{DFS09},
  \end{enumerate}
\item\label{alg:T:prod} compute $T_i$ by \eqref{eq:minprod}
  through a subproduct tree as in \cite[Algo. 10.3]{vzGG},
\item\label{alg:A:CRA} compute $A_i$ by \eqref{eq:crt} through
  Chinese Reminder Algorithm \cite[Algo. 10.16]{vzGG},
\item\label{alg:T:push} convert $T_i$ and $A_i$ into
  elements of $\U_i[X]$ using \cite[\alg{Push-down}]{DFS09}.
\end{enumerate}

Steps \ref{alg:T:gal} and \ref{alg:A:gal} are identical. Both are
repeated $p$ times, each iteration taking $O\bigl(p^{k-i}{\sf
  L}(i-i_0)\bigr) \subset O\bigl({\sf L}(k-i_0)\bigr)$ by
\cite[Th. 17]{DFS09}.

Step \ref{alg:T:prod} takes $O\bigl(\Mult(p^{k-i_0+1}d/r)\log p\bigr)$
by \cite[Lemma 10.4]{vzGG} and step \ref{alg:A:CRA} has the same
complexity by \cite[Coro. 10.17]{vzGG}.

Step \ref{alg:T:push} takes $O\bigl(p^{k-i+1}{\sf L}(i-i_0)\bigr)
\subset O\bigl(p{\sf L}(k-i_0)\bigr)$.

When $i=0$ and $\U_1\ne\F_q$ the algorithm is identical but steps
\ref{alg:T:gal} and \ref{alg:A:gal} must be computed through a generic
Frobenius algorithm (using~\cite[Algorithm 5.2]{vzGS92}, for example)
and step \ref{alg:T:push} must use the implementation of $F_q[c]$ to
make the conversion (for example, linear algebra). In this case steps
\ref{alg:T:gal} and \ref{alg:A:gal} cost
$\Theta\bigl(\frac{p^{k-i_0}}{r}\ModComp(pd)\log d \bigr)$
by~\cite[Lemma 5.3]{vzGS92} and step \ref{alg:T:push} costs
$\Theta\bigl(p^{k-i_0}(pd)^2\bigr)$.

The total cost of the algorithm is then
\begin{equation*}
  \label{eq:T:complexity}
  O\left(\bigl(k-i_0\bigr)\bigl(p{\sf L}(k-i_0) + \Mult(p^{k-i_0+1}d/r)\log p\bigr) +
    \frac{p^{k-i_0}}{r}\bigl(\ModComp(pd)\log d + r(pd)^2\bigr) \right)
  \;\text{.}
\end{equation*}


\paragraph{The complete interpolation}
We compute all the $A^{(j)}$'s using this algorithm; there's
$p^{i_0-1}r$ of them. We then recombine them through a Chinese
remainder algorithm at a cost of $O\bigl(\Mult(p^kd)\log
p^{i_0-1}r\bigr)$. The total cost of the whole interpolation phase is
then
\begin{equation*}
  O\left(\bigl(k-i_0\bigr) \bigl(p{\sf L}(k) + \Mult(p^kd)\log p\bigr) +
    p^{k-1}\ModComp(pd)\log d + p^{k-1}r(pd)^2 + i_0\Mult(p^kd)\log p
  \right)
  \;\text{,}
\end{equation*}
that is
\begin{equation}
  \label{eq:interp}
  O\left(p{\sf L}(k)\log\left(\frac{\ell}{p^{i_0}}\right) + 
    \Mult(\ell d)\log\ell\log p +
    \frac{\ell}{p}\ModComp(pd)\log d +
    \ell (pd)^2
  \right)
  \;\text{.}
\end{equation}

Alternatively, once $A^{(0)}$ is known, one could compute the other
$A^{(j)}$'s using modular composition with the multiplication maps
of $E$ and $E'$ as suggested in \cite{Cou96}. However this approach
doesn't give a better asymptotic complexity because in the worst case
$A^{(0)}=A$. From a practical point of view, though, Brent's and
Kung's algorithm for modular composition \cite{BrKu78}, despite having
a worse asymptotic complexity, could perform faster for some set of
parameters. We will discuss this matter in Section
\ref{sec:C2-AS-FI-MC}.

If more than $\euler(p^k)/2$ points are needed, but less than
$\frac{p-1}{2}$, one can use the previous algorithm to interpolate
over the primitive $p^i$-torsion points for each $i=1,\ldots,k$. The
interpolating polynomials can then be recombined through a Chinese
remainder algorithm at a cost of $O\bigl(\Mult(p^kd)\log p^k\bigr)$,
which doesn't change the overall complexity of C2-AS-FI.


Putting together the complexity estimates of C2-AS and C2-AS-FI, we
have the following theorem.

\begin{theorem}
  \label{th:complexity}
  Assuming $\Mult(n) = n\log n\log\log n$, the algorithm C2-AS-FI has
  worst case complexity
  \begin{equation*}
    \tildO_{p,d,\log\ell}\left(
      p^2d^3 +
      \ModComp(p)pd +
      (pd)^\omega\log^2\ell +
      p^3\ell^2 d\log^3\ell + 
      p^2\ell^2 d^2+
      \left(\frac{\ell^2}{p} + p\right)\ModComp(pd)
    \right)
    \;\text{.}
  \end{equation*}
\end{theorem}



% Local Variables:
% mode:flyspell
% ispell-local-dictionary:"british"
% mode:TeX-PDF
% TeX-master: "ec-isogeny"
% End:
%
% LocalWords:  Schreier Artin pseudotrace Frobenius bivariate Joux Sirvent FFT
% LocalWords:  Couveignes isogenies Schoof isogeny cryptosystems Lercier moduli
% LocalWords:  precomputation arithmetics polylogarithmic Karatsuba embeddings
% LocalWords:  irreducibility
